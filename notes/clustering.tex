\section{Clustering}
\only<article>{Clustering is the problem of automatically seggregating data of different types into clusters. When the goal is \alert{anomaly detection}, then there are typically two clusters. When the goal is \alert{compression} or \alert{auto-encoding} then there are typically as many clusters as needed for sufficienlty good accuracy.}


\begin{frame}
\frametitle{Clusters as latent variables}
  \begin{figure}[H]
    \centering
    \begin{tikzpicture}
      \node[RV] at (2,0) (data) {$x_t$};
      \node[RV,hidden] at (0,0) (cluster) {$c_{t}$};
      \draw[->] (cluster)--(data);
      \uncover<2>{
        \node[RV,hidden] at (0,1) (param) {$\param$};
        \draw[->] (param)--(cluster);
        \draw[->] (param)--(data);
      }
    \end{tikzpicture}
    \label{fig:bn-cluster}
    \caption{Graphical model for independent data from a cluster distribution.}
  \end{figure}
  \begin{block}{The clustering distribution}
    \only<presentation>{
      \begin{itemize}
      \item Cluster $c_t$
      \item Observation $x_t$
      \item Parameter $\param$.
      \end{itemize}
    }
    \only<article>{The learning problem is to estimate the parameter $\param$ describing the distribution of observations $x_t$ and clusters $c_t$.}
    \[
    x_t \mid c_t = c, \param \sim P_\param(x | c), \qquad c_t \mid \param \sim P_\param(c)
    \]
    \only<article>{
      Given a parameter $\param$, the clustering problem is to estimate the probability of each cluster for each new observation.
    }
    \[
    P_\param(c_t \mid x_t) = 
    \uncover<2>{\frac{P_\param(x_t \mid c_t) P_\param(c_t)}{\sum_{c'}P_\param(x_t \mid c_t = c') P_\param(c_t = c')}}
    \]
  \end{block}
\end{frame}

\begin{frame}
  \frametitle{Bayesian formulation of the clustering problem}
  \begin{itemize}
  \item Prior $\bel$ on parameter space $\Param$.
  \item Data $x^T = x_1, \ldots, x_T$.
  \item Posterior $\bel(\cdot \mid x^T)$.
  \end{itemize}
  \begin{block}{Marginal posterior prediction}
    \[
    P_\bel(c_t \mid x_t, x^T) = 
    \int_\Param P_\param(c_t \mid x_t) \dd \bel(\param \mid x^T)
    \]
  \end{block}
\end{frame}

\begin{frame}
  \frametitle{Example for clustering}
  
\end{frame}
%%% Local Variables:
%%% mode: latex
%%% TeX-engine: xetex
%%% TeX-master: "notes.tex"
%%% End:
