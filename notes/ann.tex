\section{Classification with stochastic gradient descent}
\only<presentation>{
  \begin{frame}
    \tableofcontents[ 
    currentsection, 
    hideothersubsections, 
    sectionstyle=show/shaded
    ] 
  \end{frame}
}

\begin{frame}
  \frametitle{Linear classifiers.}
  \only<article>{Finding the optimal policy for our belief $\bel$ is not normally very difficult. However, it requires that we maintain the complete distribution $\bel$ and that we also nder some probability distribution $P$. In simple decision problems, e.g. where the set of actions $\CA$ is finite, it is possible to do this calculation on-the-fly. However, the policies that we wish to find might be much simpler than the Bayes-optimal policy. For example, we might consider linear classifiers.}

  \begin{definition}{Linear classifier}
    \only<article>{A linear classifier is parametrised by the matrix $\Param = [\param_1 \cdots \param_{|\CY|}]$}
    \[
      \pol_\Param(a \mid x) = e^{\param_a^\top x} / \sum_{a'} e^{\param_{a'}^\top x}
    \]
  \end{definition}
  \only<article>{Even though the classifier has a linear structure, the final non-linearity at the end is there to ensure that it defines a proper probability distribution over decisions.}

  \begin{block}{The $\model$-optimal classifier}
    \only<article>{Since the performance measure is simply an expectation, it is intuitive to directly optimise the decision rule with respect to an approximation of the expectation}
    \[
      \max_\Param f(\pol_\Param, \model)  \approx \max_\Param \sum_{t=1}^T  \pol(a_t = y_t \mid x_t ) P_\model(y_t \mid x_t), \qquad (x_t, y_t) \sim P_\model.
    \]
    \only<article>{In practice, this is the empirical expectation on the training set $\cset{(x_t, y_t)}{t=1, \ldots, T}$. However, when the amount of data is insufficient, this expectation may be far from reality, and so our classification rule might be far from optimal.}
  \end{block}

  \begin{block}{The Bayes-optimal classifier}
    \only<article>{An alternative idea is to use our uncertainty to create a distribution over models, and then use this distribution to obtain a single classifier that does take the uncertainty into account.}
    \[
      \max_\Param f(\pol_\Param, \bel)
      \approx
      \max_\Param N^{-1} \sum_{n=1}^N
      \pol(a_t = y_n \mid x_t = x_n),
      \qquad
      (x_n, y_n) \sim P_{\model_n}, \model_n \sim \bel.
    \]
    \only<article>{In this case, the integrals are replaced by sampling models $\model_n$ from the belief, and then sampling $(x_n, y_n)$ pairs from $P_{\model_n}$.}
  \end{block}

\end{frame}

\begin{frame}
  \frametitle{Stochastic gradient methdos}
  \only<article>{To find the maximum of a differentiable function $g$, we can use gradient descent}
  \begin{block}{Gradient ascent}
    \[
      \param_{i+1} = \param_i + \alpha \nabla_\param g(\param_i).
    \]
  \end{block}

  \only<article>{When $f$ is an expectation, we don't need to calculate the full gradient. In fact, we only need to take one sample from the related distribution.}
  \begin{block}{Stochastic gradient ascent}
    \[
      g(\param) = \int_\Model f(\param, \model) \dd \bel(\model)
    \]
    \[
      \param_{i+1} = \param_i + \alpha \nabla_\param f(\param_i, \model_i), \qquad \model_i \sim \bel.
    \]
  \end{block}
  \only<article>{Stochastic gradient methods are commonly employed in neural networks.} 
  
\end{frame}

%%% Local Variables:
%%% mode: latex
%%% TeX-master: "notes"
%%% End:

