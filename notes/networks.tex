\section{Social networks}
\only<article>{Social networks afford us another opportunity to take a look at data. We can use connections between users to infer their similarity: if two users are connected, then they are more likely to have similar preferences. }
\begin{frame}
  \frametitle{Network model}
  \begin{figure}[H]
    \centering
    \begin{tikzpicture}
      \node[RV,hidden] at (0,0) (c1) {$c_1$};
      \node[RV,hidden] at (4,0) (c2) {$c_2$};
      \node[RV,hidden] at (8,0) (c3) {$c_3$};
      \onslide<2->{
        \node[RV] at (0,3) (x1) {$x_1$};
        \node[RV] at (4,3) (x2) {$x_2$};
        \node[RV] at (8,3) (x3) {$x_3$};
        \draw[->] (c1) to (x1);
        \draw[->] (c2) to (x2);
        \draw[->] (c3) to (x3);}
     \onslide<3>{
       \node[RV] at (2,0) (z12) {$z_{12}$};
       \node[RV] at (4,-2) (z13) {$z_{13}$};
       \node[RV] at (6,0) (z23) {$z_{23}$};
       \draw[->] (c1) to (z12);
       \draw[->] (c2) to (z12);
       \draw[->] (c1) to (z13);
       \draw[->] (c3) to (z13);
       \draw[->] (c2) to (z23);
       \draw[->] (c3) to (z23);
       }
    \end{tikzpicture}
  \caption{\only<article>{Graphical model for data from a social network.} \only<1>{$c_t$ is the characteristic cluster of user $t$.} \only<2>{$x_t$ is the data seen from user $t$.} \only<3>{$z_{tu} \in \{0, 1\}$ is whether $t,u$ are connected.}}
    \label{fig:network-model}
  \end{figure}
  \only<article>{In the model seen in Figure~\ref{fig:network-model}, each user $t$ is characterised by their cluster membership $c_t$ and emits data $x_t$. Users $t,u$ are connected when $z_{t,u} = 1$.}
\end{frame}

%%% Local Variables:
%%% mode: latex
%%% TeX-master: "notes.tex"
%%% End:
