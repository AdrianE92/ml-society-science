\section{Hierarchies of decision making problems}
\only<presentation>{
  \begin{frame}
    \tableofcontents[ 
    currentsection, 
    hideothersubsections, 
    sectionstyle=show/shaded
    ] 
  \end{frame}
}


\only<article>{
  All machine learning problems are essentially decision problems. This essentially means replacing some human decisions with machine decisions. One of the simplest decision problems is classification, where you want an algorithm to decide the correct class of some data, but even within this simple framework there is a multitude of decisions to be made. The first is how to frame the classification problem the first place. The second is how to collect, process and annotate the data. The third is choosing the type of classification model to use. The fourth is how to use the collected data to find an optimal classifier within the selected type. After all this has been done, there is the problem of classifying new data. In this course, we will take a holistic view of the problem, and consider each problem in turn, starting from the lowest level and working our way up.}


\subsection{Simple decision problems}
\begin{frame}
  \frametitle{Preferences}
  \only<article>{The simplest decision problem involves selecting one item from a set of choices, such as in the following examples}  
  \begin{example}
    \begin{block}{Food}
      \begin{itemize}
      \item[A] McDonald's cheeseburger
      \item[B] Surstromming
      \item[C] Oatmeal
      \end{itemize}
    \end{block}
    \begin{block}{Money}
      \begin{itemize}
      \item[A] 10,000,000 SEK
      \item[B] 10,000,000 USD
      \item[C] 10,000,000 BTC
      \end{itemize}
    \end{block}
    \begin{block}{Entertainment}
      \begin{itemize}
      \item[A] Ticket to Liseberg
      \item[B] Ticket to Rebstar
      \item[C] Ticket to Nutcracker
      \end{itemize}
    \end{block}
  \end{example}
\end{frame}

\begin{frame}
  \frametitle{Rewards and utilities}
  \only<article>{In the decision theoretic framework, the things we receive are called rewards, and we assign a utility value to each one of them, showing which one we prefer.}
  \begin{itemize}
  \item Each choice is called a \alert{reward} $r \in \CR$.
  \item There is a \alert{utility function} $U : \CR \to \Reals$, assigning values to reward.
  \item We (weakly) prefer $A$ to $B$ iff $U(A) \geq U(B)$.
  \end{itemize}
  \only<article>{In each case, given $U$ the choice between each reward is trivial. We just select the reward:
    \[
      r^* \in \argmax_r U(r)
    \]
    The main difficult is actually selecting the appropriate utility function. In a behavioural context, we simply assume that humans act with respect to a specific utility function. However, figuring out this function from behavioural data is non trivial. ven when this assumption is correct, individuals do not have a common utility function.
  }
  \begin{exercise}
    From your individual preferences, derive a \alert{common utility function} that reflects everybody's preferences in the class for each of the three examples. Is there a simple algorithm for deciding this? Would you consider the outcome fair?
  \end{exercise}
\end{frame}

\begin{frame}
  \frametitle{Preferences among random outcomes}
  \begin{example}
    Would you rather \ldots
    \begin{itemize}
    \item[A] Have 100 EUR now?
    \item[B] Flip a coin, and get 200 EUR if it comes heads?
    \end{itemize}    
  \end{example}
  \uncover<2->{
    \begin{block}{The expected utility hypothesis}
      Rational decision makers prefer choice $A$ to $B$ if
      \[
        \E(U | A) \geq \E(U | B),
      \]
      where the expected utility is
      \[
        \E(U | A) = \sum_r U(r) \Pr(r | A).
      \]
    \end{block}
    In the above example, $r \in \{0, 100, 200\}$ and $U(r)$ is
    increasing, and the coin is fair.
  }
  \begin{itemize}
  \item<3-> If $U$ is convex, we prefer B.
  \item<4-> If $U$ is concave, we prefer A.
  \item<5-> If $U$ is linear, we don't care.
  \end{itemize}
\end{frame}


\begin{frame}
  \frametitle{Uncertain rewards}
  \only<article>{However, in real life, there are many cases where we can only choose between uncertain outcomes. The simplest example are lottery tickets, where rewards are essentially random. However, in many cases the rewards are not really random, but simply uncertain. In those cases it is useful to represent our uncertainty with probabilities as well, even though there is nothing really random.}
  \begin{itemize}
  \item Decisions $\decision \in \Decision$
  \item Each choice is called a \alert{reward} $r \in \CR$.
  \item There is a \alert{utility function} $U : \CR \to \Reals$, assigning values to reward.
  \item We (weakly) prefer $A$ to $B$ iff $U(A) \geq U(B)$.
  \end{itemize}

  \begin{example}
    \begin{columns}
      \begin{column}{0.5\textwidth}
        You are going to work, and it might rain.  What do you do?
        \begin{itemize}
        \item $\decision_1$: Take the umbrella.
        \item $\decision_2$: Risk it!
        \item $\outcome_1$: rain
        \item $\outcome_2$: dry
        \end{itemize}
      \end{column}
      \begin{column}{0.5\textwidth}
        \begin{table}
          \centering
          \begin{tabular}{c|c|c}
            $\Rew(\outcome,\decision)$ & $\decision_1$ & $\decision_2$ \\ %ro: U has only one argument.
            \hline
            $\outcome_1$ & dry, carrying umbrella & wet\\
            $\outcome_2$ & dry, carrying umbrella & dry\\
            \hline
            \hline
            $U[\Rew(\outcome,\decision)]$ & $\decision_1$ & $\decision_2$ \\
            \hline
            $\outcome_1$ & 0 & -10\\
            $\outcome_2$ & 0 & 1
          \end{tabular}
          \caption{Rewards and utilities.}
          \label{tab:rain-utility-function}
        \end{table}

        \begin{itemize}
        \item<2-> $\max_\decision \min_\outcome U = 0$
        \item<3-> $\min_\outcome \max_\decision U = 0$
        \end{itemize}
      \end{column}

    \end{columns}
  \end{example}
\end{frame}



\begin{frame}
  \frametitle{Expected utility}
  \[
    \E (U \mid a) = \sum_r U[\Rew(\outcome, \decision)] \Pr(\outcome \mid \decision)
  \]
  \begin{example}%ro: rather an exercise?
    You are going to work, and it might rain. The forecast said that
    the probability of rain $(\outcome_1)$ was $20\%$. What do you do?
    \begin{itemize}
    \item $\decision_1$: Take the umbrella.
    \item $\decision_2$: Risk it!
    \end{itemize}
    \begin{table}
      \centering
      \begin{tabular}{c|c|c}
        $\Rew(\outcome,\decision)$ & $\decision_1$ & $\decision_2$ \\ %ro: U has only one argument.
        \hline
        $\outcome_1$ & dry, carrying umbrella & wet\\
        $\outcome_2$ & dry, carrying umbrella & dry\\
        \hline
        \hline
        $U[\Rew(\outcome,\decision)]$ & $\decision_1$ & $\decision_2$ \\
        \hline
        $\outcome_1$ & 0 & -10\\
        $\outcome_2$ & 0 & 1\\
        \hline
        \hline
        $\E_P(U \mid \decision)$ & 0 &  -1.2 \\ 
      \end{tabular}
      \caption{Rewards, utilities, expected utility for $20\%$ probability of rain.}
      \label{tab:rain-utility-function}
    \end{table}
  \end{example}
\end{frame}





\subsection{Decision rules}

\only<article>{We now move from simple decisions to decisions that
  depend on some observation. We shall start with a simple problem in applied meteorology. Then we will discuss hypothesis testing as a decision making problem. Finally, we will go through an exercise in Bayesian methods for classification.}

\begin{frame}
  \frametitle{Bayes decision rules}
  Consider the case where outcomes are independent of decisions:
  \[
    \util (P, \decision) \defn \sum_{\model}  \util (\model, \decision) P(\model)
  \]
  This corresponds e.g. to the case where $P(\omega)$ is the belief about an unknown world.
  \begin{definition}[Bayes utility]
    \label{def:bayes-utility}
    The maximising decision for $P$ has an expected utility equal to:
    \begin{equation}
      \BUtil(P) \defn \sup_{\decision \in \Decision} \util (P, \decision).
      \label{eq:bayes-utility}
    \end{equation}
  \end{definition}
\end{frame}




\begin{frame}
  \frametitle{The $n$-meteorologists problem}
  \only<article>{Of course, we may not always just be interested in classification performance in terms of predicting the most likely class. It strongly depends on the problem we are actually wanting to solve. In  biometric authentication, for example, we want to guard against the unlikely event that an impostor will successfully be authenticated. Even if the decision rule that always says 'OK' has the lowest classification error in practice, the expected cost of impostors means that the optimal decision rule must sometimes say 'Failed' even if this leads to false rejections sometimes.}
  \begin{exercise}
    \only<presentation>{
      \only<1>{
        \begin{itemize}
        \item Meteorological models $\CM = \set{\model_1, \ldots, \model_n}$
        \item Rain predictions at time $t$: $p_{t,\model} \defn  P_{\model}(x_t = \textrm{rain})$.
        \item Prior probability $\bel(\model) = 1/n$ for each model.
        \item Should we take the umbrella?
        \end{itemize}
      }
    }
    \only<article>{Assume you have $n$ meteorologists. At each day $t$, each meteorologist $i$ gives a probability $p_{t,\model_i}\defn P_{\model_i}(x_t = \textrm{rain})$ for rain. Consider the case of there being three meteorologists, and each one making the following prediction for the coming week. Start with a uniform prior $\bel(\model) = 1/3$ for each model.}
    {
      \begin{table}[h]
        \begin{tabular}{c|l|l|l|l|l|l|l}
          &M&T&W&T&F&S&S\\
          \hline
          CNN & 0.5 & 0.6 & 0.7 & 0.9 & 0.5 & 0.3 & 0.1\\
          SMHI & 0.3 & 0.7 & 0.8 & 0.9 & 0.5 & 0.2 & 0.1\\
          YR & 0.6 & 0.9 & 0.8 & 0.5 & 0.4 & 0.1 & 0.1\\
          \hline
          Rain? & Y & Y & Y & N & Y & N & N
        \end{tabular}
        \caption{Predictions by three different entities for the probability of rain on a particular day, along with whether or not it actually rained.}
        \label{tab:meteorologists}
      \end{table}
    }
    \uncover<2->{
      \begin{enumerate}
      \item<2-> What is your belief about the quality of each meteorologist after each day? 
      \item<3-> What is your belief about the probability of rain each day? 
        \[
          P_\bel(x_t = \textrm{rain} \mid x_1, x_2, \ldots x_{t-1})
          =
          \sum_{\model \in \Model} P_\model(x_t = \textrm{rain})
          \bel(\model \mid x_1, x_2, \ldots x_{t-1}) 
        \]
      \item<4-> Assume you can decide whether or not to go running each
        day. If you go running and it does not rain, your utility is 1. If
        it rains, it's -10. If you don't go running, your utility is
        0. What is the decision maximising utility in expectation (with respect to the posterior) each
        day?
      \end{enumerate}
    }
  \end{exercise}
\end{frame}


\subsection{Statistical testing}
\only<article>{A common type of decision problem is a statistical test. This arises when we have a set of possible candidate models $\CM$ and we need to be able to decide which model to select after we see the evidence.
  Many times, there is only one model under consideration, $\model_0$, the so-called \alert{null hypothesis}. Then, our only decision is whether or not to accept or reject this hypothesis.}
\begin{frame}
  \frametitle{Simple hypothesis testing}
  \only<article>{Let us start with the simple case of needing to compare two models.}
  \begin{block}{The simple hypothesis test as a decision problem}
    \begin{itemize}
    \item $\CM = \{\model_0, \model_1\}$
    \item $a_0$: Accept model $\model_0$
    \item $a_1$: Accept model $\model_1$
    \end{itemize}
    \begin{table}[H]
      \begin{tabular}{c|cc}
        $\util$& $\model_0$& $\model_1$\\\hline
        $a_0$ & 1 & 0\\
        $a_1$ & 0 & 1
      \end{tabular}
      \caption{Example utility function for simple hypothesis tests.}
    \end{table}
    \only<article>{There is no reason for us to be restricted to this utility function. As it is diagonal, it effectively treats both types of errors in the same way.}
  \end{block}

  \begin{example}[Continuation of the medium example]
    \begin{itemize}
    \item $\model_1$: that John is a medium.
    \item $\model_0$: that John is not a medium.
    \end{itemize}
    \only<article>{
    Let $x_t$ be $0$ if John makes an incorrect prediction at time $t$ and $x_t = 1$ if he makes a correct prediction. Let us once more assume a Bernoulli model, so that John's claim that he can predict our tosses perfectly means that for a sequence of tosses $\bx = x_1, \ldots, x_n$,
    \[
      P_{\model_1}(\bx) = \begin{cases}
        1, & x_t = 1 \forall t \in [n]\\
        0, & \exists t \in [n] : x_t = 0.
      \end{cases}
    \]
    That is, the probability of perfectly correct predictions is 1, and that of one or more incorrect prediction is 0. For the other model, we can assume that all draws are independently and identically distributed from a fair coin. Consequently, no matter what John's predictions are, we have that:
    \[
      P_{\model_0}(\bx = 1 \ldots 1) = 2^{-n}.
    \]
    So, for the given example, as stated, we have the following facts:
    \begin{itemize}
    \item If John makes one or more mistakes, then $\Pr(\bx \mid \model_1) = 0$ and $\Pr(\bx \mid \model_0) = 2^{-n}$. Thus, we should perhaps say that then John is not a medium
    \item If John makes no mistakes at all, then 
      \begin{align}
        \Pr(\bx = 1, \ldots, 1 \mid \model_1) &= 1,
        &
          \Pr(\bx = 1, \ldots, 1 \mid \model_0) &= 2^{-n}.
      \end{align}
    \end{itemize}
    Now we can calculate the posterior distribution, which is
    \[
      \bel(\model_1 \mid \bx = 1, \ldots, 1) = \frac{1 \times \bel(\model_1)}{1 \times \bel(model_1) + 2^{-n} (1 - \bel(\model_1))}.
    \]
    Our expected utility for taking action $a_0$ is actually
  }
  \[
    \E_\bel(\util \mid a_0) = 1 \times \bel(\model_0 \mid \bx) + 0 \times \bel(\model_1 \mid \bx), \qquad
    \E_\bel(\util \mid a_1) = 0 \times \bel(\model_0 \mid \bx) + 1 \times \bel(\model_1 \mid \bx)
  \]
  \end{example}
  
\end{frame}


\begin{frame}
  \frametitle{Null hypothesis test}
  Many times, there is only one model under consideration, $\model_0$, the so-called \alert{null hypothesis}. \only<article>{ This happens when, for example, we have no simple way of defining an appropriate alternative. Consider the example of the medium: How should we expect a medium to predict? Then, our only decision is whether or not to accept or reject this hypothesis.}
  \begin{block}{The null hypothesis test as a decision problem}
    \begin{itemize}
    \item $a_0$: Accept model $\model_0$
    \item $a_1$: Reject model $\model_0$
    \end{itemize}
  \end{block}

  \begin{example}{Construction of the test for the medium}
  \begin{itemize}
  \item<2-> $\model_0$ is simply the $\Bernoulli(1/2)$ model: responses are by chance.
  \item<3-> We need to design a policy $\pol(a \mid \bx)$ that accepts or rejects depending on the data.
  \item<4-> Since there is no alternative model, we can only construct this policy according to its properties when $\model_0$ is true.
  \item<5-> In particular, we can fix a policy that only chooses $a_1$ when $\model_0$ is true a proportion $\delta$ of the time.
  \item<6-> This can be done by construcing a threshold test from the inverse-CDF.
  \end{itemize}
\end{example}
\end{frame}
\begin{frame}
  \frametitle{Using $p$-values to construct statistical tests}
  \begin{definition}[Null statistical test]
    \only<article>{
      A statistical test $\pol$ is a decision rule for accepting or rejecting a hypothesis on the basis of evidence. A $p$-value test rejects a hypothesis whenever the value of the statistic $f(x)$ is smaller than a threshold.}
      The statistic $f : \CX \to [0,1]$ is  designed to have the property:
  \[
    P_{\model_0}(\cset{x}{f(x) \leq \delta}) = \delta.
  \]
  If our decision rule is:
  \[
    \pol(a \mid x) =
    \begin{cases}
      a_0, & f(x) \leq \delta\\
      a_1, & f(x) > \delta,
    \end{cases}
  \]
  the probability of rejecting the null hypothesis when it is true is exactly $\delta$.
\end{definition}
\only<presentation>{The value of the statistic $f(x)$, otherwise known as the \alert{$p$-value}, is uninformative.}
\only<article>{This is because, by definition, $f(x)$ has a uniform distribution under $\model_0$. Hence the value of $f(x)$ itself is uninformative: high and low values are equally likely. In theory we should simply choose $\delta$ before seeing the data and just accept or reject based on whether $f(x) \leq \delta$. However nobody does that in practice, meaning that $p$-values are used incorrectly. Better not to use them at all, if uncertain about their meaning.}
\end{frame}
\begin{frame}
  \frametitle{Issues with $p$-values}
  \begin{itemize}
  \item They only measure quality of fit \alert{on the data}.
  \item Not robust to model misspecification. \only<article>{For example, zero-mean testing using the $\chi^2$-test has a normality assumption.}
  \item They ignore effect sizes. \only<article>{For example, a linear analysis may determine that there is a significant deviation from zero-mean, but with only a small effect size of 0.01. Thus, reporting only the $p$-value is misleading}
  \item They do not consider prior information. 
  \item They do not represent the probability of having made an error. \only<article>{In particular, a $p$-value of $\delta$ does not mean that the probability that the null hypothesis is false given the data $x$, is $\delta$, i.e. $\delta \neq \Pr(\neg \model_0 \mid x)$.}
  \item The null-rejection error probability is the same irrespective of the amount of data.
  \end{itemize}
\end{frame}

\begin{frame}\frametitle{$p$-values for the medium example}
  \only<article>{Let us consider the example of the medium.}
  \begin{itemize}
  \item<2->$\model_0$ is simply the $\Bernoulli(1/2)$ model:
    responses are by chance. 
  \item<3->CDF: $P_{\model_0}(N \leq n \mid K = 100)$ \only<article> {is the probability of at most $N$ successes if we throw the coin 100 times. This is in fact the cumulative probability function of the binomial distribution. Recall that the binomial represents the distribution for the number of successes of independent experiments, each following a Bernoulli distribution.}
  \item<4->ICDF:  the number of successes that will happen with probability at least $\delta$
  \item<5->e.g. we'll get at most 50 successes a proportion $\delta = 1/2$ of the time.
  \item<6>Using the (inverse) CDF we can construct a policy $\pol$ that selects $a_1$ when $\model_0$ is true only a $\delta$ portion of the time, for any choice of $\delta$.
  \end{itemize}
  \begin{columns}
    \setlength\fheight{0.33\columnwidth}
    \setlength\fwidth{0.33\columnwidth}
    \begin{column}{0.5\textwidth}
      \only<3,4,5,6>{% This file was created by matlab2tikz.
%
%The latest updates can be retrieved from
%  http://www.mathworks.com/matlabcentral/fileexchange/22022-matlab2tikz-matlab2tikz
%where you can also make suggestions and rate matlab2tikz.
%
\begin{tikzpicture}

\begin{axis}[%
width=\fwidth,
height=0.831\fheight,
at={(0\fwidth,0\fheight)},
scale only axis,
xmin=0,
xmax=100,
xlabel={Number of successes},
ymin=0,
ymax=1,
ylabel={Probability of less than N successes},
axis background/.style={fill=white}
]
\addplot [color=blue, forget plot]
  table[row sep=crcr]{%
0	7.8886090522101e-31\\
1	7.96749514273217e-29\\
2	3.98453643227134e-27\\
3	1.31543344806511e-25\\
4	3.2248444478818e-24\\
5	6.26162256269277e-23\\
6	1.0029797609618e-21\\
7	1.36307186640302e-20\\
8	1.60428183412199e-19\\
9	1.66102448972682e-18\\
10	1.53164508771899e-17\\
11	1.27042666774617e-16\\
12	9.5567876801385e-16\\
13	6.56490776101789e-15\\
14	4.14222593604009e-14\\
15	2.41271075196861e-13\\
16	1.30296790932804e-12\\
17	6.54899932503508e-12\\
18	3.07390330752401e-11\\
19	1.35138126102441e-10\\
20	5.57954452862595e-10\\
21	2.1686833167108e-09\\
22	7.9526642368932e-09\\
23	2.75679038792502e-08\\
24	9.05001310651458e-08\\
25	2.81814101710274e-07\\
26	8.33681324725063e-07\\
27	2.34620630632112e-06\\
28	6.28957500833936e-06\\
29	1.60800076478334e-05\\
30	3.92506982279687e-05\\
31	9.15716124411769e-05\\
32	0.000204388583713412\\
33	0.000436859918456193\\
34	0.000894965195743431\\
35	0.00175882086148508\\
36	0.00331856025796311\\
37	0.00601648786268185\\
38	0.0104893678389258\\
39	0.0176001001088524\\
40	0.0284439668204906\\
41	0.044313040057034\\
42	0.0666053096036057\\
43	0.0966739522478211\\
44	0.135626512036918\\
45	0.184100808663348\\
46	0.242059206803646\\
47	0.308649706794632\\
48	0.382176717201337\\
49	0.460205381306407\\
50	0.539794618693593\\
51	0.617823282798662\\
52	0.691350293205368\\
53	0.757940793196354\\
54	0.815899191336652\\
55	0.864373487963082\\
56	0.903326047752179\\
57	0.933394690396394\\
58	0.955686959942966\\
59	0.971556033179509\\
60	0.982399899891148\\
61	0.989510632161074\\
62	0.993983512137318\\
63	0.996681439742037\\
64	0.998241179138515\\
65	0.999105034804257\\
66	0.999563140081544\\
67	0.999795611416287\\
68	0.999908428387559\\
69	0.999960749301772\\
70	0.999983919992352\\
71	0.999993710424992\\
72	0.999997653793694\\
73	0.999999166318675\\
74	0.999999718185898\\
75	0.999999909499869\\
76	0.999999972432096\\
77	0.999999992047336\\
78	0.999999997831317\\
79	0.999999999442046\\
80	0.999999999864862\\
81	0.999999999969261\\
82	0.999999999993451\\
83	0.999999999998697\\
84	0.999999999999759\\
85	0.999999999999959\\
86	0.999999999999993\\
87	0.999999999999999\\
88	1\\
89	1\\
90	1\\
91	1\\
92	1\\
93	1\\
94	1\\
95	1\\
96	1\\
97	1\\
98	1\\
99	1\\
100	1\\
};
\end{axis}
\end{tikzpicture}%}      
    \end{column}
    \begin{column}{0.5\textwidth}
      \only<4,5,6>{% This file was created by matlab2tikz.
%
%The latest updates can be retrieved from
%  http://www.mathworks.com/matlabcentral/fileexchange/22022-matlab2tikz-matlab2tikz
%where you can also make suggestions and rate matlab2tikz.
%
\begin{tikzpicture}

\begin{axis}[%
width=\fwidth,
height=0.831\fheight,
at={(0\fwidth,0\fheight)},
scale only axis,
xmin=0,
xmax=1,
xlabel={Probability of less than N successes},
ymin=0,
ymax=100,
ylabel={Number of successes},
axis background/.style={fill=white}
]
\addplot [color=blue, forget plot]
  table[row sep=crcr]{%
0	0\\
0.01	38\\
0.02	40\\
0.03	41\\
0.04	41\\
0.05	42\\
0.06	42\\
0.07	43\\
0.08	43\\
0.09	43\\
0.1	44\\
0.11	44\\
0.12	44\\
0.13	44\\
0.14	45\\
0.15	45\\
0.16	45\\
0.17	45\\
0.18	45\\
0.19	46\\
0.2	46\\
0.21	46\\
0.22	46\\
0.23	46\\
0.24	46\\
0.25	47\\
0.26	47\\
0.27	47\\
0.28	47\\
0.29	47\\
0.3	47\\
0.31	48\\
0.32	48\\
0.33	48\\
0.34	48\\
0.35	48\\
0.36	48\\
0.37	48\\
0.38	48\\
0.39	49\\
0.4	49\\
0.41	49\\
0.42	49\\
0.43	49\\
0.44	49\\
0.45	49\\
0.46	49\\
0.47	50\\
0.48	50\\
0.49	50\\
0.5	50\\
0.51	50\\
0.52	50\\
0.53	50\\
0.54	51\\
0.55	51\\
0.56	51\\
0.57	51\\
0.58	51\\
0.59	51\\
0.6	51\\
0.61	51\\
0.62	52\\
0.63	52\\
0.64	52\\
0.65	52\\
0.66	52\\
0.67	52\\
0.68	52\\
0.69	52\\
0.7	53\\
0.71	53\\
0.72	53\\
0.73	53\\
0.74	53\\
0.75	53\\
0.76	54\\
0.77	54\\
0.78	54\\
0.79	54\\
0.8	54\\
0.81	54\\
0.82	55\\
0.83	55\\
0.84	55\\
0.85	55\\
0.86	55\\
0.87	56\\
0.88	56\\
0.89	56\\
0.9	56\\
0.91	57\\
0.92	57\\
0.93	57\\
0.94	58\\
0.95	58\\
0.96	59\\
0.97	59\\
0.98	60\\
0.99	62\\
1	86\\
};
\end{axis}
\end{tikzpicture}%}
    \end{column}
  \end{columns}    
\end{frame}



\begin{frame}
  \frametitle{Building a test}
  \begin{block}{The test statistic}
    We want the test to reflect that we don't have a significant number of failures.
    \[
    f(x) = 1 - \textrm{binocdf}(\sum_{t=1}^n x_t, n, 0.5)
    \]
  \end{block}
  \begin{alertblock}{What $f(x)$ is and is not}
    \begin{itemize}
    \item It is a \textbf{statistic} which is $\leq \delta$ a $\delta$ portion of the time when $\model_0$ is true.
    \item It is \textbf{not} the probability of observing $x$ under $\model_0$.
    \item It is \textbf{not} the probability of $\model_0$ given $x$.
    \end{itemize}
  \end{alertblock}
  \begin{exercise}
    \begin{itemize}
    \item<1-> Let us throw a coin 8 times.
    \item<2-> Select a $p$-value threshold of 50
    \end{itemize}
    
  \end{exercise}
\end{frame}


\subsection{Classification problems}
\only<article>{
  One of the simplest decision problems is classification. At the simplest level, this is the problem of observing some data point $x_t \in \CX$ and making a decision about what class $\CY$ it belongs to. Typically, a fixed classifier is defined as a decision rule $\pi(a | x)$ making decisions $a \in \CA$, where the decision space includes the class labels, so that if we observe some point $x_t$ and choose $a_t = 1$, we essentially declare that $y_t = 1$.

  Typically, we wish to have a classification policy that minimises classification error.
}
\begin{frame}
  \frametitle{Deciding a class given a model}
  \only<article>{In the simplest classification problem, we observe some features $x_t$ and want to make a guess $\decision_t$ about the true class label $y_t$. Assuming we have some probabilistic model $P(y_t \mid x_t)$, we want to define a decision rule $\pol(\decision_t \mid x_t)$ that is optimal, in the sense that it maximises expected utility for $P$.}
  \begin{itemize}
  \item Features $x_t \in \CX$.
  \item Label $y_t \in \CY$.
  \item Decisions $\decision_t \in \CA$.
  \item Decision rule $\pol(\decision_t \mid x_t)$ assigns probabilities to actions.
  \end{itemize}
  
  \begin{block}{Standard classification problem}
    \only<article>{In the simplest case, the set of decisions we make are the same as the set of classes}
    \[
      \CA = \CY, \qquad
      U(\decision, y) = \ind{\decision = y}
    \]
  \end{block}

  \begin{exercise}
    If we have a model $P(y_t \mid x_t)$, and a suitable $U$, what is the optimal decision to make?
  \end{exercise}
  \only<presentation>{
    \uncover<2->{
      \[
        \decision_t \in \argmax_{\decision \in \Decision} \sum_y P(y_t = y \mid x_t) \util(\decision, y)
      \]
    }
    \uncover<3>{
      For standard classification,
      \[
        \decision_t \in \argmax_{\decision \in \Decision} P(y_t = \decision \mid x_t)
      \]
    }
  }
\end{frame}


\begin{frame}
  \frametitle{Deciding the class given a model family}
  \begin{itemize}
  \item Training data $S = (x_1, y_1, \ldots, x_n, y_n)$
  \item Models $\cset{P_\model}{\model \in \Model}$
  \item Prior $\bel$ on $\Model$.
  \end{itemize}
  \[
    \bel(\model \mid S)
    = \frac{P_\model(y_1, \ldots, y_n \mid x_1, \ldots, x_n) \bel(\model)}
    {\sum_{\model' \in \Model} P_{\model'}(y_1, \ldots, y_n \mid x_1, \ldots, x_n) \bel(\model')}
  \]
  We can then calculate the posterior marginal marginal label probability
  \[
    P_{\bel \mid S}(y_t \mid x_t) \defn
    P_{\bel}(y_t \mid x_t, S) = 
    \sum_{\model \in \Model} P_\model(y_t \mid x_t) \bel(\omega \mid S).
  \]
  We can then construct the following simple decision rule:
  \[
    \decision_t \in \argmax_{\decision \in \CY}\sum_{\model \in \Model} P_\model(y_t \mid x_t) \bel(\omega \mid S),
  \]
  otherwise known as the \alert{Bayes rule}.
\end{frame}






%%% Local Variables:
%%% mode: latex
%%% TeX-master: "notes"
%%% End:

