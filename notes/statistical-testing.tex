\subsection{Statistical testing}
\only<article>{A common type of decision problem is a statistical test. This arises when we have a set of possible candidate models $\CM$ and we need to be able to decide which model to select after we see the evidence.
  Many times, there is only one model under consideration, $\model_0$, the so-called \alert{null hypothesis}. Then, our only decision is whether or not to accept or reject this hypothesis.}
\begin{frame}
  \frametitle{Simple hypothesis testing}
  \only<article>{Let us start with the simple case of needing to compare two models.}
  \begin{block}{The simple hypothesis test as a decision problem}
    \begin{itemize}
    \item $\CM = \{\model_0, \model_1\}$
    \item $a_0$: Accept model $\model_0$
    \item $a_1$: Accept model $\model_1$
    \end{itemize}
    \begin{table}[H]
      \begin{tabular}{c|cc}
        $\util$& $\model_0$& $\model_1$\\\hline
        $a_0$ & 1 & 0\\
        $a_1$ & 0 & 1
      \end{tabular}
      \caption{Example utility function for simple hypothesis tests.}
    \end{table}
    \only<article>{There is no reason for us to be restricted to this utility function. As it is diagonal, it effectively treats both types of errors in the same way.}
  \end{block}

  \begin{example}[Continuation of the medium example]
    \begin{itemize}
    \item $\model_1$: that John is a medium.
    \item $\model_0$: that John is not a medium.
    \end{itemize}
    \only<article>{
      Let $x_t$ be $0$ if John makes an incorrect prediction at time $t$ and $x_t = 1$ if he makes a correct prediction. Let us once more assume a Bernoulli model, so that John's claim that he can predict our tosses perfectly means that for a sequence of tosses $\bx = x_1, \ldots, x_n$,
      \[
      P_{\model_1}(\bx) = \begin{cases}
        1, & x_t = 1 \forall t \in [n]\\
        0, & \exists t \in [n] : x_t = 0.
      \end{cases}
      \]
      That is, the probability of perfectly correct predictions is 1, and that of one or more incorrect prediction is 0. For the other model, we can assume that all draws are independently and identically distributed from a fair coin. Consequently, no matter what John's predictions are, we have that:
      \[
      P_{\model_0}(\bx = 1 \ldots 1) = 2^{-n}.
      \]
      So, for the given example, as stated, we have the following facts:
      \begin{itemize}
      \item If John makes one or more mistakes, then $\Pr(\bx \mid \model_1) = 0$ and $\Pr(\bx \mid \model_0) = 2^{-n}$. Thus, we should perhaps say that then John is not a medium
      \item If John makes no mistakes at all, then 
        \begin{align}
          \Pr(\bx = 1, \ldots, 1 \mid \model_1) &= 1,
          &
            \Pr(\bx = 1, \ldots, 1 \mid \model_0) &= 2^{-n}.
        \end{align}
      \end{itemize}
      Now we can calculate the posterior distribution, which is
      \[
      \bel(\model_1 \mid \bx = 1, \ldots, 1) = \frac{1 \times \bel(\model_1)}{1 \times \bel(model_1) + 2^{-n} (1 - \bel(\model_1))}.
      \]
      Our expected utility for taking action $a_0$ is actually
    }
    \[
    \E_\bel(\util \mid a_0) = 1 \times \bel(\model_0 \mid \bx) + 0 \times \bel(\model_1 \mid \bx), \qquad
    \E_\bel(\util \mid a_1) = 0 \times \bel(\model_0 \mid \bx) + 1 \times \bel(\model_1 \mid \bx)
    \]
  \end{example}
  
\end{frame}


\begin{frame}
  \frametitle{Null hypothesis test}
  Many times, there is only one model under consideration, $\model_0$, the so-called \alert{null hypothesis}. \only<article>{ This happens when, for example, we have no simple way of defining an appropriate alternative. Consider the example of the medium: How should we expect a medium to predict? Then, our only decision is whether or not to accept or reject this hypothesis.}
  \begin{block}{The null hypothesis test as a decision problem}
    \begin{itemize}
    \item $a_0$: Accept model $\model_0$
    \item $a_1$: Reject model $\model_0$
    \end{itemize}
  \end{block}

  \begin{example}{Construction of the test for the medium}
    \index{policy!for statistical testing}
    \begin{itemize}
    \item<2-> $\model_0$ is simply the $\Bernoulli(1/2)$ model: responses are by chance.
    \item<3-> We need to design a policy $\pol(a \mid \bx)$ that accepts or rejects depending on the data.
    \item<4-> Since there is no alternative model, we can only construct this policy according to its properties when $\model_0$ is true.
    \item<5-> In particular, we can fix a policy that only chooses $a_1$ when $\model_0$ is true a proportion $\delta$ of the time.
    \item<6-> This can be done by construcing a threshold test from the inverse-CDF.
    \end{itemize}
  \end{example}
\end{frame}
\begin{frame}
  \frametitle{Using $p$-values to construct statistical tests}
  \begin{definition}[Null statistical test]
    \only<article>{
      A statistical test $\pol$ is a decision rule for accepting or rejecting a hypothesis on the basis of evidence. A $p$-value test rejects a hypothesis whenever the value of the statistic $f(x)$ is smaller than a threshold.}
    The statistic $f : \CX \to [0,1]$ is  designed to have the property:
    \[
    P_{\model_0}(\cset{x}{f(x) \leq \delta}) = \delta.
    \]
    If our decision rule is:
    \[
    \pol(a \mid x) =
    \begin{cases}
      a_0, & f(x) \geq \delta\\
      a_1, & f(x) < \delta,
    \end{cases}
    \]
    the probability of rejecting the null hypothesis when it is true is exactly $\delta$.
  \end{definition}
  \only<presentation>{The value of the statistic $f(x)$, otherwise known as the \alert{$p$-value}, is uninformative.}
  \only<article>{This is because, by definition, $f(x)$ has a uniform distribution under $\model_0$. Hence the value of $f(x)$ itself is uninformative: high and low values are equally likely. In theory we should simply choose $\delta$ before seeing the data and just accept or reject based on whether $f(x) \leq \delta$. However nobody does that in practice, meaning that $p$-values are used incorrectly. Better not to use them at all, if uncertain about their meaning.}
\end{frame}
\begin{frame}
  \frametitle{Issues with $p$-values}
  \begin{itemize}
  \item They only measure quality of fit \alert{on the data}.
  \item Not robust to model misspecification. \only<article>{For example, zero-mean testing using the $\chi^2$-test has a normality assumption.}
  \item They ignore effect sizes. \only<article>{For example, a linear analysis may determine that there is a significant deviation from zero-mean, but with only a small effect size of 0.01. Thus, reporting only the $p$-value is misleading}
  \item They do not consider prior information. 
  \item They do not represent the probability of having made an error. \only<article>{In particular, a $p$-value of $\delta$ does not mean that the probability that the null hypothesis is false given the data $x$, is $\delta$, i.e. $\delta \neq \Pr(\neg \model_0 \mid x)$.}
  \item The null-rejection error probability is the same irrespective of the amount of data (by design).
  \end{itemize}
\end{frame}

\begin{frame}\frametitle{$p$-values for the medium example}
  \only<article>{Let us consider the example of the medium.}
  \begin{itemize}
  \item<2->$\model_0$ is simply the $\Bernoulli(1/2)$ model:
    responses are by chance. 
  \item<3->CDF: $P_{\model_0}(N \leq n \mid K = 100)$ \only<article> {is the probability of at most $N$ successes if we throw the coin 100 times. This is in fact the cumulative probability function of the binomial distribution. Recall that the binomial represents the distribution for the number of successes of independent experiments, each following a Bernoulli distribution.}
  \item<4->ICDF:  the number of successes that will happen with probability at least $\delta$
  \item<5->e.g. we'll get at most 50 successes a proportion $\delta = 1/2$ of the time.
  \item<6>Using the (inverse) CDF we can construct a policy $\pol$ that selects $a_1$ when $\model_0$ is true only a $\delta$ portion of the time, for any choice of $\delta$.
  \end{itemize}
  \begin{columns}
    \setlength\fheight{0.33\columnwidth}
    \setlength\fwidth{0.33\columnwidth}
    \begin{column}{0.5\textwidth}
      \only<3,4,5,6>{% This file was created by matlab2tikz.
%
%The latest updates can be retrieved from
%  http://www.mathworks.com/matlabcentral/fileexchange/22022-matlab2tikz-matlab2tikz
%where you can also make suggestions and rate matlab2tikz.
%
\begin{tikzpicture}

\begin{axis}[%
width=\fwidth,
height=0.831\fheight,
at={(0\fwidth,0\fheight)},
scale only axis,
xmin=0,
xmax=100,
xlabel={Number of successes},
ymin=0,
ymax=1,
ylabel={Probability of less than N successes},
axis background/.style={fill=white}
]
\addplot [color=blue, forget plot]
  table[row sep=crcr]{%
0	7.8886090522101e-31\\
1	7.96749514273217e-29\\
2	3.98453643227134e-27\\
3	1.31543344806511e-25\\
4	3.2248444478818e-24\\
5	6.26162256269277e-23\\
6	1.0029797609618e-21\\
7	1.36307186640302e-20\\
8	1.60428183412199e-19\\
9	1.66102448972682e-18\\
10	1.53164508771899e-17\\
11	1.27042666774617e-16\\
12	9.5567876801385e-16\\
13	6.56490776101789e-15\\
14	4.14222593604009e-14\\
15	2.41271075196861e-13\\
16	1.30296790932804e-12\\
17	6.54899932503508e-12\\
18	3.07390330752401e-11\\
19	1.35138126102441e-10\\
20	5.57954452862595e-10\\
21	2.1686833167108e-09\\
22	7.9526642368932e-09\\
23	2.75679038792502e-08\\
24	9.05001310651458e-08\\
25	2.81814101710274e-07\\
26	8.33681324725063e-07\\
27	2.34620630632112e-06\\
28	6.28957500833936e-06\\
29	1.60800076478334e-05\\
30	3.92506982279687e-05\\
31	9.15716124411769e-05\\
32	0.000204388583713412\\
33	0.000436859918456193\\
34	0.000894965195743431\\
35	0.00175882086148508\\
36	0.00331856025796311\\
37	0.00601648786268185\\
38	0.0104893678389258\\
39	0.0176001001088524\\
40	0.0284439668204906\\
41	0.044313040057034\\
42	0.0666053096036057\\
43	0.0966739522478211\\
44	0.135626512036918\\
45	0.184100808663348\\
46	0.242059206803646\\
47	0.308649706794632\\
48	0.382176717201337\\
49	0.460205381306407\\
50	0.539794618693593\\
51	0.617823282798662\\
52	0.691350293205368\\
53	0.757940793196354\\
54	0.815899191336652\\
55	0.864373487963082\\
56	0.903326047752179\\
57	0.933394690396394\\
58	0.955686959942966\\
59	0.971556033179509\\
60	0.982399899891148\\
61	0.989510632161074\\
62	0.993983512137318\\
63	0.996681439742037\\
64	0.998241179138515\\
65	0.999105034804257\\
66	0.999563140081544\\
67	0.999795611416287\\
68	0.999908428387559\\
69	0.999960749301772\\
70	0.999983919992352\\
71	0.999993710424992\\
72	0.999997653793694\\
73	0.999999166318675\\
74	0.999999718185898\\
75	0.999999909499869\\
76	0.999999972432096\\
77	0.999999992047336\\
78	0.999999997831317\\
79	0.999999999442046\\
80	0.999999999864862\\
81	0.999999999969261\\
82	0.999999999993451\\
83	0.999999999998697\\
84	0.999999999999759\\
85	0.999999999999959\\
86	0.999999999999993\\
87	0.999999999999999\\
88	1\\
89	1\\
90	1\\
91	1\\
92	1\\
93	1\\
94	1\\
95	1\\
96	1\\
97	1\\
98	1\\
99	1\\
100	1\\
};
\end{axis}
\end{tikzpicture}%}      
    \end{column}
    \begin{column}{0.5\textwidth}
      \only<4,5,6>{% This file was created by matlab2tikz.
%
%The latest updates can be retrieved from
%  http://www.mathworks.com/matlabcentral/fileexchange/22022-matlab2tikz-matlab2tikz
%where you can also make suggestions and rate matlab2tikz.
%
\begin{tikzpicture}

\begin{axis}[%
width=\fwidth,
height=0.831\fheight,
at={(0\fwidth,0\fheight)},
scale only axis,
xmin=0,
xmax=1,
xlabel={Probability of less than N successes},
ymin=0,
ymax=100,
ylabel={Number of successes},
axis background/.style={fill=white}
]
\addplot [color=blue, forget plot]
  table[row sep=crcr]{%
0	0\\
0.01	38\\
0.02	40\\
0.03	41\\
0.04	41\\
0.05	42\\
0.06	42\\
0.07	43\\
0.08	43\\
0.09	43\\
0.1	44\\
0.11	44\\
0.12	44\\
0.13	44\\
0.14	45\\
0.15	45\\
0.16	45\\
0.17	45\\
0.18	45\\
0.19	46\\
0.2	46\\
0.21	46\\
0.22	46\\
0.23	46\\
0.24	46\\
0.25	47\\
0.26	47\\
0.27	47\\
0.28	47\\
0.29	47\\
0.3	47\\
0.31	48\\
0.32	48\\
0.33	48\\
0.34	48\\
0.35	48\\
0.36	48\\
0.37	48\\
0.38	48\\
0.39	49\\
0.4	49\\
0.41	49\\
0.42	49\\
0.43	49\\
0.44	49\\
0.45	49\\
0.46	49\\
0.47	50\\
0.48	50\\
0.49	50\\
0.5	50\\
0.51	50\\
0.52	50\\
0.53	50\\
0.54	51\\
0.55	51\\
0.56	51\\
0.57	51\\
0.58	51\\
0.59	51\\
0.6	51\\
0.61	51\\
0.62	52\\
0.63	52\\
0.64	52\\
0.65	52\\
0.66	52\\
0.67	52\\
0.68	52\\
0.69	52\\
0.7	53\\
0.71	53\\
0.72	53\\
0.73	53\\
0.74	53\\
0.75	53\\
0.76	54\\
0.77	54\\
0.78	54\\
0.79	54\\
0.8	54\\
0.81	54\\
0.82	55\\
0.83	55\\
0.84	55\\
0.85	55\\
0.86	55\\
0.87	56\\
0.88	56\\
0.89	56\\
0.9	56\\
0.91	57\\
0.92	57\\
0.93	57\\
0.94	58\\
0.95	58\\
0.96	59\\
0.97	59\\
0.98	60\\
0.99	62\\
1	86\\
};
\end{axis}
\end{tikzpicture}%}
    \end{column}
  \end{columns}    
\end{frame}



\begin{frame}
  \frametitle{Building a test}
  \begin{block}{The test statistic}
    We want the test to reflect that we don't have a significant number of failures.
    \[
    f(x) = 1 - \textrm{binocdf}(\sum_{t=1}^n x_t, n, 0.5)
    \]
  \end{block}
  \begin{alertblock}{What $f(x)$ is and is not}
    \begin{itemize}
    \item It is a \textbf{statistic} which is $\leq \delta$ a $\delta$ portion of the time when $\model_0$ is true.
    \item It is \textbf{not} the probability of observing $x$ under $\model_0$.
    \item It is \textbf{not} the probability of $\model_0$ given $x$.
    \end{itemize}
  \end{alertblock}
\end{frame}
\begin{frame}
  \begin{exercise}
    \begin{itemize}
    \item<1-> Let us throw a coin 8 times, and try and predict the outcome.
    \item<2-> Select a $p$-value threshold so that $\delta = 0.05$. 
      For 8 throws, this corresponds to \uncover<3->{$ > 6$ successes or $\geq 87.5\%$ success rate}.
    \item<3-> Let's calculate the $p$-value for each one of you
    \item<4-> What is the rejection performance of the test?
    \end{itemize}
    \setlength\fheight{0.25\columnwidth}
    \setlength\fwidth{0.5\columnwidth}
    \only<2,3>{
      \begin{figure}[H]
        % This file was created by matlab2tikz.
%
%The latest updates can be retrieved from
%  http://www.mathworks.com/matlabcentral/fileexchange/22022-matlab2tikz-matlab2tikz
%where you can also make suggestions and rate matlab2tikz.
%
\begin{tikzpicture}

\begin{axis}[%
width=0.951\fwidth,
height=\fheight,
at={(0\fwidth,0\fheight)},
scale only axis,
xmode=log,
xmin=1,
xmax=1000,
xminorticks=true,
xlabel={Amount of throws},
ymin=0.5,
ymax=1,
ylabel={Success rate},
axis background/.style={fill=white},
title={The rejection threshold as data increases}
]
\addplot [color=blue, forget plot]
  table[row sep=crcr]{%
1	1\\
2	1\\
3	1\\
4	1\\
5	0.8\\
6	0.833333333333333\\
7	0.857142857142857\\
8	0.75\\
9	0.777777777777778\\
10	0.8\\
11	0.727272727272727\\
12	0.75\\
13	0.692307692307692\\
14	0.714285714285714\\
15	0.733333333333333\\
16	0.6875\\
17	0.705882352941177\\
18	0.666666666666667\\
19	0.684210526315789\\
20	0.7\\
21	0.666666666666667\\
22	0.681818181818182\\
23	0.652173913043478\\
24	0.666666666666667\\
25	0.68\\
26	0.653846153846154\\
27	0.666666666666667\\
28	0.642857142857143\\
29	0.655172413793103\\
30	0.633333333333333\\
31	0.645161290322581\\
32	0.65625\\
33	0.636363636363636\\
34	0.647058823529412\\
35	0.628571428571429\\
36	0.638888888888889\\
37	0.621621621621622\\
38	0.631578947368421\\
39	0.641025641025641\\
40	0.625\\
41	0.634146341463415\\
42	0.619047619047619\\
43	0.627906976744186\\
44	0.613636363636364\\
45	0.622222222222222\\
46	0.630434782608696\\
47	0.617021276595745\\
48	0.625\\
49	0.612244897959184\\
50	0.62\\
51	0.607843137254902\\
52	0.615384615384615\\
53	0.60377358490566\\
54	0.611111111111111\\
55	0.618181818181818\\
56	0.607142857142857\\
57	0.614035087719298\\
58	0.603448275862069\\
59	0.610169491525424\\
60	0.6\\
61	0.60655737704918\\
62	0.596774193548387\\
63	0.603174603174603\\
64	0.609375\\
65	0.6\\
66	0.606060606060606\\
67	0.597014925373134\\
68	0.602941176470588\\
69	0.594202898550725\\
70	0.6\\
71	0.591549295774648\\
72	0.597222222222222\\
73	0.602739726027397\\
74	0.594594594594595\\
75	0.6\\
76	0.592105263157895\\
77	0.597402597402597\\
78	0.58974358974359\\
79	0.594936708860759\\
80	0.5875\\
81	0.592592592592593\\
82	0.585365853658537\\
83	0.590361445783133\\
84	0.595238095238095\\
85	0.588235294117647\\
86	0.593023255813954\\
87	0.586206896551724\\
88	0.590909090909091\\
89	0.584269662921348\\
90	0.588888888888889\\
91	0.582417582417582\\
92	0.58695652173913\\
93	0.580645161290323\\
94	0.585106382978723\\
95	0.589473684210526\\
96	0.583333333333333\\
97	0.587628865979381\\
98	0.581632653061224\\
99	0.585858585858586\\
100	0.58\\
101	0.584158415841584\\
102	0.57843137254902\\
103	0.58252427184466\\
104	0.576923076923077\\
105	0.580952380952381\\
106	0.575471698113208\\
107	0.579439252336449\\
108	0.583333333333333\\
109	0.577981651376147\\
110	0.581818181818182\\
111	0.576576576576577\\
112	0.580357142857143\\
113	0.575221238938053\\
114	0.578947368421053\\
115	0.573913043478261\\
116	0.577586206896552\\
117	0.572649572649573\\
118	0.576271186440678\\
119	0.571428571428571\\
120	0.575\\
121	0.578512396694215\\
122	0.573770491803279\\
123	0.577235772357724\\
124	0.57258064516129\\
125	0.576\\
126	0.571428571428571\\
127	0.574803149606299\\
128	0.5703125\\
129	0.573643410852713\\
130	0.569230769230769\\
131	0.572519083969466\\
132	0.568181818181818\\
133	0.571428571428571\\
134	0.574626865671642\\
135	0.57037037037037\\
136	0.573529411764706\\
137	0.569343065693431\\
138	0.572463768115942\\
139	0.568345323741007\\
140	0.571428571428571\\
141	0.567375886524823\\
142	0.570422535211268\\
143	0.566433566433566\\
144	0.569444444444444\\
145	0.56551724137931\\
146	0.568493150684932\\
147	0.564625850340136\\
148	0.567567567567568\\
149	0.570469798657718\\
150	0.566666666666667\\
151	0.56953642384106\\
152	0.565789473684211\\
153	0.568627450980392\\
154	0.564935064935065\\
155	0.567741935483871\\
156	0.564102564102564\\
157	0.56687898089172\\
158	0.563291139240506\\
159	0.566037735849057\\
160	0.5625\\
161	0.565217391304348\\
162	0.561728395061728\\
163	0.56441717791411\\
164	0.567073170731707\\
165	0.563636363636364\\
166	0.566265060240964\\
167	0.562874251497006\\
168	0.56547619047619\\
169	0.562130177514793\\
170	0.564705882352941\\
171	0.56140350877193\\
172	0.563953488372093\\
173	0.560693641618497\\
174	0.563218390804598\\
175	0.56\\
176	0.5625\\
177	0.559322033898305\\
178	0.561797752808989\\
179	0.558659217877095\\
180	0.561111111111111\\
181	0.56353591160221\\
182	0.56043956043956\\
183	0.562841530054645\\
184	0.559782608695652\\
185	0.562162162162162\\
186	0.559139784946237\\
187	0.561497326203209\\
188	0.558510638297872\\
189	0.560846560846561\\
190	0.557894736842105\\
191	0.56020942408377\\
192	0.557291666666667\\
193	0.559585492227979\\
194	0.556701030927835\\
195	0.558974358974359\\
196	0.561224489795918\\
197	0.558375634517767\\
198	0.560606060606061\\
199	0.557788944723618\\
200	0.56\\
201	0.557213930348259\\
202	0.559405940594059\\
203	0.556650246305419\\
204	0.558823529411765\\
205	0.55609756097561\\
206	0.558252427184466\\
207	0.555555555555556\\
208	0.557692307692308\\
209	0.555023923444976\\
210	0.557142857142857\\
211	0.554502369668246\\
212	0.556603773584906\\
213	0.553990610328638\\
214	0.55607476635514\\
215	0.558139534883721\\
216	0.555555555555556\\
217	0.557603686635945\\
218	0.555045871559633\\
219	0.557077625570776\\
220	0.554545454545455\\
221	0.556561085972851\\
222	0.554054054054054\\
223	0.556053811659193\\
224	0.553571428571429\\
225	0.555555555555556\\
226	0.553097345132743\\
227	0.555066079295154\\
228	0.552631578947368\\
229	0.554585152838428\\
230	0.552173913043478\\
231	0.554112554112554\\
232	0.556034482758621\\
233	0.553648068669528\\
234	0.555555555555556\\
235	0.553191489361702\\
236	0.555084745762712\\
237	0.552742616033755\\
238	0.554621848739496\\
239	0.552301255230126\\
240	0.554166666666667\\
241	0.551867219917012\\
242	0.553719008264463\\
243	0.551440329218107\\
244	0.55327868852459\\
245	0.551020408163265\\
246	0.552845528455285\\
247	0.550607287449393\\
248	0.55241935483871\\
249	0.550200803212851\\
250	0.552\\
251	0.553784860557769\\
252	0.551587301587302\\
253	0.553359683794466\\
254	0.551181102362205\\
255	0.552941176470588\\
256	0.55078125\\
257	0.552529182879377\\
258	0.550387596899225\\
259	0.552123552123552\\
260	0.55\\
261	0.551724137931034\\
262	0.549618320610687\\
263	0.551330798479088\\
264	0.549242424242424\\
265	0.550943396226415\\
266	0.548872180451128\\
267	0.550561797752809\\
268	0.548507462686567\\
269	0.550185873605948\\
270	0.551851851851852\\
271	0.549815498154982\\
272	0.551470588235294\\
273	0.549450549450549\\
274	0.551094890510949\\
275	0.549090909090909\\
276	0.550724637681159\\
277	0.548736462093863\\
278	0.550359712230216\\
279	0.548387096774194\\
280	0.55\\
281	0.548042704626335\\
282	0.549645390070922\\
283	0.547703180212014\\
284	0.549295774647887\\
285	0.547368421052632\\
286	0.548951048951049\\
287	0.547038327526132\\
288	0.548611111111111\\
289	0.546712802768166\\
290	0.548275862068966\\
291	0.549828178694158\\
292	0.547945205479452\\
293	0.549488054607508\\
294	0.547619047619048\\
295	0.549152542372881\\
296	0.547297297297297\\
297	0.548821548821549\\
298	0.546979865771812\\
299	0.548494983277592\\
300	0.546666666666667\\
301	0.548172757475083\\
302	0.54635761589404\\
303	0.547854785478548\\
304	0.546052631578947\\
305	0.547540983606557\\
306	0.545751633986928\\
307	0.547231270358306\\
308	0.545454545454545\\
309	0.546925566343042\\
310	0.545161290322581\\
311	0.546623794212219\\
312	0.548076923076923\\
313	0.546325878594249\\
314	0.547770700636943\\
315	0.546031746031746\\
316	0.54746835443038\\
317	0.545741324921136\\
318	0.547169811320755\\
319	0.545454545454545\\
320	0.546875\\
321	0.545171339563863\\
322	0.546583850931677\\
323	0.544891640866873\\
324	0.546296296296296\\
325	0.544615384615385\\
326	0.54601226993865\\
327	0.54434250764526\\
328	0.545731707317073\\
329	0.544072948328267\\
330	0.545454545454545\\
331	0.54380664652568\\
332	0.545180722891566\\
333	0.546546546546547\\
334	0.544910179640719\\
335	0.546268656716418\\
336	0.544642857142857\\
337	0.545994065281899\\
338	0.544378698224852\\
339	0.545722713864307\\
340	0.544117647058823\\
341	0.545454545454545\\
342	0.543859649122807\\
343	0.545189504373178\\
344	0.543604651162791\\
345	0.544927536231884\\
346	0.543352601156069\\
347	0.544668587896254\\
348	0.543103448275862\\
349	0.544412607449857\\
350	0.542857142857143\\
351	0.544159544159544\\
352	0.542613636363636\\
353	0.543909348441926\\
354	0.542372881355932\\
355	0.543661971830986\\
356	0.544943820224719\\
357	0.543417366946779\\
358	0.544692737430168\\
359	0.543175487465181\\
360	0.544444444444444\\
361	0.542936288088643\\
362	0.544198895027624\\
363	0.542699724517906\\
364	0.543956043956044\\
365	0.542465753424658\\
366	0.543715846994536\\
367	0.542234332425068\\
368	0.543478260869565\\
369	0.5420054200542\\
370	0.543243243243243\\
371	0.54177897574124\\
372	0.543010752688172\\
373	0.541554959785523\\
374	0.542780748663102\\
375	0.541333333333333\\
376	0.542553191489362\\
377	0.541114058355438\\
378	0.542328042328042\\
379	0.54353562005277\\
380	0.542105263157895\\
381	0.543307086614173\\
382	0.541884816753927\\
383	0.543080939947781\\
384	0.541666666666667\\
385	0.542857142857143\\
386	0.541450777202073\\
387	0.542635658914729\\
388	0.541237113402062\\
389	0.542416452442159\\
390	0.541025641025641\\
391	0.542199488491049\\
392	0.540816326530612\\
393	0.541984732824427\\
394	0.540609137055838\\
395	0.541772151898734\\
396	0.54040404040404\\
397	0.541561712846348\\
398	0.540201005025126\\
399	0.541353383458647\\
400	0.54\\
401	0.541147132169576\\
402	0.539800995024876\\
403	0.540942928039702\\
404	0.542079207920792\\
405	0.540740740740741\\
406	0.541871921182266\\
407	0.540540540540541\\
408	0.541666666666667\\
409	0.540342298288509\\
410	0.541463414634146\\
411	0.54014598540146\\
412	0.54126213592233\\
413	0.539951573849879\\
414	0.541062801932367\\
415	0.539759036144578\\
416	0.540865384615385\\
417	0.539568345323741\\
418	0.54066985645933\\
419	0.539379474940334\\
420	0.54047619047619\\
421	0.539192399049881\\
422	0.540284360189573\\
423	0.539007092198582\\
424	0.540094339622642\\
425	0.538823529411765\\
426	0.539906103286385\\
427	0.53864168618267\\
428	0.539719626168224\\
429	0.540792540792541\\
430	0.53953488372093\\
431	0.540603248259861\\
432	0.539351851851852\\
433	0.540415704387991\\
434	0.539170506912442\\
435	0.540229885057471\\
436	0.538990825688073\\
437	0.540045766590389\\
438	0.538812785388128\\
439	0.539863325740319\\
440	0.538636363636364\\
441	0.53968253968254\\
442	0.538461538461538\\
443	0.539503386004515\\
444	0.538288288288288\\
445	0.539325842696629\\
446	0.538116591928251\\
447	0.539149888143177\\
448	0.537946428571429\\
449	0.538975501113586\\
450	0.537777777777778\\
451	0.53880266075388\\
452	0.537610619469027\\
453	0.538631346578366\\
454	0.539647577092511\\
455	0.538461538461538\\
456	0.539473684210526\\
457	0.538293216630197\\
458	0.539301310043668\\
459	0.538126361655773\\
460	0.539130434782609\\
461	0.537960954446855\\
462	0.538961038961039\\
463	0.537796976241901\\
464	0.538793103448276\\
465	0.537634408602151\\
466	0.53862660944206\\
467	0.537473233404711\\
468	0.538461538461538\\
469	0.537313432835821\\
470	0.538297872340426\\
471	0.537154989384289\\
472	0.538135593220339\\
473	0.536997885835095\\
474	0.537974683544304\\
475	0.536842105263158\\
476	0.53781512605042\\
477	0.536687631027254\\
478	0.53765690376569\\
479	0.536534446764092\\
480	0.5375\\
481	0.538461538461538\\
482	0.537344398340249\\
483	0.538302277432712\\
484	0.537190082644628\\
485	0.538144329896907\\
486	0.537037037037037\\
487	0.537987679671458\\
488	0.536885245901639\\
489	0.537832310838446\\
490	0.536734693877551\\
491	0.537678207739308\\
492	0.536585365853659\\
493	0.537525354969574\\
494	0.536437246963563\\
495	0.537373737373737\\
496	0.536290322580645\\
497	0.537223340040241\\
498	0.536144578313253\\
499	0.537074148296593\\
500	0.536\\
501	0.536926147704591\\
502	0.535856573705179\\
503	0.536779324055666\\
504	0.535714285714286\\
505	0.536633663366337\\
506	0.535573122529644\\
507	0.536489151873767\\
508	0.53740157480315\\
509	0.536345776031434\\
510	0.537254901960784\\
511	0.536203522504892\\
512	0.537109375\\
513	0.536062378167641\\
514	0.536964980544747\\
515	0.535922330097087\\
516	0.536821705426357\\
517	0.5357833655706\\
518	0.536679536679537\\
519	0.535645472061657\\
520	0.536538461538462\\
521	0.535508637236084\\
522	0.53639846743295\\
523	0.535372848948375\\
524	0.536259541984733\\
525	0.535238095238095\\
526	0.536121673003802\\
527	0.535104364326376\\
528	0.535984848484849\\
529	0.534971644612476\\
530	0.535849056603774\\
531	0.534839924670433\\
532	0.535714285714286\\
533	0.534709193245779\\
534	0.535580524344569\\
535	0.536448598130841\\
536	0.53544776119403\\
537	0.536312849162011\\
538	0.535315985130111\\
539	0.536178107606679\\
540	0.535185185185185\\
541	0.536044362292052\\
542	0.535055350553506\\
543	0.535911602209945\\
544	0.534926470588235\\
545	0.535779816513761\\
546	0.534798534798535\\
547	0.535648994515539\\
548	0.534671532846715\\
549	0.53551912568306\\
550	0.534545454545455\\
551	0.535390199637024\\
552	0.534420289855073\\
553	0.535262206148282\\
554	0.534296028880866\\
555	0.535135135135135\\
556	0.534172661870504\\
557	0.535008976660682\\
558	0.53405017921147\\
559	0.534883720930233\\
560	0.533928571428571\\
561	0.53475935828877\\
562	0.533807829181495\\
563	0.534635879218472\\
564	0.535460992907801\\
565	0.534513274336283\\
566	0.535335689045936\\
567	0.534391534391534\\
568	0.535211267605634\\
569	0.53427065026362\\
570	0.535087719298246\\
571	0.53415061295972\\
572	0.534965034965035\\
573	0.534031413612565\\
574	0.534843205574913\\
575	0.533913043478261\\
576	0.534722222222222\\
577	0.533795493934142\\
578	0.534602076124567\\
579	0.533678756476684\\
580	0.53448275862069\\
581	0.533562822719449\\
582	0.534364261168385\\
583	0.533447684391081\\
584	0.534246575342466\\
585	0.533333333333333\\
586	0.534129692832764\\
587	0.533219761499148\\
588	0.534013605442177\\
589	0.533106960950764\\
590	0.533898305084746\\
591	0.532994923857868\\
592	0.533783783783784\\
593	0.534569983136594\\
594	0.533670033670034\\
595	0.534453781512605\\
596	0.533557046979866\\
597	0.534338358458961\\
598	0.533444816053512\\
599	0.534223706176962\\
600	0.533333333333333\\
601	0.534109816971714\\
602	0.533222591362126\\
603	0.533996683250415\\
604	0.533112582781457\\
605	0.533884297520661\\
606	0.533003300330033\\
607	0.533772652388797\\
608	0.532894736842105\\
609	0.533661740558292\\
610	0.532786885245902\\
611	0.533551554828151\\
612	0.532679738562091\\
613	0.533442088091354\\
614	0.53257328990228\\
615	0.533333333333333\\
616	0.532467532467532\\
617	0.53322528363047\\
618	0.532362459546926\\
619	0.533117932148627\\
620	0.532258064516129\\
621	0.533011272141707\\
622	0.533762057877814\\
623	0.532905296950241\\
624	0.533653846153846\\
625	0.5328\\
626	0.533546325878594\\
627	0.532695374800638\\
628	0.53343949044586\\
629	0.532591414944356\\
630	0.533333333333333\\
631	0.532488114104596\\
632	0.533227848101266\\
633	0.532385466034755\\
634	0.533123028391167\\
635	0.532283464566929\\
636	0.533018867924528\\
637	0.532182103610675\\
638	0.532915360501567\\
639	0.5320813771518\\
640	0.5328125\\
641	0.53198127925117\\
642	0.532710280373832\\
643	0.531881804043546\\
644	0.532608695652174\\
645	0.531782945736434\\
646	0.532507739938081\\
647	0.531684698608965\\
648	0.532407407407407\\
649	0.531587057010786\\
650	0.532307692307692\\
651	0.531490015360983\\
652	0.532208588957055\\
653	0.532924961715161\\
654	0.532110091743119\\
655	0.532824427480916\\
656	0.532012195121951\\
657	0.532724505327245\\
658	0.531914893617021\\
659	0.532625189681335\\
660	0.531818181818182\\
661	0.532526475037821\\
662	0.531722054380665\\
663	0.532428355957768\\
664	0.531626506024096\\
665	0.532330827067669\\
666	0.531531531531532\\
667	0.532233883058471\\
668	0.531437125748503\\
669	0.532137518684604\\
670	0.53134328358209\\
671	0.53204172876304\\
672	0.53125\\
673	0.531946508172363\\
674	0.531157270029674\\
675	0.531851851851852\\
676	0.531065088757396\\
677	0.531757754800591\\
678	0.530973451327434\\
679	0.531664212076583\\
680	0.530882352941176\\
681	0.531571218795888\\
682	0.530791788856305\\
683	0.531478770131772\\
684	0.532163742690059\\
685	0.531386861313869\\
686	0.532069970845481\\
687	0.531295487627365\\
688	0.531976744186046\\
689	0.531204644412192\\
690	0.531884057971014\\
691	0.531114327062229\\
692	0.531791907514451\\
693	0.531024531024531\\
694	0.531700288184438\\
695	0.530935251798561\\
696	0.531609195402299\\
697	0.530846484935438\\
698	0.531518624641834\\
699	0.530758226037196\\
700	0.531428571428571\\
701	0.530670470756063\\
702	0.531339031339031\\
703	0.530583214793741\\
704	0.53125\\
705	0.530496453900709\\
706	0.531161473087819\\
707	0.53041018387553\\
708	0.531073446327684\\
709	0.530324400564175\\
710	0.530985915492958\\
711	0.530239099859353\\
712	0.530898876404494\\
713	0.53015427769986\\
714	0.530812324929972\\
715	0.53006993006993\\
716	0.53072625698324\\
717	0.531380753138075\\
718	0.530640668523677\\
719	0.531293463143255\\
720	0.530555555555556\\
721	0.53120665742025\\
722	0.530470914127424\\
723	0.531120331950207\\
724	0.530386740331492\\
725	0.531034482758621\\
726	0.53030303030303\\
727	0.530949105914718\\
728	0.53021978021978\\
729	0.530864197530864\\
730	0.53013698630137\\
731	0.53077975376197\\
732	0.530054644808743\\
733	0.530695770804911\\
734	0.529972752043597\\
735	0.530612244897959\\
736	0.529891304347826\\
737	0.530529172320217\\
738	0.529810298102981\\
739	0.530446549391069\\
740	0.52972972972973\\
741	0.530364372469636\\
742	0.529649595687331\\
743	0.53028263795424\\
744	0.529569892473118\\
745	0.530201342281879\\
746	0.529490616621984\\
747	0.530120481927711\\
748	0.529411764705882\\
749	0.530040053404539\\
750	0.530666666666667\\
751	0.529960053262317\\
752	0.530585106382979\\
753	0.529880478087649\\
754	0.530503978779841\\
755	0.529801324503311\\
756	0.53042328042328\\
757	0.529722589167767\\
758	0.530343007915567\\
759	0.529644268774704\\
760	0.530263157894737\\
761	0.529566360052562\\
762	0.530183727034121\\
763	0.529488859764089\\
764	0.530104712041885\\
765	0.529411764705882\\
766	0.530026109660574\\
767	0.529335071707953\\
768	0.529947916666667\\
769	0.52925877763329\\
770	0.52987012987013\\
771	0.529182879377432\\
772	0.52979274611399\\
773	0.529107373868047\\
774	0.529715762273902\\
775	0.529032258064516\\
776	0.529639175257732\\
777	0.528957528957529\\
778	0.529562982005141\\
779	0.528883183568678\\
780	0.529487179487179\\
781	0.528809218950064\\
782	0.529411764705882\\
783	0.530012771392082\\
784	0.529336734693878\\
785	0.529936305732484\\
786	0.529262086513995\\
787	0.529860228716646\\
788	0.529187817258883\\
789	0.5297845373891\\
790	0.529113924050633\\
791	0.529709228824273\\
792	0.529040404040404\\
793	0.529634300126103\\
794	0.52896725440806\\
795	0.529559748427673\\
796	0.528894472361809\\
797	0.529485570890841\\
798	0.528822055137845\\
799	0.529411764705882\\
800	0.52875\\
801	0.529338327091136\\
802	0.528678304239402\\
803	0.529265255292653\\
804	0.528606965174129\\
805	0.529192546583851\\
806	0.528535980148883\\
807	0.52912019826518\\
808	0.528465346534653\\
809	0.529048207663782\\
810	0.528395061728395\\
811	0.528976572133169\\
812	0.528325123152709\\
813	0.528905289052891\\
814	0.528255528255528\\
815	0.528834355828221\\
816	0.528186274509804\\
817	0.528763769889841\\
818	0.529339853300734\\
819	0.528693528693529\\
820	0.529268292682927\\
821	0.528623629719854\\
822	0.529197080291971\\
823	0.528554070473876\\
824	0.529126213592233\\
825	0.528484848484848\\
826	0.529055690072639\\
827	0.528415961305925\\
828	0.528985507246377\\
829	0.528347406513872\\
830	0.528915662650602\\
831	0.528279181708785\\
832	0.528846153846154\\
833	0.528211284513806\\
834	0.528776978417266\\
835	0.52814371257485\\
836	0.528708133971292\\
837	0.528076463560335\\
838	0.528639618138425\\
839	0.528009535160906\\
840	0.528571428571429\\
841	0.52794292508918\\
842	0.528503562945368\\
843	0.527876631079478\\
844	0.528436018957346\\
845	0.527810650887574\\
846	0.528368794326241\\
847	0.527744982290437\\
848	0.528301886792453\\
849	0.527679623085983\\
850	0.528235294117647\\
851	0.527614571092832\\
852	0.528169014084507\\
853	0.528722157092614\\
854	0.528103044496487\\
855	0.528654970760234\\
856	0.52803738317757\\
857	0.528588098016336\\
858	0.527972027972028\\
859	0.528521536670547\\
860	0.527906976744186\\
861	0.528455284552846\\
862	0.52784222737819\\
863	0.528389339513326\\
864	0.527777777777778\\
865	0.528323699421965\\
866	0.527713625866051\\
867	0.528258362168397\\
868	0.527649769585253\\
869	0.52819332566168\\
870	0.527586206896552\\
871	0.52812858783008\\
872	0.527522935779816\\
873	0.528064146620848\\
874	0.52745995423341\\
875	0.528\\
876	0.527397260273973\\
877	0.527936145952109\\
878	0.527334851936219\\
879	0.527872582480091\\
880	0.527272727272727\\
881	0.527809307604994\\
882	0.527210884353742\\
883	0.527746319365798\\
884	0.527149321266968\\
885	0.527683615819209\\
886	0.527088036117382\\
887	0.527621195039459\\
888	0.528153153153153\\
889	0.52755905511811\\
890	0.528089887640449\\
891	0.527497194163861\\
892	0.528026905829596\\
893	0.527435610302352\\
894	0.527964205816555\\
895	0.527374301675978\\
896	0.527901785714286\\
897	0.527313266443701\\
898	0.527839643652561\\
899	0.527252502780868\\
900	0.527777777777778\\
901	0.527192008879023\\
902	0.527716186252772\\
903	0.527131782945736\\
904	0.527654867256637\\
905	0.52707182320442\\
906	0.527593818984547\\
907	0.527012127894157\\
908	0.527533039647577\\
909	0.526952695269527\\
910	0.527472527472527\\
911	0.526893523600439\\
912	0.527412280701754\\
913	0.526834611171961\\
914	0.527352297592998\\
915	0.526775956284153\\
916	0.527292576419214\\
917	0.526717557251908\\
918	0.52723311546841\\
919	0.526659412404788\\
920	0.527173913043478\\
921	0.526601520086862\\
922	0.527114967462039\\
923	0.526543878656555\\
924	0.527056277056277\\
925	0.527567567567568\\
926	0.526997840172786\\
927	0.527508090614887\\
928	0.526939655172414\\
929	0.527448869752422\\
930	0.526881720430108\\
931	0.527389903329753\\
932	0.526824034334764\\
933	0.527331189710611\\
934	0.526766595289079\\
935	0.527272727272727\\
936	0.526709401709402\\
937	0.527214514407684\\
938	0.526652452025586\\
939	0.527156549520767\\
940	0.526595744680851\\
941	0.527098831030818\\
942	0.526539278131635\\
943	0.527041357370095\\
944	0.526483050847458\\
945	0.526984126984127\\
946	0.526427061310782\\
947	0.526927138331573\\
948	0.526371308016878\\
949	0.526870389884089\\
950	0.526315789473684\\
951	0.526813880126183\\
952	0.526260504201681\\
953	0.526757607555089\\
954	0.526205450733753\\
955	0.526701570680628\\
956	0.526150627615063\\
957	0.526645768025078\\
958	0.526096033402923\\
959	0.526590198123045\\
960	0.526041666666667\\
961	0.526534859521332\\
962	0.527027027027027\\
963	0.526479750778816\\
964	0.526970954356846\\
965	0.526424870466321\\
966	0.526915113871636\\
967	0.526370217166494\\
968	0.526859504132231\\
969	0.526315789473684\\
970	0.52680412371134\\
971	0.526261585993821\\
972	0.526748971193416\\
973	0.526207605344296\\
974	0.526694045174538\\
975	0.526153846153846\\
976	0.526639344262295\\
977	0.526100307062436\\
978	0.526584867075665\\
979	0.526046986721144\\
980	0.526530612244898\\
981	0.525993883792049\\
982	0.526476578411405\\
983	0.525940996948118\\
984	0.526422764227642\\
985	0.525888324873096\\
986	0.526369168356998\\
987	0.525835866261398\\
988	0.526315789473684\\
989	0.525783619817998\\
990	0.526262626262626\\
991	0.525731584258325\\
992	0.526209677419355\\
993	0.525679758308157\\
994	0.526156941649899\\
995	0.525628140703518\\
996	0.526104417670683\\
997	0.525576730190572\\
998	0.526052104208417\\
999	0.525525525525526\\
1000	0.526\\
};
\end{axis}
\end{tikzpicture}%
        \caption{Here we see how the rejection threshold, in terms of the success rate, changes with the number of throws to achieve an error rate of $\delta = 0.05$.}
      \end{figure}
      \only<article>{As the amount of throws goes to infinity, the threshold converges to $0.5$. This means that a statistically significant difference from the null hypothesis can be obtained, even when the actual model from which the data is drawn is only slightly different from 0.5.}
    }
    \only<4>{
      \begin{figure}[H]
        % This file was created by matlab2tikz.
%
%The latest updates can be retrieved from
%  http://www.mathworks.com/matlabcentral/fileexchange/22022-matlab2tikz-matlab2tikz
%where you can also make suggestions and rate matlab2tikz.
%
\begin{tikzpicture}

\begin{axis}[%
width=0.951\fwidth,
height=\fheight,
at={(0\fwidth,0\fheight)},
scale only axis,
xmin=0,
xmax=1000,
ymin=0,
ymax=1,
axis background/.style={fill=white},
title={How often we reject the null hypothesis},
legend style={legend cell align=left, align=left, legend plot pos=left, draw=black}
]
\addplot [color=blue]
  table[row sep=crcr]{%
1	0\\
2	0\\
3	0\\
4	0\\
5	0.042\\
6	0.022\\
7	0.014\\
8	0.042\\
9	0.027\\
10	0.014\\
11	0.04\\
12	0.025\\
13	0.061\\
14	0.036\\
15	0.018\\
16	0.047\\
17	0.026\\
18	0.055\\
19	0.039\\
20	0.026\\
21	0.044\\
22	0.035\\
23	0.05\\
24	0.035\\
25	0.025\\
26	0.045\\
27	0.034\\
28	0.054\\
29	0.038\\
30	0.065\\
31	0.046\\
32	0.028\\
33	0.052\\
34	0.03\\
35	0.05\\
36	0.031\\
37	0.053\\
38	0.036\\
39	0.026\\
40	0.046\\
41	0.031\\
42	0.053\\
43	0.038\\
44	0.055\\
45	0.039\\
46	0.024\\
47	0.036\\
48	0.026\\
49	0.037\\
50	0.027\\
51	0.039\\
52	0.032\\
53	0.047\\
54	0.038\\
55	0.027\\
56	0.041\\
57	0.033\\
58	0.048\\
59	0.033\\
60	0.045\\
61	0.037\\
62	0.06\\
63	0.04\\
64	0.029\\
65	0.045\\
66	0.036\\
67	0.05\\
68	0.041\\
69	0.053\\
70	0.046\\
71	0.058\\
72	0.044\\
73	0.036\\
74	0.045\\
75	0.039\\
76	0.046\\
77	0.04\\
78	0.054\\
79	0.045\\
80	0.054\\
81	0.046\\
82	0.053\\
83	0.043\\
84	0.029\\
85	0.041\\
86	0.03\\
87	0.042\\
88	0.035\\
89	0.046\\
90	0.033\\
91	0.042\\
92	0.035\\
93	0.051\\
94	0.044\\
95	0.04\\
96	0.05\\
97	0.036\\
98	0.047\\
99	0.036\\
100	0.047\\
101	0.032\\
102	0.047\\
103	0.038\\
104	0.049\\
105	0.041\\
106	0.053\\
107	0.043\\
108	0.036\\
109	0.047\\
110	0.042\\
111	0.048\\
112	0.04\\
113	0.045\\
114	0.038\\
115	0.047\\
116	0.043\\
117	0.049\\
118	0.035\\
119	0.048\\
120	0.04\\
121	0.026\\
122	0.037\\
123	0.026\\
124	0.035\\
125	0.027\\
126	0.043\\
127	0.031\\
128	0.038\\
129	0.035\\
130	0.045\\
131	0.039\\
132	0.051\\
133	0.046\\
134	0.038\\
135	0.044\\
136	0.037\\
137	0.044\\
138	0.033\\
139	0.048\\
140	0.038\\
141	0.046\\
142	0.037\\
143	0.047\\
144	0.041\\
145	0.046\\
146	0.045\\
147	0.052\\
148	0.047\\
149	0.041\\
150	0.043\\
151	0.037\\
152	0.05\\
153	0.046\\
154	0.05\\
155	0.045\\
156	0.05\\
157	0.044\\
158	0.05\\
159	0.047\\
160	0.052\\
161	0.045\\
162	0.05\\
163	0.046\\
164	0.037\\
165	0.046\\
166	0.037\\
167	0.05\\
168	0.041\\
169	0.047\\
170	0.041\\
171	0.053\\
172	0.048\\
173	0.051\\
174	0.046\\
175	0.049\\
176	0.046\\
177	0.054\\
178	0.047\\
179	0.053\\
180	0.044\\
181	0.036\\
182	0.043\\
183	0.038\\
184	0.041\\
185	0.04\\
186	0.046\\
187	0.041\\
188	0.048\\
189	0.045\\
190	0.05\\
191	0.044\\
192	0.048\\
193	0.043\\
194	0.052\\
195	0.043\\
196	0.036\\
197	0.045\\
198	0.039\\
199	0.047\\
200	0.041\\
201	0.047\\
202	0.04\\
203	0.048\\
204	0.041\\
205	0.05\\
206	0.042\\
207	0.047\\
208	0.04\\
209	0.046\\
210	0.039\\
211	0.044\\
212	0.039\\
213	0.047\\
214	0.041\\
215	0.034\\
216	0.043\\
217	0.037\\
218	0.044\\
219	0.035\\
220	0.042\\
221	0.04\\
222	0.045\\
223	0.038\\
224	0.049\\
225	0.04\\
226	0.047\\
227	0.041\\
228	0.051\\
229	0.041\\
230	0.052\\
231	0.044\\
232	0.035\\
233	0.04\\
234	0.036\\
235	0.039\\
236	0.034\\
237	0.041\\
238	0.037\\
239	0.041\\
240	0.035\\
241	0.04\\
242	0.034\\
243	0.043\\
244	0.035\\
245	0.04\\
246	0.037\\
247	0.041\\
248	0.035\\
249	0.041\\
250	0.033\\
251	0.033\\
252	0.038\\
253	0.036\\
254	0.039\\
255	0.033\\
256	0.038\\
257	0.036\\
258	0.041\\
259	0.035\\
260	0.041\\
261	0.031\\
262	0.04\\
263	0.033\\
264	0.042\\
265	0.034\\
266	0.044\\
267	0.038\\
268	0.043\\
269	0.033\\
270	0.028\\
271	0.036\\
272	0.028\\
273	0.033\\
274	0.027\\
275	0.033\\
276	0.028\\
277	0.033\\
278	0.028\\
279	0.038\\
280	0.03\\
281	0.039\\
282	0.03\\
283	0.038\\
284	0.031\\
285	0.038\\
286	0.035\\
287	0.037\\
288	0.032\\
289	0.038\\
290	0.031\\
291	0.028\\
292	0.033\\
293	0.032\\
294	0.033\\
295	0.031\\
296	0.035\\
297	0.03\\
298	0.038\\
299	0.036\\
300	0.038\\
301	0.036\\
302	0.039\\
303	0.037\\
304	0.041\\
305	0.04\\
306	0.044\\
307	0.04\\
308	0.044\\
309	0.041\\
310	0.042\\
311	0.038\\
312	0.033\\
313	0.038\\
314	0.033\\
315	0.039\\
316	0.036\\
317	0.04\\
318	0.037\\
319	0.042\\
320	0.037\\
321	0.041\\
322	0.036\\
323	0.041\\
324	0.037\\
325	0.042\\
326	0.036\\
327	0.039\\
328	0.035\\
329	0.04\\
330	0.037\\
331	0.045\\
332	0.037\\
333	0.033\\
334	0.04\\
335	0.036\\
336	0.041\\
337	0.039\\
338	0.044\\
339	0.042\\
340	0.043\\
341	0.039\\
342	0.043\\
343	0.04\\
344	0.043\\
345	0.038\\
346	0.041\\
347	0.039\\
348	0.044\\
349	0.042\\
350	0.047\\
351	0.042\\
352	0.046\\
353	0.042\\
354	0.049\\
355	0.04\\
356	0.035\\
357	0.041\\
358	0.037\\
359	0.039\\
360	0.038\\
361	0.039\\
362	0.035\\
363	0.04\\
364	0.034\\
365	0.039\\
366	0.036\\
367	0.039\\
368	0.035\\
369	0.042\\
370	0.032\\
371	0.04\\
372	0.033\\
373	0.039\\
374	0.036\\
375	0.037\\
376	0.031\\
377	0.035\\
378	0.034\\
379	0.029\\
380	0.032\\
381	0.028\\
382	0.033\\
383	0.028\\
384	0.033\\
385	0.03\\
386	0.035\\
387	0.026\\
388	0.033\\
389	0.028\\
390	0.03\\
391	0.029\\
392	0.032\\
393	0.031\\
394	0.034\\
395	0.033\\
396	0.038\\
397	0.031\\
398	0.033\\
399	0.03\\
400	0.035\\
401	0.028\\
402	0.037\\
403	0.033\\
404	0.031\\
405	0.032\\
406	0.027\\
407	0.032\\
408	0.03\\
409	0.034\\
410	0.03\\
411	0.035\\
412	0.029\\
413	0.035\\
414	0.03\\
415	0.037\\
416	0.035\\
417	0.038\\
418	0.032\\
419	0.038\\
420	0.033\\
421	0.035\\
422	0.034\\
423	0.038\\
424	0.036\\
425	0.038\\
426	0.033\\
427	0.036\\
428	0.033\\
429	0.029\\
430	0.032\\
431	0.032\\
432	0.034\\
433	0.03\\
434	0.032\\
435	0.028\\
436	0.034\\
437	0.031\\
438	0.037\\
439	0.034\\
440	0.039\\
441	0.033\\
442	0.038\\
443	0.032\\
444	0.037\\
445	0.034\\
446	0.04\\
447	0.034\\
448	0.04\\
449	0.037\\
450	0.04\\
451	0.035\\
452	0.038\\
453	0.036\\
454	0.031\\
455	0.034\\
456	0.028\\
457	0.031\\
458	0.031\\
459	0.032\\
460	0.031\\
461	0.034\\
462	0.028\\
463	0.031\\
464	0.029\\
465	0.033\\
466	0.029\\
467	0.037\\
468	0.033\\
469	0.038\\
470	0.036\\
471	0.038\\
472	0.037\\
473	0.038\\
474	0.036\\
475	0.037\\
476	0.037\\
477	0.039\\
478	0.036\\
479	0.038\\
480	0.035\\
481	0.032\\
482	0.039\\
483	0.033\\
484	0.039\\
485	0.036\\
486	0.043\\
487	0.036\\
488	0.041\\
489	0.039\\
490	0.042\\
491	0.038\\
492	0.04\\
493	0.036\\
494	0.04\\
495	0.036\\
496	0.041\\
497	0.038\\
498	0.041\\
499	0.036\\
500	0.043\\
501	0.038\\
502	0.044\\
503	0.04\\
504	0.045\\
505	0.042\\
506	0.044\\
507	0.042\\
508	0.038\\
509	0.043\\
510	0.038\\
511	0.038\\
512	0.035\\
513	0.037\\
514	0.034\\
515	0.038\\
516	0.035\\
517	0.039\\
518	0.037\\
519	0.039\\
520	0.036\\
521	0.041\\
522	0.035\\
523	0.039\\
524	0.036\\
525	0.04\\
526	0.038\\
527	0.04\\
528	0.037\\
529	0.042\\
530	0.038\\
531	0.043\\
532	0.039\\
533	0.042\\
534	0.04\\
535	0.038\\
536	0.04\\
537	0.038\\
538	0.04\\
539	0.037\\
540	0.04\\
541	0.037\\
542	0.039\\
543	0.036\\
544	0.038\\
545	0.038\\
546	0.04\\
547	0.037\\
548	0.041\\
549	0.037\\
550	0.042\\
551	0.038\\
552	0.045\\
553	0.041\\
554	0.045\\
555	0.042\\
556	0.046\\
557	0.041\\
558	0.045\\
559	0.041\\
560	0.047\\
561	0.041\\
562	0.045\\
563	0.04\\
564	0.036\\
565	0.043\\
566	0.04\\
567	0.045\\
568	0.04\\
569	0.042\\
570	0.04\\
571	0.045\\
572	0.039\\
573	0.045\\
574	0.04\\
575	0.045\\
576	0.039\\
577	0.049\\
578	0.043\\
579	0.049\\
580	0.042\\
581	0.05\\
582	0.044\\
583	0.05\\
584	0.046\\
585	0.048\\
586	0.043\\
587	0.047\\
588	0.043\\
589	0.047\\
590	0.04\\
591	0.043\\
592	0.04\\
593	0.036\\
594	0.04\\
595	0.036\\
596	0.04\\
597	0.035\\
598	0.037\\
599	0.033\\
600	0.037\\
601	0.034\\
602	0.039\\
603	0.037\\
604	0.041\\
605	0.039\\
606	0.04\\
607	0.037\\
608	0.042\\
609	0.039\\
610	0.044\\
611	0.038\\
612	0.045\\
613	0.04\\
614	0.046\\
615	0.042\\
616	0.045\\
617	0.043\\
618	0.045\\
619	0.039\\
620	0.042\\
621	0.039\\
622	0.036\\
623	0.039\\
624	0.036\\
625	0.04\\
626	0.036\\
627	0.041\\
628	0.035\\
629	0.041\\
630	0.035\\
631	0.041\\
632	0.038\\
633	0.04\\
634	0.037\\
635	0.04\\
636	0.037\\
637	0.042\\
638	0.037\\
639	0.04\\
640	0.037\\
641	0.041\\
642	0.036\\
643	0.042\\
644	0.038\\
645	0.042\\
646	0.036\\
647	0.041\\
648	0.039\\
649	0.041\\
650	0.038\\
651	0.044\\
652	0.041\\
653	0.038\\
654	0.04\\
655	0.04\\
656	0.041\\
657	0.04\\
658	0.041\\
659	0.038\\
660	0.042\\
661	0.038\\
662	0.039\\
663	0.036\\
664	0.042\\
665	0.039\\
666	0.04\\
667	0.039\\
668	0.042\\
669	0.039\\
670	0.04\\
671	0.04\\
672	0.041\\
673	0.038\\
674	0.04\\
675	0.038\\
676	0.042\\
677	0.041\\
678	0.043\\
679	0.042\\
680	0.043\\
681	0.041\\
682	0.042\\
683	0.04\\
684	0.039\\
685	0.041\\
686	0.039\\
687	0.042\\
688	0.037\\
689	0.041\\
690	0.036\\
691	0.04\\
692	0.036\\
693	0.041\\
694	0.038\\
695	0.042\\
696	0.039\\
697	0.043\\
698	0.038\\
699	0.041\\
700	0.035\\
701	0.039\\
702	0.036\\
703	0.038\\
704	0.034\\
705	0.041\\
706	0.034\\
707	0.04\\
708	0.038\\
709	0.042\\
710	0.037\\
711	0.04\\
712	0.037\\
713	0.043\\
714	0.041\\
715	0.043\\
716	0.042\\
717	0.039\\
718	0.042\\
719	0.039\\
720	0.041\\
721	0.037\\
722	0.041\\
723	0.038\\
724	0.043\\
725	0.041\\
726	0.044\\
727	0.041\\
728	0.045\\
729	0.042\\
730	0.044\\
731	0.043\\
732	0.043\\
733	0.042\\
734	0.044\\
735	0.042\\
736	0.044\\
737	0.042\\
738	0.046\\
739	0.042\\
740	0.044\\
741	0.043\\
742	0.044\\
743	0.042\\
744	0.043\\
745	0.041\\
746	0.043\\
747	0.041\\
748	0.043\\
749	0.041\\
750	0.037\\
751	0.042\\
752	0.038\\
753	0.041\\
754	0.039\\
755	0.04\\
756	0.039\\
757	0.04\\
758	0.038\\
759	0.04\\
760	0.038\\
761	0.04\\
762	0.039\\
763	0.042\\
764	0.038\\
765	0.04\\
766	0.039\\
767	0.041\\
768	0.039\\
769	0.043\\
770	0.041\\
771	0.042\\
772	0.04\\
773	0.045\\
774	0.039\\
775	0.045\\
776	0.041\\
777	0.043\\
778	0.041\\
779	0.043\\
780	0.04\\
781	0.048\\
782	0.043\\
783	0.038\\
784	0.043\\
785	0.041\\
786	0.045\\
787	0.045\\
788	0.048\\
789	0.042\\
790	0.044\\
791	0.041\\
792	0.046\\
793	0.042\\
794	0.047\\
795	0.044\\
796	0.046\\
797	0.044\\
798	0.045\\
799	0.044\\
800	0.045\\
801	0.042\\
802	0.044\\
803	0.042\\
804	0.045\\
805	0.04\\
806	0.047\\
807	0.041\\
808	0.046\\
809	0.043\\
810	0.048\\
811	0.041\\
812	0.046\\
813	0.043\\
814	0.047\\
815	0.043\\
816	0.045\\
817	0.043\\
818	0.038\\
819	0.043\\
820	0.041\\
821	0.043\\
822	0.041\\
823	0.041\\
824	0.038\\
825	0.045\\
826	0.041\\
827	0.042\\
828	0.039\\
829	0.044\\
830	0.042\\
831	0.047\\
832	0.044\\
833	0.049\\
834	0.041\\
835	0.046\\
836	0.042\\
837	0.044\\
838	0.041\\
839	0.044\\
840	0.043\\
841	0.045\\
842	0.04\\
843	0.046\\
844	0.041\\
845	0.043\\
846	0.04\\
847	0.047\\
848	0.045\\
849	0.047\\
850	0.044\\
851	0.047\\
852	0.043\\
853	0.041\\
854	0.042\\
855	0.039\\
856	0.041\\
857	0.04\\
858	0.042\\
859	0.04\\
860	0.042\\
861	0.039\\
862	0.041\\
863	0.038\\
864	0.041\\
865	0.039\\
866	0.04\\
867	0.039\\
868	0.041\\
869	0.04\\
870	0.043\\
871	0.039\\
872	0.042\\
873	0.039\\
874	0.041\\
875	0.039\\
876	0.04\\
877	0.037\\
878	0.043\\
879	0.038\\
880	0.045\\
881	0.039\\
882	0.042\\
883	0.038\\
884	0.042\\
885	0.038\\
886	0.043\\
887	0.037\\
888	0.035\\
889	0.038\\
890	0.033\\
891	0.036\\
892	0.034\\
893	0.037\\
894	0.033\\
895	0.036\\
896	0.034\\
897	0.038\\
898	0.034\\
899	0.04\\
900	0.035\\
901	0.036\\
902	0.034\\
903	0.039\\
904	0.037\\
905	0.044\\
906	0.041\\
907	0.043\\
908	0.04\\
909	0.043\\
910	0.041\\
911	0.043\\
912	0.04\\
913	0.043\\
914	0.04\\
915	0.045\\
916	0.04\\
917	0.041\\
918	0.038\\
919	0.04\\
920	0.036\\
921	0.04\\
922	0.038\\
923	0.04\\
924	0.038\\
925	0.037\\
926	0.038\\
927	0.037\\
928	0.04\\
929	0.038\\
930	0.04\\
931	0.038\\
932	0.042\\
933	0.039\\
934	0.04\\
935	0.037\\
936	0.04\\
937	0.038\\
938	0.04\\
939	0.038\\
940	0.042\\
941	0.038\\
942	0.041\\
943	0.039\\
944	0.041\\
945	0.038\\
946	0.042\\
947	0.039\\
948	0.042\\
949	0.038\\
950	0.043\\
951	0.041\\
952	0.044\\
953	0.038\\
954	0.042\\
955	0.038\\
956	0.042\\
957	0.039\\
958	0.044\\
959	0.038\\
960	0.042\\
961	0.037\\
962	0.035\\
963	0.04\\
964	0.037\\
965	0.044\\
966	0.04\\
967	0.043\\
968	0.041\\
969	0.041\\
970	0.038\\
971	0.041\\
972	0.039\\
973	0.041\\
974	0.04\\
975	0.043\\
976	0.04\\
977	0.044\\
978	0.037\\
979	0.043\\
980	0.039\\
981	0.042\\
982	0.041\\
983	0.043\\
984	0.042\\
985	0.042\\
986	0.041\\
987	0.043\\
988	0.042\\
989	0.046\\
990	0.045\\
991	0.047\\
992	0.044\\
993	0.046\\
994	0.043\\
995	0.045\\
996	0.041\\
997	0.044\\
998	0.043\\
999	0.046\\
1000	0.043\\
};
\addlegendentry{null-distributed}

\addplot [color=black!50!green]
  table[row sep=crcr]{%
1	0\\
2	0\\
3	0\\
4	0\\
5	0.056\\
6	0.028\\
7	0.018\\
8	0.076\\
9	0.048\\
10	0.028\\
11	0.071\\
12	0.049\\
13	0.096\\
14	0.06\\
15	0.046\\
16	0.094\\
17	0.069\\
18	0.114\\
19	0.08\\
20	0.056\\
21	0.099\\
22	0.068\\
23	0.103\\
24	0.084\\
25	0.065\\
26	0.101\\
27	0.085\\
28	0.11\\
29	0.084\\
30	0.138\\
31	0.106\\
32	0.087\\
33	0.134\\
34	0.109\\
35	0.149\\
36	0.116\\
37	0.17\\
38	0.134\\
39	0.104\\
40	0.15\\
41	0.117\\
42	0.166\\
43	0.138\\
44	0.184\\
45	0.152\\
46	0.128\\
47	0.169\\
48	0.143\\
49	0.185\\
50	0.166\\
51	0.192\\
52	0.167\\
53	0.205\\
54	0.167\\
55	0.144\\
56	0.19\\
57	0.157\\
58	0.193\\
59	0.165\\
60	0.203\\
61	0.172\\
62	0.213\\
63	0.176\\
64	0.157\\
65	0.189\\
66	0.163\\
67	0.209\\
68	0.172\\
69	0.213\\
70	0.181\\
71	0.221\\
72	0.193\\
73	0.16\\
74	0.21\\
75	0.173\\
76	0.219\\
77	0.186\\
78	0.229\\
79	0.195\\
80	0.227\\
81	0.201\\
82	0.234\\
83	0.212\\
84	0.176\\
85	0.215\\
86	0.183\\
87	0.222\\
88	0.194\\
89	0.229\\
90	0.208\\
91	0.237\\
92	0.214\\
93	0.247\\
94	0.215\\
95	0.191\\
96	0.22\\
97	0.198\\
98	0.224\\
99	0.198\\
100	0.234\\
101	0.206\\
102	0.244\\
103	0.214\\
104	0.257\\
105	0.225\\
106	0.261\\
107	0.233\\
108	0.201\\
109	0.24\\
110	0.215\\
111	0.245\\
112	0.226\\
113	0.264\\
114	0.232\\
115	0.277\\
116	0.239\\
117	0.29\\
118	0.256\\
119	0.283\\
120	0.261\\
121	0.233\\
122	0.271\\
123	0.245\\
124	0.283\\
125	0.248\\
126	0.281\\
127	0.249\\
128	0.289\\
129	0.258\\
130	0.295\\
131	0.267\\
132	0.317\\
133	0.282\\
134	0.252\\
135	0.29\\
136	0.267\\
137	0.285\\
138	0.267\\
139	0.295\\
140	0.276\\
141	0.31\\
142	0.283\\
143	0.317\\
144	0.286\\
145	0.321\\
146	0.297\\
147	0.33\\
148	0.301\\
149	0.276\\
150	0.306\\
151	0.289\\
152	0.318\\
153	0.291\\
154	0.323\\
155	0.292\\
156	0.322\\
157	0.296\\
158	0.331\\
159	0.309\\
160	0.348\\
161	0.32\\
162	0.355\\
163	0.335\\
164	0.302\\
165	0.344\\
166	0.313\\
167	0.351\\
168	0.318\\
169	0.353\\
170	0.317\\
171	0.353\\
172	0.326\\
173	0.363\\
174	0.335\\
175	0.369\\
176	0.345\\
177	0.373\\
178	0.346\\
179	0.391\\
180	0.363\\
181	0.331\\
182	0.368\\
183	0.332\\
184	0.37\\
185	0.348\\
186	0.378\\
187	0.351\\
188	0.382\\
189	0.357\\
190	0.394\\
191	0.363\\
192	0.397\\
193	0.373\\
194	0.402\\
195	0.378\\
196	0.357\\
197	0.385\\
198	0.363\\
199	0.387\\
200	0.37\\
201	0.403\\
202	0.378\\
203	0.405\\
204	0.375\\
205	0.415\\
206	0.383\\
207	0.423\\
208	0.39\\
209	0.424\\
210	0.401\\
211	0.431\\
212	0.407\\
213	0.436\\
214	0.413\\
215	0.383\\
216	0.407\\
217	0.388\\
218	0.418\\
219	0.393\\
220	0.416\\
221	0.396\\
222	0.426\\
223	0.404\\
224	0.426\\
225	0.407\\
226	0.436\\
227	0.416\\
228	0.443\\
229	0.428\\
230	0.456\\
231	0.43\\
232	0.408\\
233	0.44\\
234	0.412\\
235	0.444\\
236	0.421\\
237	0.448\\
238	0.417\\
239	0.448\\
240	0.428\\
241	0.457\\
242	0.439\\
243	0.464\\
244	0.442\\
245	0.468\\
246	0.45\\
247	0.473\\
248	0.448\\
249	0.474\\
250	0.456\\
251	0.435\\
252	0.462\\
253	0.442\\
254	0.469\\
255	0.449\\
256	0.471\\
257	0.449\\
258	0.475\\
259	0.452\\
260	0.477\\
261	0.457\\
262	0.487\\
263	0.469\\
264	0.49\\
265	0.469\\
266	0.497\\
267	0.48\\
268	0.504\\
269	0.481\\
270	0.454\\
271	0.483\\
272	0.462\\
273	0.491\\
274	0.465\\
275	0.493\\
276	0.474\\
277	0.501\\
278	0.48\\
279	0.512\\
280	0.484\\
281	0.512\\
282	0.489\\
283	0.522\\
284	0.494\\
285	0.516\\
286	0.496\\
287	0.521\\
288	0.504\\
289	0.526\\
290	0.51\\
291	0.492\\
292	0.519\\
293	0.498\\
294	0.525\\
295	0.502\\
296	0.53\\
297	0.512\\
298	0.534\\
299	0.514\\
300	0.547\\
301	0.52\\
302	0.548\\
303	0.526\\
304	0.553\\
305	0.525\\
306	0.555\\
307	0.525\\
308	0.564\\
309	0.54\\
310	0.566\\
311	0.546\\
312	0.518\\
313	0.553\\
314	0.528\\
315	0.559\\
316	0.528\\
317	0.555\\
318	0.528\\
319	0.557\\
320	0.532\\
321	0.554\\
322	0.529\\
323	0.568\\
324	0.543\\
325	0.57\\
326	0.551\\
327	0.578\\
328	0.549\\
329	0.581\\
330	0.558\\
331	0.592\\
332	0.564\\
333	0.544\\
334	0.581\\
335	0.553\\
336	0.579\\
337	0.564\\
338	0.587\\
339	0.564\\
340	0.595\\
341	0.564\\
342	0.593\\
343	0.569\\
344	0.597\\
345	0.578\\
346	0.599\\
347	0.584\\
348	0.615\\
349	0.589\\
350	0.62\\
351	0.599\\
352	0.619\\
353	0.599\\
354	0.63\\
355	0.616\\
356	0.588\\
357	0.61\\
358	0.596\\
359	0.615\\
360	0.601\\
361	0.619\\
362	0.598\\
363	0.622\\
364	0.604\\
365	0.624\\
366	0.615\\
367	0.629\\
368	0.618\\
369	0.636\\
370	0.62\\
371	0.635\\
372	0.623\\
373	0.645\\
374	0.625\\
375	0.648\\
376	0.633\\
377	0.653\\
378	0.635\\
379	0.622\\
380	0.642\\
381	0.619\\
382	0.64\\
383	0.629\\
384	0.644\\
385	0.622\\
386	0.642\\
387	0.628\\
388	0.648\\
389	0.628\\
390	0.661\\
391	0.636\\
392	0.659\\
393	0.644\\
394	0.658\\
395	0.638\\
396	0.659\\
397	0.642\\
398	0.662\\
399	0.644\\
400	0.672\\
401	0.651\\
402	0.669\\
403	0.65\\
404	0.634\\
405	0.655\\
406	0.634\\
407	0.648\\
408	0.634\\
409	0.658\\
410	0.637\\
411	0.651\\
412	0.637\\
413	0.661\\
414	0.642\\
415	0.66\\
416	0.639\\
417	0.668\\
418	0.645\\
419	0.668\\
420	0.648\\
421	0.665\\
422	0.644\\
423	0.671\\
424	0.652\\
425	0.676\\
426	0.661\\
427	0.682\\
428	0.662\\
429	0.637\\
430	0.66\\
431	0.637\\
432	0.661\\
433	0.645\\
434	0.663\\
435	0.644\\
436	0.66\\
437	0.649\\
438	0.669\\
439	0.655\\
440	0.671\\
441	0.656\\
442	0.68\\
443	0.663\\
444	0.682\\
445	0.668\\
446	0.685\\
447	0.669\\
448	0.686\\
449	0.666\\
450	0.686\\
451	0.673\\
452	0.693\\
453	0.679\\
454	0.659\\
455	0.688\\
456	0.669\\
457	0.692\\
458	0.672\\
459	0.687\\
460	0.673\\
461	0.688\\
462	0.671\\
463	0.688\\
464	0.67\\
465	0.69\\
466	0.673\\
467	0.69\\
468	0.674\\
469	0.697\\
470	0.678\\
471	0.696\\
472	0.685\\
473	0.705\\
474	0.69\\
475	0.708\\
476	0.694\\
477	0.709\\
478	0.703\\
479	0.712\\
480	0.702\\
481	0.688\\
482	0.703\\
483	0.693\\
484	0.708\\
485	0.693\\
486	0.716\\
487	0.697\\
488	0.712\\
489	0.697\\
490	0.713\\
491	0.699\\
492	0.713\\
493	0.703\\
494	0.716\\
495	0.705\\
496	0.72\\
497	0.705\\
498	0.722\\
499	0.708\\
500	0.722\\
501	0.708\\
502	0.728\\
503	0.713\\
504	0.731\\
505	0.713\\
506	0.732\\
507	0.718\\
508	0.707\\
509	0.723\\
510	0.708\\
511	0.726\\
512	0.709\\
513	0.725\\
514	0.709\\
515	0.73\\
516	0.718\\
517	0.734\\
518	0.719\\
519	0.737\\
520	0.722\\
521	0.741\\
522	0.723\\
523	0.738\\
524	0.721\\
525	0.741\\
526	0.723\\
527	0.747\\
528	0.729\\
529	0.749\\
530	0.731\\
531	0.75\\
532	0.737\\
533	0.752\\
534	0.737\\
535	0.726\\
536	0.745\\
537	0.727\\
538	0.743\\
539	0.736\\
540	0.748\\
541	0.731\\
542	0.749\\
543	0.74\\
544	0.757\\
545	0.742\\
546	0.759\\
547	0.743\\
548	0.767\\
549	0.752\\
550	0.77\\
551	0.752\\
552	0.762\\
553	0.755\\
554	0.768\\
555	0.754\\
556	0.768\\
557	0.755\\
558	0.769\\
559	0.758\\
560	0.772\\
561	0.764\\
562	0.775\\
563	0.769\\
564	0.76\\
565	0.769\\
566	0.758\\
567	0.765\\
568	0.759\\
569	0.77\\
570	0.758\\
571	0.773\\
572	0.76\\
573	0.77\\
574	0.757\\
575	0.774\\
576	0.755\\
577	0.77\\
578	0.759\\
579	0.776\\
580	0.756\\
581	0.774\\
582	0.758\\
583	0.778\\
584	0.765\\
585	0.777\\
586	0.764\\
587	0.783\\
588	0.764\\
589	0.779\\
590	0.766\\
591	0.783\\
592	0.767\\
593	0.762\\
594	0.771\\
595	0.76\\
596	0.781\\
597	0.765\\
598	0.785\\
599	0.766\\
600	0.784\\
601	0.77\\
602	0.783\\
603	0.77\\
604	0.78\\
605	0.771\\
606	0.783\\
607	0.771\\
608	0.782\\
609	0.774\\
610	0.787\\
611	0.778\\
612	0.792\\
613	0.78\\
614	0.787\\
615	0.78\\
616	0.789\\
617	0.781\\
618	0.789\\
619	0.778\\
620	0.792\\
621	0.784\\
622	0.771\\
623	0.788\\
624	0.778\\
625	0.796\\
626	0.783\\
627	0.797\\
628	0.784\\
629	0.796\\
630	0.784\\
631	0.793\\
632	0.787\\
633	0.797\\
634	0.791\\
635	0.8\\
636	0.791\\
637	0.813\\
638	0.794\\
639	0.807\\
640	0.792\\
641	0.808\\
642	0.794\\
643	0.818\\
644	0.798\\
645	0.817\\
646	0.803\\
647	0.822\\
648	0.803\\
649	0.822\\
650	0.816\\
651	0.827\\
652	0.818\\
653	0.81\\
654	0.819\\
655	0.808\\
656	0.822\\
657	0.811\\
658	0.827\\
659	0.819\\
660	0.831\\
661	0.821\\
662	0.831\\
663	0.822\\
664	0.832\\
665	0.825\\
666	0.835\\
667	0.827\\
668	0.839\\
669	0.827\\
670	0.842\\
671	0.833\\
672	0.839\\
673	0.833\\
674	0.843\\
675	0.835\\
676	0.844\\
677	0.836\\
678	0.845\\
679	0.839\\
680	0.849\\
681	0.838\\
682	0.847\\
683	0.839\\
684	0.834\\
685	0.843\\
686	0.833\\
687	0.843\\
688	0.832\\
689	0.839\\
690	0.829\\
691	0.84\\
692	0.83\\
693	0.838\\
694	0.832\\
695	0.846\\
696	0.833\\
697	0.845\\
698	0.835\\
699	0.844\\
700	0.834\\
701	0.846\\
702	0.839\\
703	0.844\\
704	0.838\\
705	0.849\\
706	0.841\\
707	0.854\\
708	0.843\\
709	0.853\\
710	0.844\\
711	0.856\\
712	0.85\\
713	0.856\\
714	0.852\\
715	0.855\\
716	0.848\\
717	0.838\\
718	0.849\\
719	0.844\\
720	0.854\\
721	0.845\\
722	0.852\\
723	0.845\\
724	0.852\\
725	0.846\\
726	0.856\\
727	0.846\\
728	0.856\\
729	0.848\\
730	0.856\\
731	0.849\\
732	0.858\\
733	0.847\\
734	0.861\\
735	0.855\\
736	0.864\\
737	0.854\\
738	0.86\\
739	0.853\\
740	0.862\\
741	0.855\\
742	0.863\\
743	0.854\\
744	0.859\\
745	0.856\\
746	0.859\\
747	0.854\\
748	0.859\\
749	0.856\\
750	0.853\\
751	0.855\\
752	0.85\\
753	0.859\\
754	0.851\\
755	0.857\\
756	0.854\\
757	0.859\\
758	0.857\\
759	0.862\\
760	0.858\\
761	0.866\\
762	0.859\\
763	0.866\\
764	0.857\\
765	0.866\\
766	0.862\\
767	0.869\\
768	0.865\\
769	0.871\\
770	0.865\\
771	0.876\\
772	0.864\\
773	0.872\\
774	0.863\\
775	0.872\\
776	0.864\\
777	0.877\\
778	0.866\\
779	0.88\\
780	0.876\\
781	0.88\\
782	0.871\\
783	0.864\\
784	0.872\\
785	0.865\\
786	0.873\\
787	0.868\\
788	0.877\\
789	0.867\\
790	0.877\\
791	0.867\\
792	0.876\\
793	0.87\\
794	0.88\\
795	0.873\\
796	0.882\\
797	0.874\\
798	0.883\\
799	0.873\\
800	0.884\\
801	0.877\\
802	0.89\\
803	0.883\\
804	0.892\\
805	0.885\\
806	0.894\\
807	0.886\\
808	0.898\\
809	0.887\\
810	0.892\\
811	0.883\\
812	0.892\\
813	0.884\\
814	0.893\\
815	0.885\\
816	0.891\\
817	0.883\\
818	0.88\\
819	0.889\\
820	0.88\\
821	0.892\\
822	0.882\\
823	0.892\\
824	0.88\\
825	0.891\\
826	0.887\\
827	0.896\\
828	0.889\\
829	0.895\\
830	0.892\\
831	0.899\\
832	0.893\\
833	0.898\\
834	0.892\\
835	0.899\\
836	0.897\\
837	0.905\\
838	0.898\\
839	0.906\\
840	0.9\\
841	0.906\\
842	0.898\\
843	0.906\\
844	0.899\\
845	0.904\\
846	0.9\\
847	0.906\\
848	0.901\\
849	0.904\\
850	0.899\\
851	0.908\\
852	0.901\\
853	0.899\\
854	0.908\\
855	0.902\\
856	0.909\\
857	0.901\\
858	0.91\\
859	0.902\\
860	0.909\\
861	0.9\\
862	0.909\\
863	0.903\\
864	0.911\\
865	0.904\\
866	0.91\\
867	0.906\\
868	0.912\\
869	0.91\\
870	0.915\\
871	0.91\\
872	0.913\\
873	0.905\\
874	0.913\\
875	0.909\\
876	0.918\\
877	0.911\\
878	0.919\\
879	0.911\\
880	0.918\\
881	0.914\\
882	0.92\\
883	0.914\\
884	0.922\\
885	0.918\\
886	0.926\\
887	0.919\\
888	0.91\\
889	0.916\\
890	0.914\\
891	0.92\\
892	0.913\\
893	0.921\\
894	0.915\\
895	0.924\\
896	0.92\\
897	0.924\\
898	0.919\\
899	0.927\\
900	0.922\\
901	0.933\\
902	0.925\\
903	0.931\\
904	0.922\\
905	0.93\\
906	0.922\\
907	0.929\\
908	0.925\\
909	0.933\\
910	0.925\\
911	0.932\\
912	0.921\\
913	0.929\\
914	0.924\\
915	0.929\\
916	0.923\\
917	0.935\\
918	0.926\\
919	0.935\\
920	0.928\\
921	0.931\\
922	0.929\\
923	0.933\\
924	0.928\\
925	0.923\\
926	0.931\\
927	0.922\\
928	0.929\\
929	0.924\\
930	0.932\\
931	0.926\\
932	0.93\\
933	0.926\\
934	0.932\\
935	0.927\\
936	0.932\\
937	0.928\\
938	0.937\\
939	0.928\\
940	0.934\\
941	0.929\\
942	0.936\\
943	0.932\\
944	0.94\\
945	0.933\\
946	0.935\\
947	0.934\\
948	0.937\\
949	0.929\\
950	0.938\\
951	0.932\\
952	0.941\\
953	0.936\\
954	0.943\\
955	0.939\\
956	0.941\\
957	0.937\\
958	0.943\\
959	0.936\\
960	0.942\\
961	0.935\\
962	0.931\\
963	0.938\\
964	0.935\\
965	0.938\\
966	0.934\\
967	0.937\\
968	0.931\\
969	0.935\\
970	0.93\\
971	0.934\\
972	0.932\\
973	0.937\\
974	0.935\\
975	0.938\\
976	0.935\\
977	0.942\\
978	0.938\\
979	0.942\\
980	0.939\\
981	0.942\\
982	0.939\\
983	0.945\\
984	0.942\\
985	0.948\\
986	0.941\\
987	0.947\\
988	0.942\\
989	0.947\\
990	0.941\\
991	0.945\\
992	0.941\\
993	0.945\\
994	0.941\\
995	0.946\\
996	0.942\\
997	0.943\\
998	0.94\\
999	0.945\\
1000	0.941\\
};
\addlegendentry{other distribution}

\end{axis}
\end{tikzpicture}%
        \caption{Here we see the rejection rate of the null hypothesis ($\model_0$) for two cases. Firstly, for the case when $\model_0$ is true. Secondly, when the data is generated from $\Bernoulli(0.55)$.}
      \end{figure}
      \only<article>{As we see, this method keeps its promise: the null is only rejected 0.05 of the time when it's true. We can also examine how often the null is rejected when it is false... but what should we compare against? Here we are generating data from a $\Bernoulli(0.55)$ model, and we can see the rejection of the null increases with the amount of data. This is called the \alert{power} of the test with respect to the $\Bernoulli(0.55)$ distribution. }
    }
  \end{exercise}
\end{frame}

\begin{frame}
  \begin{alertblock}{Statistical power and false discovery.}
    Beyond not rejecting the null when it's true, we also want:
    \begin{itemize}
    \item High power: Rejecting the null when it is false.
    \item Low false discovery rate: Accepting the null when it is true.
    \end{itemize}
  \end{alertblock}
  \begin{block}{Power}
    The power depends on what hypothesis we use as an alternative.
    \only<article>{This implies that we cannot simply consider a plain null hypothesis test, but must formulate a specific alternative hypothesis. }
  \end{block}

  \begin{block}{False discovery rate}
    False discovery depends on how likely it is \alert{a priori} that the null is false.
    \only<article>{This implies that we need to consider a prior probability for the null hypothesis being true.}
  \end{block}

  \only<article>{Both of these problems suggest that a Bayesian approach might be more suitable. Firstly, it allows us to consider an infinite number of possible alternative models as the alternative hypothesis, through Bayesian model averaging. Secondly, it allows us to specify prior probabilities for each alternative. This is especially important when we consider some effects unlikely.}
\end{frame}

\begin{frame}
  \frametitle{The Bayesian version of the test}
  \begin{example}
    \begin{enumerate}
    \item Set $\util(a_i, \model_j) = \ind{i =
        j}$.
      \only<article>{This choice makes sense if we care equally about
        either type of error.}
    \item Set $\bel(\model_i) = 1/2$. \only<article>{Here we place an
        equal probability in both models.}
    \item $\model_0$: $\Bernoulli(1/2)$. \only<article>{This is the
        same as the null hypothesis test.}
    \item $\model_1$: $\Bernoulli(\theta)$,
      $\theta \sim \Uniform([0,1])$. \only<article>{This is an
        extension of the simple hypothesis test, with an alternative
        hypothesis that says ``the data comes from an arbitrary
        Bernoulli model''.}
    \item Calculate $\bel(\model \mid x)$.
    \item Choose $a_i$, where $i = \argmax_{j} \bel(\model_j \mid x)$.
    \end{enumerate}
    \label{ex:bayesian-compound-hypothesis-test}
  \end{example}
  \begin{block}{Bayesian model averaging for the alternative model $\model_1$}
    \only<article>{In this scenario, $\model_0$ is a simple point model, e.g. corresponding to a $\Bernoulli(1/2)$. However $\model_1$ is a marginal distribution integrated over many models, e.g. a $Beta$ distribution over Bernoulli parameters.}
    \begin{align}
      P_{\model_1}(x) &= \int_\Param B_{\param}(x) \dd \beta(\param) \\
      \bel(\model_0 \mid x) &= \frac{P_{\model_0}(x) \bel(\model_0)}
                              {P_{\model_0}(x) \bel(\model_0) + P_{\model_1}(x) \bel(\model_1)}
    \end{align}
  \end{block}
\end{frame}
\begin{frame}
  \only<1>{
    \begin{figure}[H]
      % This file was created by matlab2tikz.
%
%The latest updates can be retrieved from
%  http://www.mathworks.com/matlabcentral/fileexchange/22022-matlab2tikz-matlab2tikz
%where you can also make suggestions and rate matlab2tikz.
%
\begin{tikzpicture}

\begin{axis}[%
width=0.951\fwidth,
height=\fheight,
at={(0\fwidth,0\fheight)},
scale only axis,
xmin=0,
xmax=1000,
ymin=0.2,
ymax=1,
axis background/.style={fill=white},
title={Posterior probability of null hypothesis},
legend style={legend cell align=left, align=left, legend plot pos=left, draw=black}
]
\addplot [color=blue]
  table[row sep=crcr]{%
1	0.5\\
2	0.516171428571436\\
3	0.535466666666673\\
4	0.550034506556244\\
5	0.569712556285531\\
6	0.586971836258271\\
7	0.598824358885737\\
8	0.609476376611345\\
9	0.619584136921417\\
10	0.630774979249318\\
11	0.640242182919652\\
12	0.648419173011169\\
13	0.653169541894778\\
14	0.659143984303536\\
15	0.664840917220847\\
16	0.670499639436288\\
17	0.671962342755179\\
18	0.678714478519902\\
19	0.683782347691456\\
20	0.687248190475712\\
21	0.688837477145383\\
22	0.690900551070723\\
23	0.69735991286549\\
24	0.703103620839838\\
25	0.704186063586353\\
26	0.707669536414682\\
27	0.710774924703927\\
28	0.715067004181241\\
29	0.716618496734365\\
30	0.717615858862786\\
31	0.720231444712136\\
32	0.724674636554406\\
33	0.727213281219485\\
34	0.730585615142712\\
35	0.732139175685491\\
36	0.735019689330458\\
37	0.738475405914221\\
38	0.741765127525359\\
39	0.741966129897234\\
40	0.744779746925783\\
41	0.745946270610887\\
42	0.748147444605808\\
43	0.751188008410198\\
44	0.753784984163141\\
45	0.755459049494883\\
46	0.75749786205863\\
47	0.758262924610365\\
48	0.759388464494938\\
49	0.761784763316287\\
50	0.762617073387721\\
51	0.764135884768579\\
52	0.766299536548086\\
53	0.767310124311921\\
54	0.768942087113261\\
55	0.770319895848398\\
56	0.770865574883398\\
57	0.773263797018355\\
58	0.773638187590094\\
59	0.773951620564031\\
60	0.775388470771721\\
61	0.775587128400446\\
62	0.775627015377442\\
63	0.777143031451916\\
64	0.777499020876838\\
65	0.777147745193645\\
66	0.778041154145078\\
67	0.779672743123915\\
68	0.780518122654991\\
69	0.782823669515883\\
70	0.783524417695319\\
71	0.783886299780829\\
72	0.784511562740409\\
73	0.785442457544485\\
74	0.787347577032045\\
75	0.788437784497083\\
76	0.789172422400383\\
77	0.790827730902383\\
78	0.791940364539621\\
79	0.792651630353538\\
80	0.79359522932103\\
81	0.794792355702016\\
82	0.795822035300703\\
83	0.796380443703854\\
84	0.797119860005947\\
85	0.799280725113856\\
86	0.800565218819824\\
87	0.801874292091669\\
88	0.801908594886565\\
89	0.802405355162714\\
90	0.802573764592711\\
91	0.802331514265314\\
92	0.804443969548121\\
93	0.804800632828316\\
94	0.805749191970235\\
95	0.805598471173095\\
96	0.806572130546011\\
97	0.805987963692656\\
98	0.807560672465213\\
99	0.807385105837819\\
100	0.807892464282806\\
101	0.808285621281281\\
102	0.808275000297245\\
103	0.808736935642369\\
104	0.809394456043004\\
105	0.810150764132847\\
106	0.811124268245275\\
107	0.812240890882716\\
108	0.812974697313457\\
109	0.812910474169027\\
110	0.813943859736163\\
111	0.814994515894582\\
112	0.816259231998095\\
113	0.816497192638426\\
114	0.815749136346871\\
115	0.816998961846152\\
116	0.817475784223419\\
117	0.818343128946596\\
118	0.819205779952721\\
119	0.819859365050954\\
120	0.820295983955343\\
121	0.821171022716326\\
122	0.822629533299866\\
123	0.822538073022841\\
124	0.823615324119648\\
125	0.823643046108769\\
126	0.823943405213328\\
127	0.824305957706221\\
128	0.824271384192619\\
129	0.824470304025457\\
130	0.824521219083804\\
131	0.824676325782307\\
132	0.824984993779755\\
133	0.825915047596898\\
134	0.827048101248258\\
135	0.827181819134712\\
136	0.827417419664532\\
137	0.828800254784867\\
138	0.829564453383597\\
139	0.82963019703213\\
140	0.830011693043187\\
141	0.830842920650895\\
142	0.831801521365524\\
143	0.832219331880857\\
144	0.83321227988921\\
145	0.832933303019886\\
146	0.833875642853606\\
147	0.834589586628389\\
148	0.834793272472398\\
149	0.835694004152549\\
150	0.835832209826242\\
151	0.836263767157901\\
152	0.836104872872923\\
153	0.836468288861265\\
154	0.837542051505371\\
155	0.83801716449995\\
156	0.838244121662638\\
157	0.837067143204226\\
158	0.838141015046892\\
159	0.839299689237072\\
160	0.839762986479896\\
161	0.840423996362007\\
162	0.841044751191785\\
163	0.841987983419324\\
164	0.841986775868868\\
165	0.842469732958572\\
166	0.842931968511096\\
167	0.842683728166453\\
168	0.842242508086633\\
169	0.842836329859358\\
170	0.842965182225767\\
171	0.843632527971724\\
172	0.844200195293049\\
173	0.844145875511102\\
174	0.843986354861182\\
175	0.844468665293805\\
176	0.844852691008123\\
177	0.844912016451621\\
178	0.845915999741469\\
179	0.846337822256921\\
180	0.847304866733321\\
181	0.847701495450281\\
182	0.848199616495633\\
183	0.84849507644888\\
184	0.848744121213752\\
185	0.848994554574576\\
186	0.848963050647616\\
187	0.849470658907815\\
188	0.849557028239663\\
189	0.849911096509894\\
190	0.850972540038715\\
191	0.851750258657795\\
192	0.852154473530596\\
193	0.852324338939384\\
194	0.852404890333652\\
195	0.852298207747931\\
196	0.852592236804179\\
197	0.852513535157063\\
198	0.852274999120379\\
199	0.852087625197114\\
200	0.852923562030208\\
201	0.853307351183334\\
202	0.853531805687482\\
203	0.853824993255424\\
204	0.853872676343313\\
205	0.854007158610538\\
206	0.853592285060832\\
207	0.854052960699795\\
208	0.854442579426724\\
209	0.854830442614409\\
210	0.854962261778693\\
211	0.855423732861563\\
212	0.855566618559226\\
213	0.855633385837554\\
214	0.856263431750899\\
215	0.856423262343987\\
216	0.856579787076641\\
217	0.856395972368933\\
218	0.856653744830908\\
219	0.856975502609777\\
220	0.857176873670694\\
221	0.857538896211339\\
222	0.857975924740256\\
223	0.857860927086367\\
224	0.85840731758912\\
225	0.858659291643003\\
226	0.858358802198496\\
227	0.859028657193046\\
228	0.859408694401642\\
229	0.859855251089871\\
230	0.860347958540727\\
231	0.861147298267854\\
232	0.861399039540563\\
233	0.860918062185768\\
234	0.860765815065527\\
235	0.861520236441991\\
236	0.862151880103815\\
237	0.862490557502573\\
238	0.862894026074612\\
239	0.863349324302862\\
240	0.863440057612984\\
241	0.863636788653255\\
242	0.863697354934527\\
243	0.86371122612461\\
244	0.864076786304795\\
245	0.863846825090094\\
246	0.863904243663628\\
247	0.864227553647963\\
248	0.864635198375418\\
249	0.864923071287702\\
250	0.864832317749586\\
251	0.865125489468053\\
252	0.86516339973098\\
253	0.86552149694025\\
254	0.864838778229462\\
255	0.864950166190015\\
256	0.864999330428259\\
257	0.864733287341423\\
258	0.864875858902282\\
259	0.864960955040276\\
260	0.865045422513675\\
261	0.864960244391502\\
262	0.864776935657125\\
263	0.864560188052414\\
264	0.863739566282368\\
265	0.863968766641334\\
266	0.864329073806401\\
267	0.865256643029309\\
268	0.865425462633427\\
269	0.865684389100375\\
270	0.866057295847242\\
271	0.866383890980823\\
272	0.866320068545324\\
273	0.86644931114518\\
274	0.866744302413571\\
275	0.866962954128692\\
276	0.867728427192914\\
277	0.868280214521837\\
278	0.868212880063604\\
279	0.868513721221439\\
280	0.869119029334356\\
281	0.869215245299519\\
282	0.869646039709926\\
283	0.869837558135076\\
284	0.870145401407182\\
285	0.870649592734686\\
286	0.870581492095452\\
287	0.87091416881689\\
288	0.870776662107206\\
289	0.870096919064711\\
290	0.870061866109259\\
291	0.870374522605673\\
292	0.870566807378489\\
293	0.871005475362699\\
294	0.871311659666718\\
295	0.870980068997869\\
296	0.870895401837666\\
297	0.870717709124924\\
298	0.870814400657541\\
299	0.870767225023854\\
300	0.870525688517567\\
301	0.87060552001514\\
302	0.871180620052785\\
303	0.871361971221629\\
304	0.871278316393105\\
305	0.871376500577976\\
306	0.87150731268103\\
307	0.872018011877818\\
308	0.872836316274285\\
309	0.872776352571677\\
310	0.872879833181072\\
311	0.872911320647165\\
312	0.873355712861926\\
313	0.873799931953295\\
314	0.873944804961948\\
315	0.87365443381312\\
316	0.873883961765558\\
317	0.874230917271931\\
318	0.873954904906644\\
319	0.874660274030422\\
320	0.87503835804755\\
321	0.8751819589035\\
322	0.874962735558875\\
323	0.874783355950881\\
324	0.874912976123745\\
325	0.87561784570302\\
326	0.876166422723236\\
327	0.876429896006615\\
328	0.875762727345545\\
329	0.87596601892794\\
330	0.876485976598687\\
331	0.876408198531266\\
332	0.87666789984728\\
333	0.876936196667369\\
334	0.877193272088085\\
335	0.877013739064744\\
336	0.877168651018056\\
337	0.877306484276536\\
338	0.877472316378681\\
339	0.877044211211801\\
340	0.877080976971793\\
341	0.877424796753154\\
342	0.87749148928866\\
343	0.877592282028579\\
344	0.878082698856795\\
345	0.878424794797604\\
346	0.878085123435767\\
347	0.878302486723713\\
348	0.878732668946809\\
349	0.878609981193797\\
350	0.878674322240543\\
351	0.878890570968614\\
352	0.878948259707891\\
353	0.879291985786219\\
354	0.879231035429995\\
355	0.87952354367787\\
356	0.879423130997405\\
357	0.879242358217874\\
358	0.87920469803272\\
359	0.879690498526571\\
360	0.879515147370103\\
361	0.879790790640195\\
362	0.880203797349729\\
363	0.88046967295543\\
364	0.880522575693746\\
365	0.880630608985514\\
366	0.880743199214455\\
367	0.880647391514941\\
368	0.880678554346144\\
369	0.880698453382363\\
370	0.880923698974178\\
371	0.880943220514562\\
372	0.880548221914376\\
373	0.880229261111958\\
374	0.880391134863646\\
375	0.880408222870672\\
376	0.88050289246236\\
377	0.880574661561889\\
378	0.880399072713107\\
379	0.880489032478343\\
380	0.880422997948444\\
381	0.880834546948974\\
382	0.881106314087438\\
383	0.880908159082077\\
384	0.880895131248045\\
385	0.881119753151543\\
386	0.880983119684527\\
387	0.880991659848793\\
388	0.880726639021971\\
389	0.880668760345688\\
390	0.880943238358579\\
391	0.880960857425669\\
392	0.881218870679934\\
393	0.88150982875455\\
394	0.881237501242532\\
395	0.881159512876333\\
396	0.881132244907038\\
397	0.880945581753447\\
398	0.881387352119661\\
399	0.881289590628087\\
400	0.881545573965627\\
401	0.882234951256557\\
402	0.881914306185716\\
403	0.881979843403332\\
404	0.881976250038986\\
405	0.882084161582567\\
406	0.882144407181798\\
407	0.882239140889993\\
408	0.882329394250124\\
409	0.882473308898841\\
410	0.882490082983842\\
411	0.882341140903155\\
412	0.882407920504362\\
413	0.882520106187809\\
414	0.882687422701821\\
415	0.882811798759632\\
416	0.882725473034099\\
417	0.88277378678022\\
418	0.88287312958683\\
419	0.88283906890887\\
420	0.883121376205526\\
421	0.882909799504115\\
422	0.883144121811879\\
423	0.883019769729232\\
424	0.883064784493508\\
425	0.88298393243526\\
426	0.882933994872879\\
427	0.883311284825949\\
428	0.883392340694294\\
429	0.883711707742544\\
430	0.883283882521972\\
431	0.883540978548827\\
432	0.883913590303854\\
433	0.883536658061259\\
434	0.883904130324648\\
435	0.883722511614471\\
436	0.883718300164387\\
437	0.883727346572551\\
438	0.883433409930035\\
439	0.88402027113788\\
440	0.884583429496908\\
441	0.884555243337303\\
442	0.88458716605665\\
443	0.884599069944919\\
444	0.88466529783239\\
445	0.88484759426371\\
446	0.884916874412435\\
447	0.884829384495437\\
448	0.885075504252574\\
449	0.885357496813445\\
450	0.885564203277752\\
451	0.885519839790167\\
452	0.885291000221375\\
453	0.885582008505758\\
454	0.885833793363564\\
455	0.886131108435595\\
456	0.886372966845361\\
457	0.886666421814218\\
458	0.886554607694576\\
459	0.887055934270273\\
460	0.88734365484455\\
461	0.887562801307213\\
462	0.887696156667226\\
463	0.888122236978325\\
464	0.888294301138359\\
465	0.888587723316158\\
466	0.88861187218534\\
467	0.888553712568247\\
468	0.888432990317811\\
469	0.888295230355881\\
470	0.888667775914826\\
471	0.888855655463578\\
472	0.889021936167083\\
473	0.888923416820137\\
474	0.888752082570035\\
475	0.888681905054052\\
476	0.888873303341606\\
477	0.889197218971726\\
478	0.889355265867763\\
479	0.889505348410656\\
480	0.889389184376642\\
481	0.889412665098588\\
482	0.889787572368096\\
483	0.889406858686137\\
484	0.889742150481983\\
485	0.889343574466247\\
486	0.889764318191235\\
487	0.889774399970055\\
488	0.89011062736007\\
489	0.890277478857022\\
490	0.890203189579036\\
491	0.890391076995115\\
492	0.890751636439171\\
493	0.890845561223611\\
494	0.891348577084188\\
495	0.891351578962153\\
496	0.891088383431465\\
497	0.89118292168045\\
498	0.891143102998327\\
499	0.891707755523157\\
500	0.891882259284032\\
501	0.892138159883411\\
502	0.892243860933729\\
503	0.89222330626075\\
504	0.892348374316097\\
505	0.892491840784049\\
506	0.893113871115086\\
507	0.893170302969662\\
508	0.892882891307088\\
509	0.892931961195946\\
510	0.893068354698061\\
511	0.892983233584503\\
512	0.892755600851445\\
513	0.892788879416389\\
514	0.893124104596186\\
515	0.893576170667735\\
516	0.893939648672445\\
517	0.894047095791585\\
518	0.89439691218004\\
519	0.894676850204371\\
520	0.894302441123928\\
521	0.894121412716048\\
522	0.893991518411364\\
523	0.893936028404665\\
524	0.894441741020592\\
525	0.894427564487184\\
526	0.894893945475224\\
527	0.894923625604784\\
528	0.89477075262225\\
529	0.894896026740726\\
530	0.894925122873249\\
531	0.894881164459495\\
532	0.895269369464976\\
533	0.895175464364895\\
534	0.895833440820597\\
535	0.895801789707351\\
536	0.895976466181118\\
537	0.896196567799627\\
538	0.896192091430851\\
539	0.896165214813461\\
540	0.896081004365764\\
541	0.895994365357284\\
542	0.89634066206612\\
543	0.897059992540526\\
544	0.896975094509083\\
545	0.89723625473624\\
546	0.897230387194526\\
547	0.897198726577683\\
548	0.89711163335426\\
549	0.897343641173181\\
550	0.897386337302102\\
551	0.897737072129404\\
552	0.897799082772052\\
553	0.897559089119052\\
554	0.897493650729466\\
555	0.897855414239663\\
556	0.897908068454722\\
557	0.897411738605085\\
558	0.897386992567074\\
559	0.897207567603346\\
560	0.89756830645043\\
561	0.897655623735106\\
562	0.897706324153535\\
563	0.897804738278119\\
564	0.897777312901982\\
565	0.897527846351243\\
566	0.897646515613258\\
567	0.897976968666478\\
568	0.897894605275148\\
569	0.898161743605811\\
570	0.898310453164478\\
571	0.89854450432143\\
572	0.898443200010186\\
573	0.898622931098886\\
574	0.898969931778717\\
575	0.899003870516058\\
576	0.899045815479291\\
577	0.899285934081667\\
578	0.899288221618004\\
579	0.899084066371459\\
580	0.899411038744386\\
581	0.899533812887501\\
582	0.899590461075371\\
583	0.899442626245372\\
584	0.89939564529005\\
585	0.899570567466638\\
586	0.899866601396002\\
587	0.899780502827531\\
588	0.899856546211336\\
589	0.899890708757641\\
590	0.899816729016841\\
591	0.900159980047976\\
592	0.900428348044062\\
593	0.900538059574955\\
594	0.900539255518335\\
595	0.900549430651703\\
596	0.900016634660299\\
597	0.899894548334701\\
598	0.899767218123311\\
599	0.899754514412856\\
600	0.899525213951458\\
601	0.899705369976203\\
602	0.899718547822538\\
603	0.90006820228306\\
604	0.900060620010996\\
605	0.900505044870231\\
606	0.900615611857404\\
607	0.90065325054281\\
608	0.900524267940232\\
609	0.900650713438199\\
610	0.900914590906454\\
611	0.90097284579405\\
612	0.900982230726294\\
613	0.90130241019108\\
614	0.901373919359924\\
615	0.901545125878927\\
616	0.901879654751418\\
617	0.901908463846752\\
618	0.901525057683918\\
619	0.901386992680722\\
620	0.901569278137653\\
621	0.901702657800938\\
622	0.901752153596209\\
623	0.901816268178741\\
624	0.901633265934516\\
625	0.901615023517047\\
626	0.901359545581605\\
627	0.901480059913605\\
628	0.901460421616229\\
629	0.901391975792794\\
630	0.90169615386262\\
631	0.90181748330337\\
632	0.901602700278665\\
633	0.901538763916694\\
634	0.90167723381863\\
635	0.901931872600567\\
636	0.901845524791466\\
637	0.901846078586316\\
638	0.902004307351562\\
639	0.90204675721506\\
640	0.901903291968111\\
641	0.902096899611274\\
642	0.902337730926544\\
643	0.902727585669538\\
644	0.90263583505289\\
645	0.902552599253052\\
646	0.902585958584247\\
647	0.902494024721517\\
648	0.902654208216168\\
649	0.902756408763906\\
650	0.902891250679459\\
651	0.903030223074258\\
652	0.902984721835815\\
653	0.903543430153208\\
654	0.903478871198839\\
655	0.903296590307401\\
656	0.903106133870122\\
657	0.903133800120047\\
658	0.903228681410508\\
659	0.903341400120147\\
660	0.903608774397574\\
661	0.903732463598909\\
662	0.904010845361597\\
663	0.904413689665895\\
664	0.904524551561514\\
665	0.904695333107484\\
666	0.904721920752294\\
667	0.904753255048139\\
668	0.904678060054608\\
669	0.904935034966096\\
670	0.905262632596072\\
671	0.905539890276874\\
672	0.905371954398864\\
673	0.905151885447937\\
674	0.905440412764398\\
675	0.905337853560574\\
676	0.905612477824618\\
677	0.90577219280748\\
678	0.905666041386848\\
679	0.905830985758873\\
680	0.906057055862278\\
681	0.906106270451599\\
682	0.906353326742332\\
683	0.906506125659624\\
684	0.906661205609956\\
685	0.906803577771142\\
686	0.907037385122361\\
687	0.907190636531273\\
688	0.907365261356362\\
689	0.907675110757364\\
690	0.907718074427436\\
691	0.907909178530686\\
692	0.908146432956573\\
693	0.908249898054806\\
694	0.908502481007108\\
695	0.908644183143344\\
696	0.908555405392437\\
697	0.90867080373895\\
698	0.908753212218814\\
699	0.909054207153359\\
700	0.90890063828655\\
701	0.908850332097132\\
702	0.908884318322487\\
703	0.908965813636614\\
704	0.908989902523334\\
705	0.908916445511572\\
706	0.909213288791662\\
707	0.909152771257143\\
708	0.909463233447273\\
709	0.909717945063862\\
710	0.909533925494777\\
711	0.909740687949592\\
712	0.909691283826234\\
713	0.910052924051358\\
714	0.910133540614908\\
715	0.910052424252235\\
716	0.910276183437044\\
717	0.910637520794337\\
718	0.910442479082729\\
719	0.910375769182508\\
720	0.910488950048349\\
721	0.910364567836895\\
722	0.910403053413497\\
723	0.910549264858123\\
724	0.910659782243104\\
725	0.910810445617008\\
726	0.910939796456806\\
727	0.9108005182303\\
728	0.910814678413354\\
729	0.910974832176209\\
730	0.910784395862162\\
731	0.910819072718014\\
732	0.910717024238883\\
733	0.910547590052316\\
734	0.91063528646118\\
735	0.910809566932182\\
736	0.910938517525762\\
737	0.911026521270771\\
738	0.911314904489409\\
739	0.911143045386341\\
740	0.911331960223572\\
741	0.911454035089695\\
742	0.911266176052089\\
743	0.91139581238404\\
744	0.911522289896852\\
745	0.911260469044253\\
746	0.911504426419523\\
747	0.911394257971242\\
748	0.911774833290719\\
749	0.911734113122636\\
750	0.911892992166305\\
751	0.911785613811722\\
752	0.912147382161355\\
753	0.912421959587149\\
754	0.912135494456116\\
755	0.912045474757213\\
756	0.912013147325606\\
757	0.91201289079607\\
758	0.912172499289696\\
759	0.91226424050642\\
760	0.912603219493863\\
761	0.912577268755883\\
762	0.912566436417558\\
763	0.912405045472025\\
764	0.91244521906295\\
765	0.91224428051742\\
766	0.912317599325899\\
767	0.912394473561366\\
768	0.912449875402141\\
769	0.91251738711851\\
770	0.912720331002038\\
771	0.912816926194681\\
772	0.91268163365\\
773	0.912789379845302\\
774	0.912779606509189\\
775	0.912925151775764\\
776	0.912878343231236\\
777	0.912676505078622\\
778	0.912793878821591\\
779	0.912855625963328\\
780	0.912684396788898\\
781	0.912572745327558\\
782	0.912765566312514\\
783	0.912928077317406\\
784	0.913013445281674\\
785	0.913203930210435\\
786	0.91339463437878\\
787	0.913534601155355\\
788	0.913728490894168\\
789	0.913826813451232\\
790	0.913940845887192\\
791	0.914112514612025\\
792	0.914183018635227\\
793	0.914064906707441\\
794	0.9141865972506\\
795	0.914411700184427\\
796	0.914304250349474\\
797	0.914323813961639\\
798	0.9145323343398\\
799	0.914614356337323\\
800	0.914892845137304\\
801	0.914917715849067\\
802	0.914873428867972\\
803	0.914871221289352\\
804	0.91478701187856\\
805	0.914932852721562\\
806	0.915038050020159\\
807	0.915141034174544\\
808	0.915149682254174\\
809	0.915146686333699\\
810	0.915171543857016\\
811	0.915375294675781\\
812	0.915396054745457\\
813	0.91560605091877\\
814	0.915500894213795\\
815	0.915301028797034\\
816	0.915404230871579\\
817	0.915250753476283\\
818	0.915192533000144\\
819	0.915028605301418\\
820	0.915251899105413\\
821	0.915230083426673\\
822	0.915307223504849\\
823	0.915158912232802\\
824	0.915082362384047\\
825	0.914998942568762\\
826	0.914792588909826\\
827	0.914777966824885\\
828	0.914955752134602\\
829	0.914931380331426\\
830	0.914941523714118\\
831	0.915030513918394\\
832	0.915309370757999\\
833	0.915299240997692\\
834	0.915153871646524\\
835	0.91501364621945\\
836	0.915050306300341\\
837	0.915089913312647\\
838	0.915161746313424\\
839	0.915058542401726\\
840	0.915052065807686\\
841	0.915374015528099\\
842	0.915376164163171\\
843	0.915504051776971\\
844	0.915586393585018\\
845	0.915863827606719\\
846	0.915849367708674\\
847	0.916051875243217\\
848	0.916008675373021\\
849	0.915718616767925\\
850	0.915547235332371\\
851	0.915665074994086\\
852	0.915722046390945\\
853	0.915835066307326\\
854	0.915636200167381\\
855	0.915614775720576\\
856	0.915718284082082\\
857	0.915477493100832\\
858	0.915440098859882\\
859	0.915455571781583\\
860	0.915557700990874\\
861	0.915569383850482\\
862	0.915604011830024\\
863	0.915590048287865\\
864	0.915446280148472\\
865	0.915239919566922\\
866	0.915184808937877\\
867	0.915399733351274\\
868	0.915113059443874\\
869	0.914954142556317\\
870	0.915153228870243\\
871	0.915531041169181\\
872	0.915545049441126\\
873	0.915597738969984\\
874	0.91571318268768\\
875	0.915718069584487\\
876	0.915839389012341\\
877	0.915744046781744\\
878	0.915797778167255\\
879	0.915592756977034\\
880	0.915696940159904\\
881	0.915753918555924\\
882	0.915998261352823\\
883	0.916148227592552\\
884	0.916153298395849\\
885	0.915977018575578\\
886	0.915859983118674\\
887	0.915872994293752\\
888	0.91603160021596\\
889	0.916082434040739\\
890	0.916410567600981\\
891	0.91675474475424\\
892	0.916715635404787\\
893	0.916935011179733\\
894	0.916958215901519\\
895	0.917033822739697\\
896	0.917289471122671\\
897	0.917342316664964\\
898	0.91749498727209\\
899	0.917476490252223\\
900	0.917387258313442\\
901	0.917240045962969\\
902	0.917167170498814\\
903	0.917370572879384\\
904	0.917128143441557\\
905	0.917327705211346\\
906	0.917404246994261\\
907	0.917776280484053\\
908	0.917666690014572\\
909	0.917724142918911\\
910	0.917896766878741\\
911	0.918190260775574\\
912	0.918345025875835\\
913	0.918307070959321\\
914	0.918252267151595\\
915	0.918229283648111\\
916	0.918084084070488\\
917	0.917953507532265\\
918	0.917995594016518\\
919	0.918213836293586\\
920	0.918329273298921\\
921	0.918498491006878\\
922	0.918346522966441\\
923	0.918393600401156\\
924	0.918186949919011\\
925	0.918142069026667\\
926	0.918303194668836\\
927	0.918224706638235\\
928	0.918214584132132\\
929	0.918151615369326\\
930	0.918138837483877\\
931	0.918230378720453\\
932	0.918250725554561\\
933	0.918343654107917\\
934	0.918397568960151\\
935	0.918560901136775\\
936	0.918356778801985\\
937	0.918231823715435\\
938	0.918404016160631\\
939	0.91821047475024\\
940	0.918387736812185\\
941	0.918721315134624\\
942	0.918462400921876\\
943	0.918492379632599\\
944	0.918425475517422\\
945	0.91817134111801\\
946	0.918353156631883\\
947	0.918017354191054\\
948	0.918358984780302\\
949	0.918391183806349\\
950	0.91854412811842\\
951	0.918623006639771\\
952	0.918351395754892\\
953	0.918333528107821\\
954	0.918536684929103\\
955	0.918709362471464\\
956	0.918865682882781\\
957	0.919064130707737\\
958	0.919018794143734\\
959	0.919240721622824\\
960	0.919037324002579\\
961	0.918970269014811\\
962	0.919275605580243\\
963	0.919255337551036\\
964	0.91927417472909\\
965	0.919356998558558\\
966	0.919342468385738\\
967	0.919229415685191\\
968	0.91933646627594\\
969	0.919227410557453\\
970	0.919327829393836\\
971	0.9191483482392\\
972	0.919243856861833\\
973	0.919243803980787\\
974	0.919382681814874\\
975	0.919463724234132\\
976	0.919365193792976\\
977	0.919444928243144\\
978	0.919293680699418\\
979	0.919175069465694\\
980	0.919521829113373\\
981	0.919474246449753\\
982	0.919324431753175\\
983	0.919121775845435\\
984	0.919028425623168\\
985	0.918983356654179\\
986	0.919174688336572\\
987	0.9192995724094\\
988	0.919223453883839\\
989	0.918986841810612\\
990	0.918919323788917\\
991	0.918739739417495\\
992	0.918946281365214\\
993	0.918826367569661\\
994	0.918804615287205\\
995	0.919101466300651\\
996	0.91909188398597\\
997	0.919015480970684\\
998	0.918898008250188\\
999	0.919166261896097\\
1000	0.919020948243146\\
};
\addlegendentry{null-distributed}

\addplot [color=black!50!green]
  table[row sep=crcr]{%
1	0.5\\
2	0.513942857142864\\
3	0.525333333333339\\
4	0.546404416839197\\
5	0.563664279914378\\
6	0.578287297033043\\
7	0.587821478767767\\
8	0.599509885691713\\
9	0.607311277875057\\
10	0.614878307065523\\
11	0.61942896553621\\
12	0.626843778190632\\
13	0.635940563737373\\
14	0.640292021507716\\
15	0.648141562151822\\
16	0.655747515631002\\
17	0.660406073985582\\
18	0.666798266220354\\
19	0.670446767953772\\
20	0.673734767871229\\
21	0.6793204579945\\
22	0.682128862850864\\
23	0.686736272372917\\
24	0.686943092239843\\
25	0.68828329956071\\
26	0.692271861055805\\
27	0.696320611676308\\
28	0.700692974172805\\
29	0.702241048583795\\
30	0.704390901621819\\
31	0.706523863841647\\
32	0.70696736362342\\
33	0.706111197928436\\
34	0.707123758501178\\
35	0.709239940580157\\
36	0.709567603462631\\
37	0.711450053485339\\
38	0.713408733475397\\
39	0.714424438401756\\
40	0.71366029968831\\
41	0.714711564367458\\
42	0.714824059687775\\
43	0.71481965934665\\
44	0.714417529419534\\
45	0.718042635318856\\
46	0.719137006489847\\
47	0.72144051791789\\
48	0.722238055619036\\
49	0.722449488791589\\
50	0.722222882167521\\
51	0.722884696657733\\
52	0.722101588844764\\
53	0.723630614840175\\
54	0.724949328818612\\
55	0.724063942804378\\
56	0.722617555119498\\
57	0.722420091782217\\
58	0.72271552832246\\
59	0.724280640603191\\
60	0.724906471805137\\
61	0.725147543810183\\
62	0.724440911052475\\
63	0.725613189429684\\
64	0.726657231259837\\
65	0.725914613948953\\
66	0.725156319520119\\
67	0.725927110676031\\
68	0.725586654208811\\
69	0.724257551395394\\
70	0.724498381756816\\
71	0.725127246404118\\
72	0.725103307534331\\
73	0.724841099641631\\
74	0.725026768578907\\
75	0.725108132311802\\
76	0.723680269154024\\
77	0.722264491573715\\
78	0.722629568809739\\
79	0.723414658935985\\
80	0.724535046316314\\
81	0.724328138139447\\
82	0.724973641471454\\
83	0.72431992597801\\
84	0.724385106811669\\
85	0.72541670844585\\
86	0.724619462205194\\
87	0.724196451328243\\
88	0.723537799597613\\
89	0.724403100560871\\
90	0.722769179538055\\
91	0.723237008801881\\
92	0.722253383470021\\
93	0.722051198423602\\
94	0.722143879480019\\
95	0.721130038083614\\
96	0.720815415015211\\
97	0.71973195757975\\
98	0.719442273245657\\
99	0.718649208126782\\
100	0.718133676247915\\
101	0.717444289109234\\
102	0.71569046974137\\
103	0.715548845045133\\
104	0.716122550901745\\
105	0.715388243335012\\
106	0.713228972608302\\
107	0.713030401336259\\
108	0.714000514185002\\
109	0.713547794350056\\
110	0.713924480259168\\
111	0.713795604194274\\
112	0.712658899294572\\
113	0.712904734269066\\
114	0.712181137142015\\
115	0.712296750783674\\
116	0.711341588889368\\
117	0.712049823276795\\
118	0.71054213969262\\
119	0.70977618628854\\
120	0.709549521311225\\
121	0.710186674593611\\
122	0.708230302699964\\
123	0.708651671853796\\
124	0.708402136774239\\
125	0.707478372134844\\
126	0.708708849468534\\
127	0.707336878502362\\
128	0.706813608861234\\
129	0.707634776301996\\
130	0.708014966388104\\
131	0.707502955062245\\
132	0.706517827966801\\
133	0.706561630968778\\
134	0.705850783564613\\
135	0.706685470379868\\
136	0.706307492306676\\
137	0.705920580067026\\
138	0.705356058351516\\
139	0.704476659834834\\
140	0.70566336312029\\
141	0.705806315015553\\
142	0.704037838727964\\
143	0.703428087915985\\
144	0.703209596131186\\
145	0.703023647856046\\
146	0.703276509468797\\
147	0.70414439320598\\
148	0.702573803496115\\
149	0.701755682528389\\
150	0.700887936792761\\
151	0.701241801854414\\
152	0.700017040874211\\
153	0.69979192650949\\
154	0.698618257824843\\
155	0.697626421172996\\
156	0.696997387201981\\
157	0.697284137212546\\
158	0.695971595197934\\
159	0.695299752928896\\
160	0.69470418061909\\
161	0.693718994863996\\
162	0.69368192324781\\
163	0.692062479623159\\
164	0.690477019525689\\
165	0.690508900136272\\
166	0.689932595606299\\
167	0.689345440204234\\
168	0.68986332910415\\
169	0.688845580141737\\
170	0.688408510063165\\
171	0.688428890799856\\
172	0.687688394356403\\
173	0.686883847637499\\
174	0.68692282187821\\
175	0.687097903805092\\
176	0.685585306981997\\
177	0.684363238371658\\
178	0.683826754095846\\
179	0.682990112751384\\
180	0.68269192706384\\
181	0.681363866718169\\
182	0.680381938914794\\
183	0.680755968936581\\
184	0.679417644167737\\
185	0.678830632012048\\
186	0.679072258493557\\
187	0.678750770652114\\
188	0.677594491182415\\
189	0.678479711177533\\
190	0.678641132653696\\
191	0.678800156007754\\
192	0.678068671148061\\
193	0.677708731738284\\
194	0.676558725817296\\
195	0.676565854165445\\
196	0.675899545030432\\
197	0.67581245061502\\
198	0.675188373677226\\
199	0.674101072196277\\
200	0.673309305857127\\
201	0.673955674876275\\
202	0.67488962384195\\
203	0.674400694765239\\
204	0.672447545044203\\
205	0.672799038490921\\
206	0.6713039151279\\
207	0.670499203347542\\
208	0.670632672541431\\
209	0.670474053478289\\
210	0.670438247703692\\
211	0.669970001363851\\
212	0.669029799860119\\
213	0.668830447835637\\
214	0.668793352307904\\
215	0.668888773027717\\
216	0.667074985104093\\
217	0.666876429776684\\
218	0.66652367577426\\
219	0.666938914246616\\
220	0.666379428032444\\
221	0.66624328064939\\
222	0.66550935463141\\
223	0.66627892265451\\
224	0.6652741228868\\
225	0.664556888994183\\
226	0.664205432738977\\
227	0.663319295027025\\
228	0.664073135077853\\
229	0.664640612971051\\
230	0.662978028990357\\
231	0.662800564260769\\
232	0.662255426550206\\
233	0.663038570210468\\
234	0.662420448346968\\
235	0.661912688420642\\
236	0.660958704051511\\
237	0.660867074676399\\
238	0.659563212704836\\
239	0.659175404841912\\
240	0.660759543823676\\
241	0.660608530568051\\
242	0.660482397934791\\
243	0.66109311951469\\
244	0.661216130263376\\
245	0.659641909231272\\
246	0.659572297768978\\
247	0.660272961584462\\
248	0.659961322083712\\
249	0.659543195574815\\
250	0.658454097908837\\
251	0.657543191327115\\
252	0.657270112766131\\
253	0.656715272155536\\
254	0.656288647829992\\
255	0.655119966953994\\
256	0.653985882924536\\
257	0.65281165220576\\
258	0.652210510963118\\
259	0.651044458507381\\
260	0.649111647992201\\
261	0.647883129063704\\
262	0.647070485018223\\
263	0.645734566454622\\
264	0.646099774715535\\
265	0.645585016228687\\
266	0.645257392477941\\
267	0.644368758639524\\
268	0.644532595326916\\
269	0.643660191382904\\
270	0.6432014590246\\
271	0.642983765162167\\
272	0.642851923481855\\
273	0.644029006016115\\
274	0.643426193041184\\
275	0.642779743930451\\
276	0.642198067303313\\
277	0.642417950699785\\
278	0.641573037470582\\
279	0.640331024473939\\
280	0.638831563163237\\
281	0.638544150201401\\
282	0.637508495475684\\
283	0.636642531739991\\
284	0.636224761063899\\
285	0.636491600474961\\
286	0.635908411267386\\
287	0.635525273250826\\
288	0.634813530656442\\
289	0.635387263885571\\
290	0.634378592520861\\
291	0.633233684902057\\
292	0.632713218638815\\
293	0.63147418081822\\
294	0.630266610201846\\
295	0.630595675989272\\
296	0.630336571019094\\
297	0.62997039595021\\
298	0.630447525645991\\
299	0.631093359009009\\
300	0.629626783264297\\
301	0.629970065908439\\
302	0.62973179171613\\
303	0.629081605574149\\
304	0.629016391875177\\
305	0.628936751098454\\
306	0.629183347262911\\
307	0.629000118187104\\
308	0.628600578695524\\
309	0.628302435932324\\
310	0.628270121517269\\
311	0.628343196651723\\
312	0.628037298701906\\
313	0.626619161397581\\
314	0.625504956570856\\
315	0.625496446904361\\
316	0.624635690930658\\
317	0.624589847681706\\
318	0.623893393657425\\
319	0.623030974647937\\
320	0.623468455139542\\
321	0.622845351561341\\
322	0.621309629280833\\
323	0.62065235749662\\
324	0.618571256376867\\
325	0.617729622066238\\
326	0.616850809332352\\
327	0.614854354613671\\
328	0.614900947789136\\
329	0.613528993383905\\
330	0.612681710581075\\
331	0.61291567186491\\
332	0.613220929679525\\
333	0.612736467241297\\
334	0.613204766873615\\
335	0.612652653771668\\
336	0.613319715430996\\
337	0.612231345470457\\
338	0.612359242182746\\
339	0.612344692747308\\
340	0.612268748222221\\
341	0.610556346548945\\
342	0.60991423582998\\
343	0.60889204318404\\
344	0.608480385914882\\
345	0.60716839291354\\
346	0.606149728914301\\
347	0.606865341006897\\
348	0.607224795166342\\
349	0.605963148899609\\
350	0.605376179013207\\
351	0.604906355728751\\
352	0.604855102289231\\
353	0.604484288657839\\
354	0.604981880242304\\
355	0.603719848780149\\
356	0.603154918934341\\
357	0.602039626853713\\
358	0.600264251193326\\
359	0.600007008263151\\
360	0.599944393839198\\
361	0.598605624407042\\
362	0.597819563111072\\
363	0.597310306788753\\
364	0.596452225278151\\
365	0.595515577512336\\
366	0.595042505484878\\
367	0.595215979419132\\
368	0.594941985032149\\
369	0.594481720458652\\
370	0.59471966186622\\
371	0.594951271319101\\
372	0.594597639232159\\
373	0.594363767657641\\
374	0.593686917919418\\
375	0.593367622066724\\
376	0.592377229724427\\
377	0.590745444682819\\
378	0.590125276026654\\
379	0.590098504570838\\
380	0.589670248463759\\
381	0.58946801747974\\
382	0.590284887560635\\
383	0.588913515515518\\
384	0.588420796215904\\
385	0.587547043355099\\
386	0.587150958387885\\
387	0.586673436503915\\
388	0.585586378622872\\
389	0.585708073083785\\
390	0.585455568979977\\
391	0.584291382294535\\
392	0.583586943829963\\
393	0.582709852702703\\
394	0.581613638237278\\
395	0.581086487399467\\
396	0.58074736239231\\
397	0.581834324171997\\
398	0.581107774186558\\
399	0.581287394714208\\
400	0.580613829557256\\
401	0.579741649101872\\
402	0.579800457214036\\
403	0.578849116582536\\
404	0.578541358081551\\
405	0.578249474888206\\
406	0.57766722487589\\
407	0.57674841014049\\
408	0.576907847388152\\
409	0.57570042627312\\
410	0.574964388326317\\
411	0.574463243093739\\
412	0.574196487873648\\
413	0.574084367872161\\
414	0.573448227769746\\
415	0.572681804071925\\
416	0.571909531045384\\
417	0.570190323975144\\
418	0.569268247417564\\
419	0.568884235218011\\
420	0.568230049883394\\
421	0.566886003531586\\
422	0.565126033302696\\
423	0.564202169465791\\
424	0.564024655882769\\
425	0.563707371017417\\
426	0.563317290696125\\
427	0.562801240120277\\
428	0.561812624099095\\
429	0.561304649360094\\
430	0.560671297250255\\
431	0.560390365010322\\
432	0.560169551578494\\
433	0.559308378602905\\
434	0.558895912359754\\
435	0.55772311670353\\
436	0.557025675807837\\
437	0.556685771521932\\
438	0.556209837848745\\
439	0.555853208704557\\
440	0.556160454546336\\
441	0.55623156088712\\
442	0.555333680807174\\
443	0.554507509549576\\
444	0.553967139411944\\
445	0.554452816388795\\
446	0.553912666679959\\
447	0.553667910899054\\
448	0.553314340307601\\
449	0.552572280408765\\
450	0.551533984354584\\
451	0.551184413365218\\
452	0.550271317470612\\
453	0.549591360554304\\
454	0.548875893029228\\
455	0.548264133710187\\
456	0.548341541344244\\
457	0.546720793188367\\
458	0.546348170979781\\
459	0.545766056992203\\
460	0.545486726424171\\
461	0.545475346055948\\
462	0.544840396939695\\
463	0.543830714812002\\
464	0.542898145690271\\
465	0.541913772563901\\
466	0.541158435086556\\
467	0.54125543021283\\
468	0.540757316190432\\
469	0.541164823447092\\
470	0.540298761270525\\
471	0.539670172715418\\
472	0.538968031801389\\
473	0.538108340557737\\
474	0.539159742378997\\
475	0.539069982165388\\
476	0.538024137331363\\
477	0.537548075703682\\
478	0.537979879159078\\
479	0.536123425315079\\
480	0.534983254042553\\
481	0.533943967161006\\
482	0.533873282375418\\
483	0.53428309574986\\
484	0.533095766230911\\
485	0.532536021707665\\
486	0.531967342318331\\
487	0.531523602268408\\
488	0.530879692912181\\
489	0.530168802826463\\
490	0.530157702442634\\
491	0.529404332177237\\
492	0.528976997578506\\
493	0.528670166681012\\
494	0.528847946537851\\
495	0.528705054880268\\
496	0.527969782000002\\
497	0.527075570075546\\
498	0.526535213989747\\
499	0.526104842053608\\
500	0.52629707756311\\
501	0.525184595279436\\
502	0.524559382932094\\
503	0.523571942349571\\
504	0.523739527026638\\
505	0.523859803068343\\
506	0.523524255989232\\
507	0.521997409562456\\
508	0.521866690914211\\
509	0.522287824926174\\
510	0.522161506162563\\
511	0.52141891551891\\
512	0.521364906000159\\
513	0.521564718062396\\
514	0.520792779975214\\
515	0.520748644844857\\
516	0.52112582547267\\
517	0.520349645684189\\
518	0.519886372277643\\
519	0.519022249991769\\
520	0.517765069622732\\
521	0.517406240372728\\
522	0.5165380995334\\
523	0.516448907489477\\
524	0.516268841651992\\
525	0.515266469371506\\
526	0.514909432515885\\
527	0.514288159309872\\
528	0.513441795765246\\
529	0.513168975729034\\
530	0.511986962294362\\
531	0.512508452027206\\
532	0.511529131776803\\
533	0.511000079270791\\
534	0.510390642313395\\
535	0.510009892919608\\
536	0.510152018377698\\
537	0.509318077115718\\
538	0.508573824013557\\
539	0.508040445811428\\
540	0.507785456027295\\
541	0.507206276627205\\
542	0.506232149373118\\
543	0.506120339138234\\
544	0.505359159700101\\
545	0.505383615878573\\
546	0.505397291957952\\
547	0.504881781832182\\
548	0.50449474285599\\
549	0.50406536000267\\
550	0.503134502065654\\
551	0.503219765819139\\
552	0.503320511951271\\
553	0.502666245136646\\
554	0.502272086171009\\
555	0.501744043471355\\
556	0.501172757469932\\
557	0.500635310752108\\
558	0.500837069532719\\
559	0.500585734490029\\
560	0.499716950985859\\
561	0.498705461718568\\
562	0.498421924657258\\
563	0.497930136208206\\
564	0.498193276253146\\
565	0.498729989967714\\
566	0.497620567416711\\
567	0.496544328639115\\
568	0.495253876919042\\
569	0.49432936491642\\
570	0.493441893854905\\
571	0.493352643283484\\
572	0.492967114141757\\
573	0.492252082411779\\
574	0.492017107220033\\
575	0.490530761040674\\
576	0.489904679385188\\
577	0.488919715172651\\
578	0.489097908389129\\
579	0.488118892387526\\
580	0.488230497411074\\
581	0.487251442059809\\
582	0.48760937892687\\
583	0.487254093792511\\
584	0.487234327787051\\
585	0.486915375394504\\
586	0.486266425682014\\
587	0.486228839962641\\
588	0.486005797929462\\
589	0.48617412296461\\
590	0.485854995696878\\
591	0.485314526841098\\
592	0.484830007116671\\
593	0.484815872727909\\
594	0.484494344498256\\
595	0.483958734404359\\
596	0.483740673322484\\
597	0.482942840237643\\
598	0.482147668670813\\
599	0.481092095518117\\
600	0.480558126452035\\
601	0.480362781350784\\
602	0.479995771567739\\
603	0.479623131180517\\
604	0.479561867889894\\
605	0.478895171689077\\
606	0.47816595386048\\
607	0.477858709090963\\
608	0.477747455348367\\
609	0.476124624888711\\
610	0.475653491919836\\
611	0.475905162309375\\
612	0.475328473447875\\
613	0.474544570920017\\
614	0.474339276752698\\
615	0.474167842749532\\
616	0.472980202759118\\
617	0.472723251010846\\
618	0.473250883908676\\
619	0.47295644862552\\
620	0.471759117231933\\
621	0.471760750330549\\
622	0.471717736971781\\
623	0.471566070113202\\
624	0.470467161050637\\
625	0.469662495539681\\
626	0.469146070305535\\
627	0.468524541945752\\
628	0.467776914022503\\
629	0.467067720122715\\
630	0.46639937973305\\
631	0.465466439590086\\
632	0.46503970529123\\
633	0.464626908086484\\
634	0.464205494655554\\
635	0.46355629508944\\
636	0.46279980560886\\
637	0.463524319449003\\
638	0.462219027408265\\
639	0.462348342368444\\
640	0.461951174442057\\
641	0.461212370254154\\
642	0.461735329135965\\
643	0.460841255151134\\
644	0.459924005869366\\
645	0.45885265428512\\
646	0.458740858164173\\
647	0.458443023846652\\
648	0.457059535100114\\
649	0.456432349402005\\
650	0.455374433047414\\
651	0.453926871009875\\
652	0.453736623313728\\
653	0.452756035805489\\
654	0.451740684768896\\
655	0.450565794300399\\
656	0.450341551892142\\
657	0.449894816450538\\
658	0.449419587473518\\
659	0.449187607142915\\
660	0.448856145525668\\
661	0.4483140715626\\
662	0.447970474786024\\
663	0.447073744977013\\
664	0.447188230127553\\
665	0.446281596076747\\
666	0.445930191217683\\
667	0.445955094403397\\
668	0.445954344028826\\
669	0.445390495493107\\
670	0.444636134206525\\
671	0.444167877869218\\
672	0.443281469250553\\
673	0.443412587078003\\
674	0.442880068559997\\
675	0.441528810241933\\
676	0.44106629276735\\
677	0.440257301095231\\
678	0.43927832194972\\
679	0.439209846226354\\
680	0.438075032710221\\
681	0.437359292110485\\
682	0.436910929479711\\
683	0.435878646779217\\
684	0.435610914186566\\
685	0.43557311724811\\
686	0.434958493540607\\
687	0.434190388999896\\
688	0.43360097681899\\
689	0.433105832314922\\
690	0.432951976575234\\
691	0.432709495238119\\
692	0.433093622069183\\
693	0.432928662384246\\
694	0.431934380057485\\
695	0.43220602581338\\
696	0.431210866423807\\
697	0.430969484874678\\
698	0.430430402535359\\
699	0.430257720935252\\
700	0.430658232388091\\
701	0.430115084277768\\
702	0.429367966023162\\
703	0.428458824550885\\
704	0.427938236937655\\
705	0.427298091750692\\
706	0.426236761920677\\
707	0.425616957385159\\
708	0.424834028759915\\
709	0.424604752085485\\
710	0.423208716356204\\
711	0.422686095091078\\
712	0.422317227803299\\
713	0.421797157455279\\
714	0.421292683762069\\
715	0.420338767276787\\
716	0.419898459445156\\
717	0.419285933690607\\
718	0.419182245353556\\
719	0.418176293365776\\
720	0.417805063032399\\
721	0.416784471257724\\
722	0.416714915978007\\
723	0.416351459321945\\
724	0.41593865978035\\
725	0.415435391109323\\
726	0.414146823695219\\
727	0.414143590672261\\
728	0.412933294170027\\
729	0.41169446800185\\
730	0.41121080199639\\
731	0.410577483291192\\
732	0.409286917599678\\
733	0.408830053262051\\
734	0.408588519293258\\
735	0.408057237871212\\
736	0.407766854818661\\
737	0.407318750349415\\
738	0.406193540244973\\
739	0.406579926491861\\
740	0.406124698861418\\
741	0.405770002576774\\
742	0.405015138195054\\
743	0.40445468977384\\
744	0.403735074316275\\
745	0.403622107932461\\
746	0.402776662005238\\
747	0.402332506567219\\
748	0.401958285383694\\
749	0.401740598608534\\
750	0.401438258337991\\
751	0.401129222431801\\
752	0.401427052543191\\
753	0.400083743490874\\
754	0.400049064399837\\
755	0.399864116900748\\
756	0.399472418648439\\
757	0.399088233078191\\
758	0.399030644869663\\
759	0.399042057848833\\
760	0.398975531058884\\
761	0.398783943681213\\
762	0.39809465487475\\
763	0.398657040179093\\
764	0.397542128845531\\
765	0.396720721214668\\
766	0.396416016070012\\
767	0.396612638585283\\
768	0.39574616021499\\
769	0.395152662873473\\
770	0.39536374504269\\
771	0.394857236961354\\
772	0.394529028684758\\
773	0.394227475687067\\
774	0.393110177897741\\
775	0.3929599569242\\
776	0.393692345662958\\
777	0.393390358960062\\
778	0.392540239165285\\
779	0.391925004603758\\
780	0.391261827970482\\
781	0.390594826062816\\
782	0.390120782503143\\
783	0.389178348319198\\
784	0.389366380730813\\
785	0.388787040303088\\
786	0.389376586476452\\
787	0.389185708900994\\
788	0.388695967115441\\
789	0.387637259135205\\
790	0.387266961798338\\
791	0.38645391892321\\
792	0.38614522337462\\
793	0.385325623932817\\
794	0.384934618091758\\
795	0.384832788640185\\
796	0.384412506931093\\
797	0.384045425293432\\
798	0.383519151428068\\
799	0.38351300226265\\
800	0.383539409149739\\
801	0.382982652284074\\
802	0.382110344927967\\
803	0.38118721830257\\
804	0.381063112954521\\
805	0.380362092183035\\
806	0.380345675686216\\
807	0.379891679543367\\
808	0.379651318471131\\
809	0.380257833981168\\
810	0.379276309060895\\
811	0.378919227042354\\
812	0.378377974989111\\
813	0.377880431050613\\
814	0.377160205905844\\
815	0.376526302945332\\
816	0.375419456780959\\
817	0.374414548639019\\
818	0.374380083510008\\
819	0.37337820226657\\
820	0.372751309265919\\
821	0.371326229355152\\
822	0.371737097003268\\
823	0.371776987657233\\
824	0.371603722429232\\
825	0.371465717364563\\
826	0.371586998446385\\
827	0.370741416637112\\
828	0.370315157493872\\
829	0.370359501590705\\
830	0.370084441122208\\
831	0.369451965731414\\
832	0.369016792655531\\
833	0.36739194707352\\
834	0.366917783119777\\
835	0.366285315657217\\
836	0.365730781661895\\
837	0.364677780978921\\
838	0.363877359268758\\
839	0.362933599252789\\
840	0.361969408200796\\
841	0.362411883328394\\
842	0.36220726682821\\
843	0.362657310463425\\
844	0.362599034903516\\
845	0.361763501108188\\
846	0.361093757201119\\
847	0.36086402056596\\
848	0.360409783313308\\
849	0.360511969707803\\
850	0.360453258659117\\
851	0.360147265798823\\
852	0.359178977081687\\
853	0.359238859157726\\
854	0.358281772122768\\
855	0.357644151766169\\
856	0.356582375308447\\
857	0.356014021119466\\
858	0.356240255721197\\
859	0.355088833970236\\
860	0.354423386693532\\
861	0.353801300184312\\
862	0.353260831840737\\
863	0.353820645571033\\
864	0.35298734213429\\
865	0.353509485712804\\
866	0.353396554299141\\
867	0.35301002156271\\
868	0.353185537090005\\
869	0.352187680295526\\
870	0.351387605666739\\
871	0.351327261889225\\
872	0.350903650846441\\
873	0.350850206817812\\
874	0.350232300555305\\
875	0.349673052173512\\
876	0.349148723398263\\
877	0.348295148526418\\
878	0.347934407329278\\
879	0.347426715657098\\
880	0.346996977730943\\
881	0.347183411580336\\
882	0.347287560791499\\
883	0.346560550744635\\
884	0.346395778641461\\
885	0.345986006794965\\
886	0.344524146416242\\
887	0.344540464362926\\
888	0.344139969803329\\
889	0.34429938902648\\
890	0.343840436714344\\
891	0.343806054960867\\
892	0.342937143127471\\
893	0.342256678388858\\
894	0.341979518955133\\
895	0.34186793200406\\
896	0.340597875932381\\
897	0.339899197551053\\
898	0.339640257521332\\
899	0.339787449067995\\
900	0.33892037499241\\
901	0.33820102911678\\
902	0.338353782814088\\
903	0.338509263067654\\
904	0.33766999713155\\
905	0.337271766595697\\
906	0.337066995397932\\
907	0.336935300605187\\
908	0.336129059252331\\
909	0.335935021948242\\
910	0.336139845577104\\
911	0.336192534640188\\
912	0.3356344161272\\
913	0.334770201526099\\
914	0.334367642604582\\
915	0.334063272496505\\
916	0.333375699567559\\
917	0.333280823592091\\
918	0.332107277799141\\
919	0.331324585121808\\
920	0.330642604026613\\
921	0.32986729325812\\
922	0.329312591387683\\
923	0.328935334752623\\
924	0.328521372680921\\
925	0.327253668894209\\
926	0.327298336079201\\
927	0.327394369979519\\
928	0.328133259547277\\
929	0.327666850774273\\
930	0.327763881807135\\
931	0.326605905056291\\
932	0.326525047244072\\
933	0.325962799922214\\
934	0.325627504267993\\
935	0.325124245839169\\
936	0.324184925494078\\
937	0.323912302158319\\
938	0.322970197941211\\
939	0.323108700419947\\
940	0.322664743602024\\
941	0.322321088387867\\
942	0.321311461232306\\
943	0.320282895143933\\
944	0.319925272564732\\
945	0.319607464040407\\
946	0.319133800398384\\
947	0.318456078268956\\
948	0.317877762798595\\
949	0.317166555756292\\
950	0.317010439852196\\
951	0.315797675284573\\
952	0.315140726735912\\
953	0.314617463555279\\
954	0.314991483916675\\
955	0.315064379153405\\
956	0.314790090321577\\
957	0.314242901915546\\
958	0.313456242114012\\
959	0.312931614000478\\
960	0.312745495815496\\
961	0.312574337494792\\
962	0.31188632629579\\
963	0.311751719764068\\
964	0.311400998710236\\
965	0.310394203372314\\
966	0.309977390129629\\
967	0.309076193434459\\
968	0.308914686857899\\
969	0.308514455719222\\
970	0.308237834783132\\
971	0.307972620377665\\
972	0.307107396850545\\
973	0.306887022426041\\
974	0.306789507695762\\
975	0.306572140672593\\
976	0.305828966534718\\
977	0.305463629150419\\
978	0.304582296155657\\
979	0.304563262618209\\
980	0.304460937710031\\
981	0.303784445944709\\
982	0.303348105619065\\
983	0.303025268625964\\
984	0.302688001369936\\
985	0.301978625652606\\
986	0.301486250316236\\
987	0.301088305462718\\
988	0.300042384106731\\
989	0.299925935921835\\
990	0.299709277756571\\
991	0.299153642224413\\
992	0.298432747447492\\
993	0.297644233025162\\
994	0.297743976354175\\
995	0.297576028908003\\
996	0.297476543859398\\
997	0.296749727235235\\
998	0.295984061849509\\
999	0.295541081001737\\
1000	0.294666079091381\\
};
\addlegendentry{other-distributed}

\end{axis}
\end{tikzpicture}%
      \caption{Here we see the convergence of the posterior probability.}
    \end{figure}
    \only<article>{As can be seen in the figure above, in both cases, the posterior converges to the correct value, so it can be used to indicate our confidence that the null is true.}
  }
  \only<2>{
    \begin{figure}[H]
      % This file was created by matlab2tikz.
%
%The latest updates can be retrieved from
%  http://www.mathworks.com/matlabcentral/fileexchange/22022-matlab2tikz-matlab2tikz
%where you can also make suggestions and rate matlab2tikz.
%
\begin{tikzpicture}

\begin{axis}[%
width=0.951\fwidth,
height=\fheight,
at={(0\fwidth,0\fheight)},
scale only axis,
xmin=0,
xmax=1000,
ymin=0,
ymax=0.7,
axis background/.style={fill=white},
title={Rejection of null hypothesis for Bernoulli(0.5)},
legend style={legend cell align=left, align=left, legend plot pos=left, draw=black}
]
\addplot [color=blue]
  table[row sep=crcr]{%
1	0\\
2	0\\
3	0\\
4	0\\
5	0.042\\
6	0.022\\
7	0.014\\
8	0.042\\
9	0.027\\
10	0.014\\
11	0.04\\
12	0.025\\
13	0.061\\
14	0.036\\
15	0.018\\
16	0.047\\
17	0.026\\
18	0.055\\
19	0.039\\
20	0.026\\
21	0.044\\
22	0.035\\
23	0.05\\
24	0.035\\
25	0.025\\
26	0.045\\
27	0.034\\
28	0.054\\
29	0.038\\
30	0.065\\
31	0.046\\
32	0.028\\
33	0.052\\
34	0.03\\
35	0.05\\
36	0.031\\
37	0.053\\
38	0.036\\
39	0.026\\
40	0.046\\
41	0.031\\
42	0.053\\
43	0.038\\
44	0.055\\
45	0.039\\
46	0.024\\
47	0.036\\
48	0.026\\
49	0.037\\
50	0.027\\
51	0.039\\
52	0.032\\
53	0.047\\
54	0.038\\
55	0.027\\
56	0.041\\
57	0.033\\
58	0.048\\
59	0.033\\
60	0.045\\
61	0.037\\
62	0.06\\
63	0.04\\
64	0.029\\
65	0.045\\
66	0.036\\
67	0.05\\
68	0.041\\
69	0.053\\
70	0.046\\
71	0.058\\
72	0.044\\
73	0.036\\
74	0.045\\
75	0.039\\
76	0.046\\
77	0.04\\
78	0.054\\
79	0.045\\
80	0.054\\
81	0.046\\
82	0.053\\
83	0.043\\
84	0.029\\
85	0.041\\
86	0.03\\
87	0.042\\
88	0.035\\
89	0.046\\
90	0.033\\
91	0.042\\
92	0.035\\
93	0.051\\
94	0.044\\
95	0.04\\
96	0.05\\
97	0.036\\
98	0.047\\
99	0.036\\
100	0.047\\
101	0.032\\
102	0.047\\
103	0.038\\
104	0.049\\
105	0.041\\
106	0.053\\
107	0.043\\
108	0.036\\
109	0.047\\
110	0.042\\
111	0.048\\
112	0.04\\
113	0.045\\
114	0.038\\
115	0.047\\
116	0.043\\
117	0.049\\
118	0.035\\
119	0.048\\
120	0.04\\
121	0.026\\
122	0.037\\
123	0.026\\
124	0.035\\
125	0.027\\
126	0.043\\
127	0.031\\
128	0.038\\
129	0.035\\
130	0.045\\
131	0.039\\
132	0.051\\
133	0.046\\
134	0.038\\
135	0.044\\
136	0.037\\
137	0.044\\
138	0.033\\
139	0.048\\
140	0.038\\
141	0.046\\
142	0.037\\
143	0.047\\
144	0.041\\
145	0.046\\
146	0.045\\
147	0.052\\
148	0.047\\
149	0.041\\
150	0.043\\
151	0.037\\
152	0.05\\
153	0.046\\
154	0.05\\
155	0.045\\
156	0.05\\
157	0.044\\
158	0.05\\
159	0.047\\
160	0.052\\
161	0.045\\
162	0.05\\
163	0.046\\
164	0.037\\
165	0.046\\
166	0.037\\
167	0.05\\
168	0.041\\
169	0.047\\
170	0.041\\
171	0.053\\
172	0.048\\
173	0.051\\
174	0.046\\
175	0.049\\
176	0.046\\
177	0.054\\
178	0.047\\
179	0.053\\
180	0.044\\
181	0.036\\
182	0.043\\
183	0.038\\
184	0.041\\
185	0.04\\
186	0.046\\
187	0.041\\
188	0.048\\
189	0.045\\
190	0.05\\
191	0.044\\
192	0.048\\
193	0.043\\
194	0.052\\
195	0.043\\
196	0.036\\
197	0.045\\
198	0.039\\
199	0.047\\
200	0.041\\
201	0.047\\
202	0.04\\
203	0.048\\
204	0.041\\
205	0.05\\
206	0.042\\
207	0.047\\
208	0.04\\
209	0.046\\
210	0.039\\
211	0.044\\
212	0.039\\
213	0.047\\
214	0.041\\
215	0.034\\
216	0.043\\
217	0.037\\
218	0.044\\
219	0.035\\
220	0.042\\
221	0.04\\
222	0.045\\
223	0.038\\
224	0.049\\
225	0.04\\
226	0.047\\
227	0.041\\
228	0.051\\
229	0.041\\
230	0.052\\
231	0.044\\
232	0.035\\
233	0.04\\
234	0.036\\
235	0.039\\
236	0.034\\
237	0.041\\
238	0.037\\
239	0.041\\
240	0.035\\
241	0.04\\
242	0.034\\
243	0.043\\
244	0.035\\
245	0.04\\
246	0.037\\
247	0.041\\
248	0.035\\
249	0.041\\
250	0.033\\
251	0.033\\
252	0.038\\
253	0.036\\
254	0.039\\
255	0.033\\
256	0.038\\
257	0.036\\
258	0.041\\
259	0.035\\
260	0.041\\
261	0.031\\
262	0.04\\
263	0.033\\
264	0.042\\
265	0.034\\
266	0.044\\
267	0.038\\
268	0.043\\
269	0.033\\
270	0.028\\
271	0.036\\
272	0.028\\
273	0.033\\
274	0.027\\
275	0.033\\
276	0.028\\
277	0.033\\
278	0.028\\
279	0.038\\
280	0.03\\
281	0.039\\
282	0.03\\
283	0.038\\
284	0.031\\
285	0.038\\
286	0.035\\
287	0.037\\
288	0.032\\
289	0.038\\
290	0.031\\
291	0.028\\
292	0.033\\
293	0.032\\
294	0.033\\
295	0.031\\
296	0.035\\
297	0.03\\
298	0.038\\
299	0.036\\
300	0.038\\
301	0.036\\
302	0.039\\
303	0.037\\
304	0.041\\
305	0.04\\
306	0.044\\
307	0.04\\
308	0.044\\
309	0.041\\
310	0.042\\
311	0.038\\
312	0.033\\
313	0.038\\
314	0.033\\
315	0.039\\
316	0.036\\
317	0.04\\
318	0.037\\
319	0.042\\
320	0.037\\
321	0.041\\
322	0.036\\
323	0.041\\
324	0.037\\
325	0.042\\
326	0.036\\
327	0.039\\
328	0.035\\
329	0.04\\
330	0.037\\
331	0.045\\
332	0.037\\
333	0.033\\
334	0.04\\
335	0.036\\
336	0.041\\
337	0.039\\
338	0.044\\
339	0.042\\
340	0.043\\
341	0.039\\
342	0.043\\
343	0.04\\
344	0.043\\
345	0.038\\
346	0.041\\
347	0.039\\
348	0.044\\
349	0.042\\
350	0.047\\
351	0.042\\
352	0.046\\
353	0.042\\
354	0.049\\
355	0.04\\
356	0.035\\
357	0.041\\
358	0.037\\
359	0.039\\
360	0.038\\
361	0.039\\
362	0.035\\
363	0.04\\
364	0.034\\
365	0.039\\
366	0.036\\
367	0.039\\
368	0.035\\
369	0.042\\
370	0.032\\
371	0.04\\
372	0.033\\
373	0.039\\
374	0.036\\
375	0.037\\
376	0.031\\
377	0.035\\
378	0.034\\
379	0.029\\
380	0.032\\
381	0.028\\
382	0.033\\
383	0.028\\
384	0.033\\
385	0.03\\
386	0.035\\
387	0.026\\
388	0.033\\
389	0.028\\
390	0.03\\
391	0.029\\
392	0.032\\
393	0.031\\
394	0.034\\
395	0.033\\
396	0.038\\
397	0.031\\
398	0.033\\
399	0.03\\
400	0.035\\
401	0.028\\
402	0.037\\
403	0.033\\
404	0.031\\
405	0.032\\
406	0.027\\
407	0.032\\
408	0.03\\
409	0.034\\
410	0.03\\
411	0.035\\
412	0.029\\
413	0.035\\
414	0.03\\
415	0.037\\
416	0.035\\
417	0.038\\
418	0.032\\
419	0.038\\
420	0.033\\
421	0.035\\
422	0.034\\
423	0.038\\
424	0.036\\
425	0.038\\
426	0.033\\
427	0.036\\
428	0.033\\
429	0.029\\
430	0.032\\
431	0.032\\
432	0.034\\
433	0.03\\
434	0.032\\
435	0.028\\
436	0.034\\
437	0.031\\
438	0.037\\
439	0.034\\
440	0.039\\
441	0.033\\
442	0.038\\
443	0.032\\
444	0.037\\
445	0.034\\
446	0.04\\
447	0.034\\
448	0.04\\
449	0.037\\
450	0.04\\
451	0.035\\
452	0.038\\
453	0.036\\
454	0.031\\
455	0.034\\
456	0.028\\
457	0.031\\
458	0.031\\
459	0.032\\
460	0.031\\
461	0.034\\
462	0.028\\
463	0.031\\
464	0.029\\
465	0.033\\
466	0.029\\
467	0.037\\
468	0.033\\
469	0.038\\
470	0.036\\
471	0.038\\
472	0.037\\
473	0.038\\
474	0.036\\
475	0.037\\
476	0.037\\
477	0.039\\
478	0.036\\
479	0.038\\
480	0.035\\
481	0.032\\
482	0.039\\
483	0.033\\
484	0.039\\
485	0.036\\
486	0.043\\
487	0.036\\
488	0.041\\
489	0.039\\
490	0.042\\
491	0.038\\
492	0.04\\
493	0.036\\
494	0.04\\
495	0.036\\
496	0.041\\
497	0.038\\
498	0.041\\
499	0.036\\
500	0.043\\
501	0.038\\
502	0.044\\
503	0.04\\
504	0.045\\
505	0.042\\
506	0.044\\
507	0.042\\
508	0.038\\
509	0.043\\
510	0.038\\
511	0.038\\
512	0.035\\
513	0.037\\
514	0.034\\
515	0.038\\
516	0.035\\
517	0.039\\
518	0.037\\
519	0.039\\
520	0.036\\
521	0.041\\
522	0.035\\
523	0.039\\
524	0.036\\
525	0.04\\
526	0.038\\
527	0.04\\
528	0.037\\
529	0.042\\
530	0.038\\
531	0.043\\
532	0.039\\
533	0.042\\
534	0.04\\
535	0.038\\
536	0.04\\
537	0.038\\
538	0.04\\
539	0.037\\
540	0.04\\
541	0.037\\
542	0.039\\
543	0.036\\
544	0.038\\
545	0.038\\
546	0.04\\
547	0.037\\
548	0.041\\
549	0.037\\
550	0.042\\
551	0.038\\
552	0.045\\
553	0.041\\
554	0.045\\
555	0.042\\
556	0.046\\
557	0.041\\
558	0.045\\
559	0.041\\
560	0.047\\
561	0.041\\
562	0.045\\
563	0.04\\
564	0.036\\
565	0.043\\
566	0.04\\
567	0.045\\
568	0.04\\
569	0.042\\
570	0.04\\
571	0.045\\
572	0.039\\
573	0.045\\
574	0.04\\
575	0.045\\
576	0.039\\
577	0.049\\
578	0.043\\
579	0.049\\
580	0.042\\
581	0.05\\
582	0.044\\
583	0.05\\
584	0.046\\
585	0.048\\
586	0.043\\
587	0.047\\
588	0.043\\
589	0.047\\
590	0.04\\
591	0.043\\
592	0.04\\
593	0.036\\
594	0.04\\
595	0.036\\
596	0.04\\
597	0.035\\
598	0.037\\
599	0.033\\
600	0.037\\
601	0.034\\
602	0.039\\
603	0.037\\
604	0.041\\
605	0.039\\
606	0.04\\
607	0.037\\
608	0.042\\
609	0.039\\
610	0.044\\
611	0.038\\
612	0.045\\
613	0.04\\
614	0.046\\
615	0.042\\
616	0.045\\
617	0.043\\
618	0.045\\
619	0.039\\
620	0.042\\
621	0.039\\
622	0.036\\
623	0.039\\
624	0.036\\
625	0.04\\
626	0.036\\
627	0.041\\
628	0.035\\
629	0.041\\
630	0.035\\
631	0.041\\
632	0.038\\
633	0.04\\
634	0.037\\
635	0.04\\
636	0.037\\
637	0.042\\
638	0.037\\
639	0.04\\
640	0.037\\
641	0.041\\
642	0.036\\
643	0.042\\
644	0.038\\
645	0.042\\
646	0.036\\
647	0.041\\
648	0.039\\
649	0.041\\
650	0.038\\
651	0.044\\
652	0.041\\
653	0.038\\
654	0.04\\
655	0.04\\
656	0.041\\
657	0.04\\
658	0.041\\
659	0.038\\
660	0.042\\
661	0.038\\
662	0.039\\
663	0.036\\
664	0.042\\
665	0.039\\
666	0.04\\
667	0.039\\
668	0.042\\
669	0.039\\
670	0.04\\
671	0.04\\
672	0.041\\
673	0.038\\
674	0.04\\
675	0.038\\
676	0.042\\
677	0.041\\
678	0.043\\
679	0.042\\
680	0.043\\
681	0.041\\
682	0.042\\
683	0.04\\
684	0.039\\
685	0.041\\
686	0.039\\
687	0.042\\
688	0.037\\
689	0.041\\
690	0.036\\
691	0.04\\
692	0.036\\
693	0.041\\
694	0.038\\
695	0.042\\
696	0.039\\
697	0.043\\
698	0.038\\
699	0.041\\
700	0.035\\
701	0.039\\
702	0.036\\
703	0.038\\
704	0.034\\
705	0.041\\
706	0.034\\
707	0.04\\
708	0.038\\
709	0.042\\
710	0.037\\
711	0.04\\
712	0.037\\
713	0.043\\
714	0.041\\
715	0.043\\
716	0.042\\
717	0.039\\
718	0.042\\
719	0.039\\
720	0.041\\
721	0.037\\
722	0.041\\
723	0.038\\
724	0.043\\
725	0.041\\
726	0.044\\
727	0.041\\
728	0.045\\
729	0.042\\
730	0.044\\
731	0.043\\
732	0.043\\
733	0.042\\
734	0.044\\
735	0.042\\
736	0.044\\
737	0.042\\
738	0.046\\
739	0.042\\
740	0.044\\
741	0.043\\
742	0.044\\
743	0.042\\
744	0.043\\
745	0.041\\
746	0.043\\
747	0.041\\
748	0.043\\
749	0.041\\
750	0.037\\
751	0.042\\
752	0.038\\
753	0.041\\
754	0.039\\
755	0.04\\
756	0.039\\
757	0.04\\
758	0.038\\
759	0.04\\
760	0.038\\
761	0.04\\
762	0.039\\
763	0.042\\
764	0.038\\
765	0.04\\
766	0.039\\
767	0.041\\
768	0.039\\
769	0.043\\
770	0.041\\
771	0.042\\
772	0.04\\
773	0.045\\
774	0.039\\
775	0.045\\
776	0.041\\
777	0.043\\
778	0.041\\
779	0.043\\
780	0.04\\
781	0.048\\
782	0.043\\
783	0.038\\
784	0.043\\
785	0.041\\
786	0.045\\
787	0.045\\
788	0.048\\
789	0.042\\
790	0.044\\
791	0.041\\
792	0.046\\
793	0.042\\
794	0.047\\
795	0.044\\
796	0.046\\
797	0.044\\
798	0.045\\
799	0.044\\
800	0.045\\
801	0.042\\
802	0.044\\
803	0.042\\
804	0.045\\
805	0.04\\
806	0.047\\
807	0.041\\
808	0.046\\
809	0.043\\
810	0.048\\
811	0.041\\
812	0.046\\
813	0.043\\
814	0.047\\
815	0.043\\
816	0.045\\
817	0.043\\
818	0.038\\
819	0.043\\
820	0.041\\
821	0.043\\
822	0.041\\
823	0.041\\
824	0.038\\
825	0.045\\
826	0.041\\
827	0.042\\
828	0.039\\
829	0.044\\
830	0.042\\
831	0.047\\
832	0.044\\
833	0.049\\
834	0.041\\
835	0.046\\
836	0.042\\
837	0.044\\
838	0.041\\
839	0.044\\
840	0.043\\
841	0.045\\
842	0.04\\
843	0.046\\
844	0.041\\
845	0.043\\
846	0.04\\
847	0.047\\
848	0.045\\
849	0.047\\
850	0.044\\
851	0.047\\
852	0.043\\
853	0.041\\
854	0.042\\
855	0.039\\
856	0.041\\
857	0.04\\
858	0.042\\
859	0.04\\
860	0.042\\
861	0.039\\
862	0.041\\
863	0.038\\
864	0.041\\
865	0.039\\
866	0.04\\
867	0.039\\
868	0.041\\
869	0.04\\
870	0.043\\
871	0.039\\
872	0.042\\
873	0.039\\
874	0.041\\
875	0.039\\
876	0.04\\
877	0.037\\
878	0.043\\
879	0.038\\
880	0.045\\
881	0.039\\
882	0.042\\
883	0.038\\
884	0.042\\
885	0.038\\
886	0.043\\
887	0.037\\
888	0.035\\
889	0.038\\
890	0.033\\
891	0.036\\
892	0.034\\
893	0.037\\
894	0.033\\
895	0.036\\
896	0.034\\
897	0.038\\
898	0.034\\
899	0.04\\
900	0.035\\
901	0.036\\
902	0.034\\
903	0.039\\
904	0.037\\
905	0.044\\
906	0.041\\
907	0.043\\
908	0.04\\
909	0.043\\
910	0.041\\
911	0.043\\
912	0.04\\
913	0.043\\
914	0.04\\
915	0.045\\
916	0.04\\
917	0.041\\
918	0.038\\
919	0.04\\
920	0.036\\
921	0.04\\
922	0.038\\
923	0.04\\
924	0.038\\
925	0.037\\
926	0.038\\
927	0.037\\
928	0.04\\
929	0.038\\
930	0.04\\
931	0.038\\
932	0.042\\
933	0.039\\
934	0.04\\
935	0.037\\
936	0.04\\
937	0.038\\
938	0.04\\
939	0.038\\
940	0.042\\
941	0.038\\
942	0.041\\
943	0.039\\
944	0.041\\
945	0.038\\
946	0.042\\
947	0.039\\
948	0.042\\
949	0.038\\
950	0.043\\
951	0.041\\
952	0.044\\
953	0.038\\
954	0.042\\
955	0.038\\
956	0.042\\
957	0.039\\
958	0.044\\
959	0.038\\
960	0.042\\
961	0.037\\
962	0.035\\
963	0.04\\
964	0.037\\
965	0.044\\
966	0.04\\
967	0.043\\
968	0.041\\
969	0.041\\
970	0.038\\
971	0.041\\
972	0.039\\
973	0.041\\
974	0.04\\
975	0.043\\
976	0.04\\
977	0.044\\
978	0.037\\
979	0.043\\
980	0.039\\
981	0.042\\
982	0.041\\
983	0.043\\
984	0.042\\
985	0.042\\
986	0.041\\
987	0.043\\
988	0.042\\
989	0.046\\
990	0.045\\
991	0.047\\
992	0.044\\
993	0.046\\
994	0.043\\
995	0.045\\
996	0.041\\
997	0.044\\
998	0.043\\
999	0.046\\
1000	0.043\\
};
\addlegendentry{null test}

\addplot [color=black!50!green]
  table[row sep=crcr]{%
1	0\\
2	0.505\\
3	0.261\\
4	0.135\\
5	0.395\\
6	0.23\\
7	0.14\\
8	0.306\\
9	0.201\\
10	0.124\\
11	0.232\\
12	0.156\\
13	0.108\\
14	0.176\\
15	0.116\\
16	0.08\\
17	0.147\\
18	0.102\\
19	0.072\\
20	0.119\\
21	0.088\\
22	0.14\\
23	0.096\\
24	0.065\\
25	0.106\\
26	0.075\\
27	0.123\\
28	0.094\\
29	0.066\\
30	0.105\\
31	0.077\\
32	0.114\\
33	0.088\\
34	0.057\\
35	0.087\\
36	0.06\\
37	0.094\\
38	0.067\\
39	0.047\\
40	0.084\\
41	0.056\\
42	0.091\\
43	0.063\\
44	0.049\\
45	0.069\\
46	0.046\\
47	0.066\\
48	0.05\\
49	0.04\\
50	0.058\\
51	0.045\\
52	0.069\\
53	0.053\\
54	0.074\\
55	0.056\\
56	0.036\\
57	0.058\\
58	0.038\\
59	0.064\\
60	0.046\\
61	0.036\\
62	0.051\\
63	0.039\\
64	0.064\\
65	0.049\\
66	0.069\\
67	0.056\\
68	0.041\\
69	0.06\\
70	0.051\\
71	0.071\\
72	0.055\\
73	0.067\\
74	0.056\\
75	0.041\\
76	0.06\\
77	0.043\\
78	0.062\\
79	0.044\\
80	0.058\\
81	0.048\\
82	0.038\\
83	0.05\\
84	0.042\\
85	0.049\\
86	0.041\\
87	0.053\\
88	0.042\\
89	0.038\\
90	0.048\\
91	0.036\\
92	0.052\\
93	0.041\\
94	0.054\\
95	0.039\\
96	0.054\\
97	0.046\\
98	0.034\\
99	0.045\\
100	0.034\\
101	0.046\\
102	0.037\\
103	0.046\\
104	0.034\\
105	0.028\\
106	0.037\\
107	0.033\\
108	0.044\\
109	0.035\\
110	0.045\\
111	0.039\\
112	0.046\\
113	0.037\\
114	0.028\\
115	0.034\\
116	0.03\\
117	0.038\\
118	0.03\\
119	0.037\\
120	0.028\\
121	0.034\\
122	0.031\\
123	0.029\\
124	0.034\\
125	0.031\\
126	0.033\\
127	0.033\\
128	0.039\\
129	0.035\\
130	0.029\\
131	0.031\\
132	0.029\\
133	0.034\\
134	0.03\\
135	0.033\\
136	0.029\\
137	0.037\\
138	0.032\\
139	0.029\\
140	0.032\\
141	0.028\\
142	0.036\\
143	0.031\\
144	0.037\\
145	0.032\\
146	0.042\\
147	0.032\\
148	0.044\\
149	0.034\\
150	0.027\\
151	0.035\\
152	0.03\\
153	0.037\\
154	0.032\\
155	0.037\\
156	0.032\\
157	0.04\\
158	0.031\\
159	0.026\\
160	0.033\\
161	0.031\\
162	0.036\\
163	0.032\\
164	0.035\\
165	0.033\\
166	0.04\\
167	0.032\\
168	0.026\\
169	0.034\\
170	0.029\\
171	0.034\\
172	0.03\\
173	0.035\\
174	0.028\\
175	0.034\\
176	0.032\\
177	0.038\\
178	0.03\\
179	0.027\\
180	0.034\\
181	0.028\\
182	0.033\\
183	0.03\\
184	0.036\\
185	0.031\\
186	0.037\\
187	0.031\\
188	0.026\\
189	0.033\\
190	0.025\\
191	0.032\\
192	0.028\\
193	0.034\\
194	0.028\\
195	0.035\\
196	0.029\\
197	0.038\\
198	0.035\\
199	0.025\\
200	0.031\\
201	0.028\\
202	0.032\\
203	0.032\\
204	0.034\\
205	0.03\\
206	0.037\\
207	0.034\\
208	0.037\\
209	0.03\\
210	0.028\\
211	0.032\\
212	0.028\\
213	0.035\\
214	0.029\\
215	0.04\\
216	0.031\\
217	0.035\\
218	0.031\\
219	0.034\\
220	0.028\\
221	0.026\\
222	0.027\\
223	0.025\\
224	0.028\\
225	0.028\\
226	0.032\\
227	0.026\\
228	0.028\\
229	0.026\\
230	0.03\\
231	0.026\\
232	0.024\\
233	0.029\\
234	0.024\\
235	0.028\\
236	0.025\\
237	0.03\\
238	0.028\\
239	0.031\\
240	0.027\\
241	0.032\\
242	0.027\\
243	0.026\\
244	0.028\\
245	0.022\\
246	0.024\\
247	0.021\\
248	0.027\\
249	0.023\\
250	0.031\\
251	0.028\\
252	0.032\\
253	0.03\\
254	0.031\\
255	0.026\\
256	0.023\\
257	0.028\\
258	0.024\\
259	0.028\\
260	0.025\\
261	0.029\\
262	0.022\\
263	0.027\\
264	0.024\\
265	0.027\\
266	0.026\\
267	0.021\\
268	0.026\\
269	0.02\\
270	0.024\\
271	0.021\\
272	0.025\\
273	0.024\\
274	0.026\\
275	0.023\\
276	0.027\\
277	0.023\\
278	0.027\\
279	0.024\\
280	0.019\\
281	0.022\\
282	0.022\\
283	0.023\\
284	0.023\\
285	0.025\\
286	0.022\\
287	0.023\\
288	0.021\\
289	0.023\\
290	0.022\\
291	0.02\\
292	0.021\\
293	0.021\\
294	0.021\\
295	0.019\\
296	0.021\\
297	0.019\\
298	0.024\\
299	0.022\\
300	0.024\\
301	0.022\\
302	0.025\\
303	0.022\\
304	0.02\\
305	0.022\\
306	0.02\\
307	0.024\\
308	0.022\\
309	0.026\\
310	0.023\\
311	0.024\\
312	0.023\\
313	0.025\\
314	0.022\\
315	0.024\\
316	0.023\\
317	0.019\\
318	0.021\\
319	0.021\\
320	0.022\\
321	0.019\\
322	0.02\\
323	0.019\\
324	0.019\\
325	0.019\\
326	0.019\\
327	0.019\\
328	0.019\\
329	0.018\\
330	0.018\\
331	0.019\\
332	0.018\\
333	0.021\\
334	0.017\\
335	0.02\\
336	0.017\\
337	0.018\\
338	0.017\\
339	0.02\\
340	0.017\\
341	0.019\\
342	0.016\\
343	0.016\\
344	0.017\\
345	0.015\\
346	0.02\\
347	0.016\\
348	0.019\\
349	0.016\\
350	0.018\\
351	0.017\\
352	0.018\\
353	0.016\\
354	0.017\\
355	0.015\\
356	0.016\\
357	0.014\\
358	0.013\\
359	0.017\\
360	0.016\\
361	0.018\\
362	0.016\\
363	0.017\\
364	0.014\\
365	0.016\\
366	0.015\\
367	0.015\\
368	0.015\\
369	0.015\\
370	0.015\\
371	0.014\\
372	0.015\\
373	0.015\\
374	0.015\\
375	0.015\\
376	0.016\\
377	0.015\\
378	0.017\\
379	0.014\\
380	0.016\\
381	0.014\\
382	0.019\\
383	0.015\\
384	0.018\\
385	0.015\\
386	0.013\\
387	0.016\\
388	0.013\\
389	0.015\\
390	0.013\\
391	0.017\\
392	0.014\\
393	0.019\\
394	0.016\\
395	0.018\\
396	0.016\\
397	0.018\\
398	0.016\\
399	0.015\\
400	0.018\\
401	0.016\\
402	0.018\\
403	0.017\\
404	0.017\\
405	0.015\\
406	0.017\\
407	0.015\\
408	0.015\\
409	0.014\\
410	0.017\\
411	0.011\\
412	0.015\\
413	0.012\\
414	0.009\\
415	0.014\\
416	0.012\\
417	0.014\\
418	0.012\\
419	0.013\\
420	0.01\\
421	0.014\\
422	0.012\\
423	0.015\\
424	0.014\\
425	0.016\\
426	0.014\\
427	0.015\\
428	0.013\\
429	0.012\\
430	0.014\\
431	0.013\\
432	0.014\\
433	0.013\\
434	0.016\\
435	0.015\\
436	0.017\\
437	0.016\\
438	0.017\\
439	0.014\\
440	0.016\\
441	0.014\\
442	0.015\\
443	0.015\\
444	0.014\\
445	0.014\\
446	0.014\\
447	0.014\\
448	0.013\\
449	0.014\\
450	0.013\\
451	0.014\\
452	0.014\\
453	0.016\\
454	0.015\\
455	0.017\\
456	0.016\\
457	0.019\\
458	0.018\\
459	0.017\\
460	0.017\\
461	0.015\\
462	0.016\\
463	0.014\\
464	0.016\\
465	0.014\\
466	0.015\\
467	0.013\\
468	0.016\\
469	0.013\\
470	0.015\\
471	0.011\\
472	0.013\\
473	0.011\\
474	0.011\\
475	0.013\\
476	0.011\\
477	0.014\\
478	0.012\\
479	0.015\\
480	0.015\\
481	0.015\\
482	0.014\\
483	0.015\\
484	0.014\\
485	0.017\\
486	0.014\\
487	0.017\\
488	0.015\\
489	0.017\\
490	0.016\\
491	0.016\\
492	0.017\\
493	0.015\\
494	0.017\\
495	0.017\\
496	0.019\\
497	0.017\\
498	0.019\\
499	0.018\\
500	0.019\\
501	0.018\\
502	0.02\\
503	0.019\\
504	0.023\\
505	0.019\\
506	0.017\\
507	0.02\\
508	0.018\\
509	0.02\\
510	0.017\\
511	0.018\\
512	0.017\\
513	0.018\\
514	0.017\\
515	0.02\\
516	0.017\\
517	0.018\\
518	0.017\\
519	0.017\\
520	0.017\\
521	0.017\\
522	0.017\\
523	0.016\\
524	0.018\\
525	0.017\\
526	0.019\\
527	0.017\\
528	0.019\\
529	0.019\\
530	0.019\\
531	0.019\\
532	0.019\\
533	0.016\\
534	0.017\\
535	0.017\\
536	0.02\\
537	0.016\\
538	0.015\\
539	0.018\\
540	0.018\\
541	0.019\\
542	0.015\\
543	0.018\\
544	0.016\\
545	0.019\\
546	0.014\\
547	0.015\\
548	0.014\\
549	0.016\\
550	0.015\\
551	0.018\\
552	0.017\\
553	0.018\\
554	0.017\\
555	0.016\\
556	0.017\\
557	0.016\\
558	0.017\\
559	0.014\\
560	0.015\\
561	0.013\\
562	0.014\\
563	0.014\\
564	0.015\\
565	0.014\\
566	0.018\\
567	0.015\\
568	0.017\\
569	0.017\\
570	0.017\\
571	0.014\\
572	0.014\\
573	0.015\\
574	0.015\\
575	0.016\\
576	0.013\\
577	0.016\\
578	0.013\\
579	0.013\\
580	0.013\\
581	0.016\\
582	0.014\\
583	0.017\\
584	0.015\\
585	0.017\\
586	0.017\\
587	0.017\\
588	0.016\\
589	0.015\\
590	0.016\\
591	0.015\\
592	0.016\\
593	0.016\\
594	0.018\\
595	0.015\\
596	0.016\\
597	0.015\\
598	0.017\\
599	0.014\\
600	0.016\\
601	0.016\\
602	0.017\\
603	0.017\\
604	0.017\\
605	0.016\\
606	0.013\\
607	0.014\\
608	0.014\\
609	0.014\\
610	0.013\\
611	0.016\\
612	0.014\\
613	0.015\\
614	0.014\\
615	0.016\\
616	0.016\\
617	0.016\\
618	0.014\\
619	0.015\\
620	0.013\\
621	0.015\\
622	0.014\\
623	0.012\\
624	0.016\\
625	0.014\\
626	0.015\\
627	0.014\\
628	0.015\\
629	0.015\\
630	0.016\\
631	0.015\\
632	0.016\\
633	0.015\\
634	0.016\\
635	0.015\\
636	0.016\\
637	0.015\\
638	0.015\\
639	0.015\\
640	0.015\\
641	0.015\\
642	0.014\\
643	0.015\\
644	0.014\\
645	0.015\\
646	0.014\\
647	0.016\\
648	0.014\\
649	0.016\\
650	0.016\\
651	0.016\\
652	0.014\\
653	0.017\\
654	0.015\\
655	0.016\\
656	0.014\\
657	0.014\\
658	0.013\\
659	0.013\\
660	0.015\\
661	0.015\\
662	0.016\\
663	0.014\\
664	0.015\\
665	0.012\\
666	0.017\\
667	0.015\\
668	0.016\\
669	0.016\\
670	0.017\\
671	0.015\\
672	0.018\\
673	0.016\\
674	0.017\\
675	0.016\\
676	0.015\\
677	0.018\\
678	0.015\\
679	0.017\\
680	0.016\\
681	0.016\\
682	0.016\\
683	0.017\\
684	0.016\\
685	0.017\\
686	0.014\\
687	0.016\\
688	0.015\\
689	0.016\\
690	0.015\\
691	0.016\\
692	0.016\\
693	0.017\\
694	0.017\\
695	0.014\\
696	0.017\\
697	0.014\\
698	0.015\\
699	0.013\\
700	0.015\\
701	0.014\\
702	0.014\\
703	0.014\\
704	0.016\\
705	0.015\\
706	0.017\\
707	0.015\\
708	0.016\\
709	0.016\\
710	0.018\\
711	0.016\\
712	0.014\\
713	0.016\\
714	0.014\\
715	0.015\\
716	0.013\\
717	0.014\\
718	0.014\\
719	0.014\\
720	0.014\\
721	0.015\\
722	0.014\\
723	0.015\\
724	0.014\\
725	0.017\\
726	0.016\\
727	0.019\\
728	0.014\\
729	0.018\\
730	0.016\\
731	0.013\\
732	0.014\\
733	0.014\\
734	0.014\\
735	0.014\\
736	0.014\\
737	0.014\\
738	0.015\\
739	0.014\\
740	0.015\\
741	0.014\\
742	0.017\\
743	0.013\\
744	0.015\\
745	0.014\\
746	0.015\\
747	0.013\\
748	0.013\\
749	0.013\\
750	0.012\\
751	0.013\\
752	0.013\\
753	0.014\\
754	0.012\\
755	0.012\\
756	0.011\\
757	0.012\\
758	0.011\\
759	0.014\\
760	0.011\\
761	0.015\\
762	0.01\\
763	0.012\\
764	0.011\\
765	0.013\\
766	0.012\\
767	0.015\\
768	0.011\\
769	0.008\\
770	0.01\\
771	0.01\\
772	0.011\\
773	0.011\\
774	0.011\\
775	0.01\\
776	0.011\\
777	0.01\\
778	0.011\\
779	0.011\\
780	0.013\\
781	0.012\\
782	0.014\\
783	0.012\\
784	0.014\\
785	0.011\\
786	0.014\\
787	0.012\\
788	0.011\\
789	0.013\\
790	0.013\\
791	0.013\\
792	0.013\\
793	0.013\\
794	0.013\\
795	0.014\\
796	0.013\\
797	0.014\\
798	0.012\\
799	0.013\\
800	0.012\\
801	0.013\\
802	0.012\\
803	0.014\\
804	0.012\\
805	0.013\\
806	0.011\\
807	0.013\\
808	0.013\\
809	0.013\\
810	0.014\\
811	0.013\\
812	0.013\\
813	0.012\\
814	0.012\\
815	0.012\\
816	0.013\\
817	0.011\\
818	0.012\\
819	0.011\\
820	0.011\\
821	0.011\\
822	0.012\\
823	0.01\\
824	0.012\\
825	0.011\\
826	0.011\\
827	0.011\\
828	0.011\\
829	0.011\\
830	0.01\\
831	0.01\\
832	0.01\\
833	0.01\\
834	0.009\\
835	0.01\\
836	0.01\\
837	0.01\\
838	0.01\\
839	0.01\\
840	0.009\\
841	0.01\\
842	0.008\\
843	0.01\\
844	0.009\\
845	0.01\\
846	0.009\\
847	0.011\\
848	0.008\\
849	0.007\\
850	0.009\\
851	0.007\\
852	0.008\\
853	0.008\\
854	0.008\\
855	0.008\\
856	0.009\\
857	0.009\\
858	0.009\\
859	0.008\\
860	0.009\\
861	0.007\\
862	0.009\\
863	0.008\\
864	0.01\\
865	0.009\\
866	0.009\\
867	0.009\\
868	0.007\\
869	0.009\\
870	0.007\\
871	0.009\\
872	0.008\\
873	0.008\\
874	0.008\\
875	0.008\\
876	0.007\\
877	0.008\\
878	0.007\\
879	0.009\\
880	0.008\\
881	0.008\\
882	0.008\\
883	0.008\\
884	0.008\\
885	0.009\\
886	0.008\\
887	0.008\\
888	0.008\\
889	0.008\\
890	0.008\\
891	0.008\\
892	0.009\\
893	0.007\\
894	0.008\\
895	0.006\\
896	0.007\\
897	0.006\\
898	0.009\\
899	0.008\\
900	0.008\\
901	0.008\\
902	0.008\\
903	0.008\\
904	0.008\\
905	0.008\\
906	0.009\\
907	0.008\\
908	0.01\\
909	0.01\\
910	0.009\\
911	0.009\\
912	0.008\\
913	0.009\\
914	0.008\\
915	0.009\\
916	0.009\\
917	0.009\\
918	0.009\\
919	0.009\\
920	0.009\\
921	0.009\\
922	0.009\\
923	0.009\\
924	0.009\\
925	0.009\\
926	0.009\\
927	0.009\\
928	0.009\\
929	0.008\\
930	0.009\\
931	0.009\\
932	0.009\\
933	0.008\\
934	0.009\\
935	0.008\\
936	0.01\\
937	0.008\\
938	0.009\\
939	0.008\\
940	0.009\\
941	0.008\\
942	0.008\\
943	0.008\\
944	0.009\\
945	0.008\\
946	0.01\\
947	0.008\\
948	0.009\\
949	0.008\\
950	0.007\\
951	0.007\\
952	0.006\\
953	0.007\\
954	0.006\\
955	0.007\\
956	0.007\\
957	0.009\\
958	0.008\\
959	0.008\\
960	0.007\\
961	0.007\\
962	0.007\\
963	0.007\\
964	0.007\\
965	0.007\\
966	0.007\\
967	0.007\\
968	0.006\\
969	0.009\\
970	0.007\\
971	0.006\\
972	0.006\\
973	0.006\\
974	0.008\\
975	0.007\\
976	0.008\\
977	0.007\\
978	0.007\\
979	0.007\\
980	0.008\\
981	0.008\\
982	0.008\\
983	0.007\\
984	0.009\\
985	0.008\\
986	0.009\\
987	0.008\\
988	0.009\\
989	0.008\\
990	0.008\\
991	0.008\\
992	0.009\\
993	0.006\\
994	0.006\\
995	0.006\\
996	0.006\\
997	0.006\\
998	0.006\\
999	0.006\\
1000	0.006\\
};
\addlegendentry{Bayes test}

\end{axis}
\end{tikzpicture}%
      \caption{Comparison of the rejection probability for the null and the Bayesian test when $\model_0$ is true.}
    \end{figure}
    \only<article>{Now we can use this Bayesian test, with uniform prior, to see how well it performs. While the plain null hypothesis test has a fixed rejection rate of $0.05$, the Bayesian test's rejection rate converges to 0 as we collect more data.}
  }
  \only<3>{
    \begin{figure}[H]
      % This file was created by matlab2tikz.
%
%The latest updates can be retrieved from
%  http://www.mathworks.com/matlabcentral/fileexchange/22022-matlab2tikz-matlab2tikz
%where you can also make suggestions and rate matlab2tikz.
%
\begin{tikzpicture}

\begin{axis}[%
width=0.951\fwidth,
height=\fheight,
at={(0\fwidth,0\fheight)},
scale only axis,
xmin=0,
xmax=1000,
ymin=0,
ymax=1,
axis background/.style={fill=white},
title={Rejection of null hypothesis for Bernoulli(0.45)},
legend style={legend cell align=left, align=left, legend plot pos=left, draw=black}
]
\addplot [color=blue]
  table[row sep=crcr]{%
1	1\\
2	0.311\\
3	0.173\\
4	0.102\\
5	0.265\\
6	0.173\\
7	0.107\\
8	0.235\\
9	0.162\\
10	0.102\\
11	0.203\\
12	0.146\\
13	0.245\\
14	0.181\\
15	0.132\\
16	0.218\\
17	0.162\\
18	0.261\\
19	0.194\\
20	0.151\\
21	0.213\\
22	0.172\\
23	0.242\\
24	0.19\\
25	0.146\\
26	0.217\\
27	0.163\\
28	0.232\\
29	0.189\\
30	0.249\\
31	0.2\\
32	0.162\\
33	0.227\\
34	0.189\\
35	0.248\\
36	0.211\\
37	0.259\\
38	0.219\\
39	0.183\\
40	0.234\\
41	0.189\\
42	0.251\\
43	0.22\\
44	0.264\\
45	0.228\\
46	0.186\\
47	0.247\\
48	0.198\\
49	0.247\\
50	0.215\\
51	0.263\\
52	0.225\\
53	0.279\\
54	0.24\\
55	0.205\\
56	0.252\\
57	0.218\\
58	0.262\\
59	0.224\\
60	0.268\\
61	0.234\\
62	0.277\\
63	0.245\\
64	0.216\\
65	0.251\\
66	0.219\\
67	0.27\\
68	0.226\\
69	0.28\\
70	0.243\\
71	0.287\\
72	0.259\\
73	0.222\\
74	0.263\\
75	0.234\\
76	0.277\\
77	0.238\\
78	0.282\\
79	0.256\\
80	0.296\\
81	0.262\\
82	0.305\\
83	0.28\\
84	0.247\\
85	0.286\\
86	0.26\\
87	0.3\\
88	0.271\\
89	0.309\\
90	0.282\\
91	0.316\\
92	0.282\\
93	0.324\\
94	0.291\\
95	0.262\\
96	0.305\\
97	0.271\\
98	0.314\\
99	0.285\\
100	0.323\\
101	0.301\\
102	0.346\\
103	0.312\\
104	0.35\\
105	0.312\\
106	0.359\\
107	0.32\\
108	0.294\\
109	0.332\\
110	0.305\\
111	0.339\\
112	0.31\\
113	0.346\\
114	0.318\\
115	0.364\\
116	0.328\\
117	0.372\\
118	0.34\\
119	0.38\\
120	0.355\\
121	0.318\\
122	0.358\\
123	0.327\\
124	0.366\\
125	0.33\\
126	0.366\\
127	0.338\\
128	0.375\\
129	0.34\\
130	0.382\\
131	0.349\\
132	0.379\\
133	0.352\\
134	0.327\\
135	0.363\\
136	0.333\\
137	0.367\\
138	0.338\\
139	0.367\\
140	0.345\\
141	0.388\\
142	0.363\\
143	0.384\\
144	0.364\\
145	0.387\\
146	0.366\\
147	0.401\\
148	0.38\\
149	0.345\\
150	0.378\\
151	0.356\\
152	0.389\\
153	0.363\\
154	0.393\\
155	0.371\\
156	0.411\\
157	0.379\\
158	0.408\\
159	0.38\\
160	0.414\\
161	0.383\\
162	0.412\\
163	0.389\\
164	0.363\\
165	0.393\\
166	0.362\\
167	0.401\\
168	0.374\\
169	0.409\\
170	0.385\\
171	0.412\\
172	0.39\\
173	0.418\\
174	0.397\\
175	0.424\\
176	0.409\\
177	0.441\\
178	0.417\\
179	0.452\\
180	0.431\\
181	0.399\\
182	0.428\\
183	0.403\\
184	0.428\\
185	0.4\\
186	0.443\\
187	0.411\\
188	0.447\\
189	0.421\\
190	0.452\\
191	0.428\\
192	0.455\\
193	0.433\\
194	0.469\\
195	0.446\\
196	0.418\\
197	0.453\\
198	0.426\\
199	0.46\\
200	0.426\\
201	0.461\\
202	0.434\\
203	0.459\\
204	0.443\\
205	0.475\\
206	0.452\\
207	0.483\\
208	0.459\\
209	0.483\\
210	0.464\\
211	0.491\\
212	0.472\\
213	0.504\\
214	0.478\\
215	0.448\\
216	0.482\\
217	0.46\\
218	0.488\\
219	0.463\\
220	0.486\\
221	0.462\\
222	0.485\\
223	0.464\\
224	0.488\\
225	0.461\\
226	0.49\\
227	0.463\\
228	0.505\\
229	0.482\\
230	0.513\\
231	0.485\\
232	0.457\\
233	0.485\\
234	0.465\\
235	0.494\\
236	0.47\\
237	0.503\\
238	0.478\\
239	0.495\\
240	0.472\\
241	0.497\\
242	0.475\\
243	0.507\\
244	0.491\\
245	0.52\\
246	0.492\\
247	0.524\\
248	0.504\\
249	0.526\\
250	0.502\\
251	0.484\\
252	0.502\\
253	0.485\\
254	0.504\\
255	0.489\\
256	0.505\\
257	0.495\\
258	0.505\\
259	0.497\\
260	0.513\\
261	0.497\\
262	0.518\\
263	0.503\\
264	0.533\\
265	0.51\\
266	0.524\\
267	0.508\\
268	0.536\\
269	0.51\\
270	0.494\\
271	0.518\\
272	0.497\\
273	0.523\\
274	0.503\\
275	0.524\\
276	0.501\\
277	0.522\\
278	0.505\\
279	0.526\\
280	0.512\\
281	0.527\\
282	0.506\\
283	0.539\\
284	0.522\\
285	0.543\\
286	0.533\\
287	0.555\\
288	0.537\\
289	0.562\\
290	0.543\\
291	0.525\\
292	0.551\\
293	0.528\\
294	0.551\\
295	0.532\\
296	0.565\\
297	0.545\\
298	0.572\\
299	0.551\\
300	0.579\\
301	0.558\\
302	0.575\\
303	0.558\\
304	0.578\\
305	0.555\\
306	0.584\\
307	0.572\\
308	0.586\\
309	0.571\\
310	0.592\\
311	0.569\\
312	0.549\\
313	0.577\\
314	0.556\\
315	0.583\\
316	0.56\\
317	0.584\\
318	0.561\\
319	0.589\\
320	0.568\\
321	0.592\\
322	0.57\\
323	0.594\\
324	0.571\\
325	0.6\\
326	0.582\\
327	0.606\\
328	0.586\\
329	0.609\\
330	0.587\\
331	0.607\\
332	0.587\\
333	0.563\\
334	0.591\\
335	0.565\\
336	0.6\\
337	0.576\\
338	0.598\\
339	0.578\\
340	0.597\\
341	0.58\\
342	0.602\\
343	0.579\\
344	0.597\\
345	0.584\\
346	0.606\\
347	0.588\\
348	0.616\\
349	0.598\\
350	0.622\\
351	0.606\\
352	0.624\\
353	0.603\\
354	0.627\\
355	0.611\\
356	0.593\\
357	0.615\\
358	0.597\\
359	0.622\\
360	0.602\\
361	0.625\\
362	0.608\\
363	0.63\\
364	0.617\\
365	0.639\\
366	0.619\\
367	0.643\\
368	0.624\\
369	0.645\\
370	0.626\\
371	0.648\\
372	0.63\\
373	0.654\\
374	0.63\\
375	0.652\\
376	0.638\\
377	0.66\\
378	0.641\\
379	0.628\\
380	0.646\\
381	0.627\\
382	0.65\\
383	0.633\\
384	0.654\\
385	0.636\\
386	0.658\\
387	0.641\\
388	0.656\\
389	0.645\\
390	0.667\\
391	0.656\\
392	0.673\\
393	0.659\\
394	0.678\\
395	0.657\\
396	0.679\\
397	0.66\\
398	0.68\\
399	0.655\\
400	0.676\\
401	0.659\\
402	0.692\\
403	0.666\\
404	0.647\\
405	0.67\\
406	0.647\\
407	0.668\\
408	0.646\\
409	0.665\\
410	0.652\\
411	0.669\\
412	0.653\\
413	0.668\\
414	0.647\\
415	0.669\\
416	0.653\\
417	0.675\\
418	0.661\\
419	0.673\\
420	0.66\\
421	0.674\\
422	0.667\\
423	0.679\\
424	0.665\\
425	0.685\\
426	0.667\\
427	0.684\\
428	0.671\\
429	0.658\\
430	0.677\\
431	0.662\\
432	0.677\\
433	0.67\\
434	0.685\\
435	0.669\\
436	0.688\\
437	0.674\\
438	0.688\\
439	0.673\\
440	0.689\\
441	0.679\\
442	0.691\\
443	0.677\\
444	0.694\\
445	0.679\\
446	0.694\\
447	0.676\\
448	0.697\\
449	0.685\\
450	0.703\\
451	0.69\\
452	0.703\\
453	0.693\\
454	0.681\\
455	0.694\\
456	0.682\\
457	0.703\\
458	0.683\\
459	0.702\\
460	0.689\\
461	0.703\\
462	0.687\\
463	0.704\\
464	0.686\\
465	0.699\\
466	0.687\\
467	0.699\\
468	0.687\\
469	0.705\\
470	0.694\\
471	0.713\\
472	0.699\\
473	0.72\\
474	0.706\\
475	0.722\\
476	0.709\\
477	0.724\\
478	0.707\\
479	0.722\\
480	0.711\\
481	0.703\\
482	0.714\\
483	0.706\\
484	0.72\\
485	0.711\\
486	0.725\\
487	0.709\\
488	0.732\\
489	0.71\\
490	0.728\\
491	0.715\\
492	0.728\\
493	0.715\\
494	0.732\\
495	0.727\\
496	0.743\\
497	0.721\\
498	0.743\\
499	0.726\\
500	0.747\\
501	0.729\\
502	0.747\\
503	0.731\\
504	0.748\\
505	0.738\\
506	0.755\\
507	0.745\\
508	0.729\\
509	0.753\\
510	0.737\\
511	0.751\\
512	0.74\\
513	0.754\\
514	0.742\\
515	0.753\\
516	0.743\\
517	0.755\\
518	0.743\\
519	0.758\\
520	0.751\\
521	0.762\\
522	0.752\\
523	0.761\\
524	0.75\\
525	0.765\\
526	0.756\\
527	0.767\\
528	0.761\\
529	0.773\\
530	0.764\\
531	0.782\\
532	0.767\\
533	0.785\\
534	0.772\\
535	0.758\\
536	0.773\\
537	0.761\\
538	0.776\\
539	0.762\\
540	0.774\\
541	0.764\\
542	0.776\\
543	0.762\\
544	0.777\\
545	0.766\\
546	0.781\\
547	0.77\\
548	0.781\\
549	0.765\\
550	0.789\\
551	0.777\\
552	0.785\\
553	0.772\\
554	0.784\\
555	0.776\\
556	0.789\\
557	0.781\\
558	0.798\\
559	0.785\\
560	0.799\\
561	0.787\\
562	0.801\\
563	0.791\\
564	0.784\\
565	0.801\\
566	0.783\\
567	0.8\\
568	0.787\\
569	0.8\\
570	0.79\\
571	0.799\\
572	0.793\\
573	0.803\\
574	0.794\\
575	0.805\\
576	0.791\\
577	0.801\\
578	0.794\\
579	0.806\\
580	0.794\\
581	0.807\\
582	0.794\\
583	0.807\\
584	0.795\\
585	0.811\\
586	0.802\\
587	0.814\\
588	0.802\\
589	0.819\\
590	0.81\\
591	0.817\\
592	0.807\\
593	0.8\\
594	0.809\\
595	0.799\\
596	0.812\\
597	0.802\\
598	0.809\\
599	0.798\\
600	0.815\\
601	0.806\\
602	0.82\\
603	0.807\\
604	0.822\\
605	0.809\\
606	0.819\\
607	0.814\\
608	0.824\\
609	0.812\\
610	0.825\\
611	0.814\\
612	0.823\\
613	0.807\\
614	0.818\\
615	0.805\\
616	0.817\\
617	0.805\\
618	0.819\\
619	0.81\\
620	0.824\\
621	0.807\\
622	0.795\\
623	0.813\\
624	0.804\\
625	0.82\\
626	0.808\\
627	0.818\\
628	0.808\\
629	0.822\\
630	0.811\\
631	0.822\\
632	0.812\\
633	0.826\\
634	0.819\\
635	0.828\\
636	0.821\\
637	0.831\\
638	0.821\\
639	0.83\\
640	0.822\\
641	0.832\\
642	0.825\\
643	0.834\\
644	0.823\\
645	0.836\\
646	0.828\\
647	0.84\\
648	0.833\\
649	0.841\\
650	0.836\\
651	0.845\\
652	0.839\\
653	0.827\\
654	0.839\\
655	0.83\\
656	0.841\\
657	0.826\\
658	0.837\\
659	0.826\\
660	0.835\\
661	0.824\\
662	0.835\\
663	0.822\\
664	0.832\\
665	0.826\\
666	0.833\\
667	0.826\\
668	0.838\\
669	0.828\\
670	0.838\\
671	0.832\\
672	0.838\\
673	0.832\\
674	0.843\\
675	0.838\\
676	0.847\\
677	0.838\\
678	0.846\\
679	0.84\\
680	0.848\\
681	0.838\\
682	0.853\\
683	0.837\\
684	0.827\\
685	0.838\\
686	0.83\\
687	0.84\\
688	0.832\\
689	0.839\\
690	0.835\\
691	0.843\\
692	0.836\\
693	0.845\\
694	0.841\\
695	0.85\\
696	0.841\\
697	0.852\\
698	0.842\\
699	0.85\\
700	0.841\\
701	0.853\\
702	0.846\\
703	0.855\\
704	0.848\\
705	0.855\\
706	0.849\\
707	0.856\\
708	0.851\\
709	0.858\\
710	0.853\\
711	0.86\\
712	0.853\\
713	0.863\\
714	0.856\\
715	0.864\\
716	0.855\\
717	0.845\\
718	0.856\\
719	0.847\\
720	0.861\\
721	0.851\\
722	0.862\\
723	0.856\\
724	0.864\\
725	0.855\\
726	0.868\\
727	0.857\\
728	0.863\\
729	0.858\\
730	0.867\\
731	0.862\\
732	0.868\\
733	0.861\\
734	0.868\\
735	0.862\\
736	0.871\\
737	0.862\\
738	0.87\\
739	0.867\\
740	0.871\\
741	0.869\\
742	0.875\\
743	0.865\\
744	0.878\\
745	0.875\\
746	0.879\\
747	0.871\\
748	0.877\\
749	0.874\\
750	0.867\\
751	0.874\\
752	0.866\\
753	0.872\\
754	0.866\\
755	0.877\\
756	0.867\\
757	0.873\\
758	0.867\\
759	0.872\\
760	0.865\\
761	0.869\\
762	0.862\\
763	0.872\\
764	0.869\\
765	0.874\\
766	0.867\\
767	0.873\\
768	0.87\\
769	0.876\\
770	0.871\\
771	0.876\\
772	0.869\\
773	0.88\\
774	0.874\\
775	0.882\\
776	0.877\\
777	0.883\\
778	0.875\\
779	0.883\\
780	0.878\\
781	0.882\\
782	0.876\\
783	0.872\\
784	0.877\\
785	0.867\\
786	0.875\\
787	0.869\\
788	0.877\\
789	0.873\\
790	0.879\\
791	0.875\\
792	0.88\\
793	0.875\\
794	0.884\\
795	0.877\\
796	0.886\\
797	0.88\\
798	0.887\\
799	0.884\\
800	0.888\\
801	0.884\\
802	0.886\\
803	0.882\\
804	0.888\\
805	0.884\\
806	0.89\\
807	0.884\\
808	0.892\\
809	0.882\\
810	0.889\\
811	0.882\\
812	0.893\\
813	0.886\\
814	0.894\\
815	0.887\\
816	0.896\\
817	0.89\\
818	0.886\\
819	0.889\\
820	0.884\\
821	0.893\\
822	0.885\\
823	0.893\\
824	0.885\\
825	0.892\\
826	0.881\\
827	0.89\\
828	0.888\\
829	0.896\\
830	0.887\\
831	0.896\\
832	0.889\\
833	0.895\\
834	0.888\\
835	0.896\\
836	0.888\\
837	0.898\\
838	0.894\\
839	0.9\\
840	0.895\\
841	0.906\\
842	0.899\\
843	0.905\\
844	0.898\\
845	0.906\\
846	0.902\\
847	0.905\\
848	0.9\\
849	0.909\\
850	0.901\\
851	0.905\\
852	0.901\\
853	0.896\\
854	0.903\\
855	0.898\\
856	0.902\\
857	0.897\\
858	0.903\\
859	0.899\\
860	0.906\\
861	0.902\\
862	0.905\\
863	0.9\\
864	0.908\\
865	0.903\\
866	0.907\\
867	0.904\\
868	0.911\\
869	0.905\\
870	0.91\\
871	0.901\\
872	0.908\\
873	0.905\\
874	0.911\\
875	0.908\\
876	0.914\\
877	0.907\\
878	0.914\\
879	0.908\\
880	0.914\\
881	0.907\\
882	0.915\\
883	0.91\\
884	0.915\\
885	0.912\\
886	0.914\\
887	0.908\\
888	0.906\\
889	0.909\\
890	0.907\\
891	0.912\\
892	0.906\\
893	0.911\\
894	0.907\\
895	0.909\\
896	0.907\\
897	0.908\\
898	0.908\\
899	0.911\\
900	0.908\\
901	0.91\\
902	0.907\\
903	0.909\\
904	0.907\\
905	0.915\\
906	0.91\\
907	0.915\\
908	0.911\\
909	0.915\\
910	0.912\\
911	0.918\\
912	0.913\\
913	0.915\\
914	0.913\\
915	0.916\\
916	0.912\\
917	0.915\\
918	0.914\\
919	0.916\\
920	0.913\\
921	0.918\\
922	0.913\\
923	0.919\\
924	0.914\\
925	0.909\\
926	0.917\\
927	0.913\\
928	0.919\\
929	0.912\\
930	0.917\\
931	0.913\\
932	0.92\\
933	0.917\\
934	0.919\\
935	0.914\\
936	0.919\\
937	0.917\\
938	0.919\\
939	0.915\\
940	0.922\\
941	0.917\\
942	0.922\\
943	0.919\\
944	0.921\\
945	0.918\\
946	0.919\\
947	0.919\\
948	0.925\\
949	0.922\\
950	0.924\\
951	0.92\\
952	0.924\\
953	0.92\\
954	0.922\\
955	0.92\\
956	0.922\\
957	0.92\\
958	0.923\\
959	0.921\\
960	0.925\\
961	0.92\\
962	0.915\\
963	0.923\\
964	0.917\\
965	0.921\\
966	0.918\\
967	0.924\\
968	0.918\\
969	0.925\\
970	0.921\\
971	0.927\\
972	0.923\\
973	0.927\\
974	0.924\\
975	0.928\\
976	0.923\\
977	0.927\\
978	0.926\\
979	0.933\\
980	0.929\\
981	0.932\\
982	0.929\\
983	0.931\\
984	0.929\\
985	0.931\\
986	0.929\\
987	0.932\\
988	0.929\\
989	0.934\\
990	0.931\\
991	0.937\\
992	0.933\\
993	0.936\\
994	0.935\\
995	0.937\\
996	0.936\\
997	0.94\\
998	0.935\\
999	0.94\\
1000	0.936\\
};
\addlegendentry{null test}

\addplot [color=black!50!green]
  table[row sep=crcr]{%
1	0\\
2	0.506\\
3	0.262\\
4	0.138\\
5	0.375\\
6	0.24\\
7	0.139\\
8	0.326\\
9	0.212\\
10	0.131\\
11	0.264\\
12	0.188\\
13	0.124\\
14	0.225\\
15	0.162\\
16	0.118\\
17	0.197\\
18	0.139\\
19	0.102\\
20	0.172\\
21	0.129\\
22	0.197\\
23	0.138\\
24	0.097\\
25	0.168\\
26	0.119\\
27	0.186\\
28	0.141\\
29	0.111\\
30	0.172\\
31	0.142\\
32	0.185\\
33	0.145\\
34	0.119\\
35	0.173\\
36	0.136\\
37	0.184\\
38	0.151\\
39	0.109\\
40	0.152\\
41	0.125\\
42	0.178\\
43	0.142\\
44	0.11\\
45	0.151\\
46	0.119\\
47	0.168\\
48	0.139\\
49	0.109\\
50	0.138\\
51	0.11\\
52	0.143\\
53	0.122\\
54	0.164\\
55	0.134\\
56	0.108\\
57	0.154\\
58	0.121\\
59	0.16\\
60	0.133\\
61	0.106\\
62	0.143\\
63	0.119\\
64	0.155\\
65	0.126\\
66	0.166\\
67	0.141\\
68	0.116\\
69	0.146\\
70	0.123\\
71	0.15\\
72	0.128\\
73	0.158\\
74	0.134\\
75	0.117\\
76	0.144\\
77	0.132\\
78	0.162\\
79	0.134\\
80	0.166\\
81	0.142\\
82	0.122\\
83	0.152\\
84	0.131\\
85	0.155\\
86	0.136\\
87	0.166\\
88	0.143\\
89	0.126\\
90	0.157\\
91	0.138\\
92	0.163\\
93	0.147\\
94	0.173\\
95	0.155\\
96	0.181\\
97	0.164\\
98	0.142\\
99	0.168\\
100	0.141\\
101	0.166\\
102	0.147\\
103	0.173\\
104	0.153\\
105	0.131\\
106	0.163\\
107	0.138\\
108	0.176\\
109	0.149\\
110	0.181\\
111	0.155\\
112	0.184\\
113	0.16\\
114	0.146\\
115	0.168\\
116	0.156\\
117	0.174\\
118	0.158\\
119	0.176\\
120	0.161\\
121	0.192\\
122	0.17\\
123	0.15\\
124	0.176\\
125	0.157\\
126	0.179\\
127	0.168\\
128	0.191\\
129	0.17\\
130	0.154\\
131	0.17\\
132	0.151\\
133	0.183\\
134	0.166\\
135	0.189\\
136	0.177\\
137	0.203\\
138	0.177\\
139	0.161\\
140	0.193\\
141	0.171\\
142	0.193\\
143	0.173\\
144	0.201\\
145	0.179\\
146	0.218\\
147	0.196\\
148	0.225\\
149	0.204\\
150	0.18\\
151	0.207\\
152	0.193\\
153	0.211\\
154	0.195\\
155	0.224\\
156	0.205\\
157	0.228\\
158	0.217\\
159	0.197\\
160	0.218\\
161	0.206\\
162	0.23\\
163	0.205\\
164	0.231\\
165	0.204\\
166	0.236\\
167	0.208\\
168	0.193\\
169	0.212\\
170	0.201\\
171	0.216\\
172	0.197\\
173	0.22\\
174	0.201\\
175	0.226\\
176	0.207\\
177	0.226\\
178	0.211\\
179	0.198\\
180	0.214\\
181	0.201\\
182	0.224\\
183	0.201\\
184	0.226\\
185	0.206\\
186	0.236\\
187	0.21\\
188	0.196\\
189	0.214\\
190	0.196\\
191	0.224\\
192	0.206\\
193	0.231\\
194	0.216\\
195	0.235\\
196	0.219\\
197	0.236\\
198	0.219\\
199	0.205\\
200	0.226\\
201	0.213\\
202	0.234\\
203	0.212\\
204	0.234\\
205	0.218\\
206	0.239\\
207	0.217\\
208	0.236\\
209	0.22\\
210	0.207\\
211	0.226\\
212	0.216\\
213	0.231\\
214	0.219\\
215	0.233\\
216	0.218\\
217	0.237\\
218	0.222\\
219	0.25\\
220	0.227\\
221	0.206\\
222	0.225\\
223	0.208\\
224	0.237\\
225	0.222\\
226	0.242\\
227	0.228\\
228	0.25\\
229	0.236\\
230	0.26\\
231	0.24\\
232	0.228\\
233	0.251\\
234	0.232\\
235	0.258\\
236	0.241\\
237	0.265\\
238	0.249\\
239	0.273\\
240	0.258\\
241	0.284\\
242	0.268\\
243	0.248\\
244	0.27\\
245	0.256\\
246	0.269\\
247	0.257\\
248	0.274\\
249	0.258\\
250	0.278\\
251	0.262\\
252	0.284\\
253	0.277\\
254	0.285\\
255	0.275\\
256	0.265\\
257	0.275\\
258	0.261\\
259	0.277\\
260	0.264\\
261	0.286\\
262	0.271\\
263	0.283\\
264	0.267\\
265	0.286\\
266	0.266\\
267	0.257\\
268	0.273\\
269	0.26\\
270	0.284\\
271	0.261\\
272	0.284\\
273	0.267\\
274	0.292\\
275	0.271\\
276	0.291\\
277	0.277\\
278	0.307\\
279	0.283\\
280	0.275\\
281	0.288\\
282	0.269\\
283	0.292\\
284	0.275\\
285	0.293\\
286	0.277\\
287	0.299\\
288	0.279\\
289	0.3\\
290	0.292\\
291	0.276\\
292	0.293\\
293	0.282\\
294	0.298\\
295	0.282\\
296	0.301\\
297	0.286\\
298	0.308\\
299	0.291\\
300	0.309\\
301	0.285\\
302	0.316\\
303	0.293\\
304	0.282\\
305	0.305\\
306	0.286\\
307	0.301\\
308	0.284\\
309	0.304\\
310	0.295\\
311	0.312\\
312	0.291\\
313	0.314\\
314	0.295\\
315	0.316\\
316	0.3\\
317	0.285\\
318	0.304\\
319	0.291\\
320	0.314\\
321	0.292\\
322	0.317\\
323	0.302\\
324	0.323\\
325	0.309\\
326	0.326\\
327	0.314\\
328	0.334\\
329	0.321\\
330	0.298\\
331	0.319\\
332	0.304\\
333	0.325\\
334	0.304\\
335	0.326\\
336	0.309\\
337	0.334\\
338	0.31\\
339	0.334\\
340	0.32\\
341	0.344\\
342	0.327\\
343	0.311\\
344	0.33\\
345	0.317\\
346	0.339\\
347	0.322\\
348	0.34\\
349	0.325\\
350	0.341\\
351	0.32\\
352	0.352\\
353	0.331\\
354	0.353\\
355	0.334\\
356	0.353\\
357	0.341\\
358	0.32\\
359	0.338\\
360	0.321\\
361	0.35\\
362	0.33\\
363	0.348\\
364	0.332\\
365	0.348\\
366	0.333\\
367	0.358\\
368	0.345\\
369	0.363\\
370	0.348\\
371	0.327\\
372	0.345\\
373	0.325\\
374	0.347\\
375	0.336\\
376	0.357\\
377	0.337\\
378	0.355\\
379	0.334\\
380	0.366\\
381	0.349\\
382	0.363\\
383	0.353\\
384	0.373\\
385	0.36\\
386	0.343\\
387	0.361\\
388	0.341\\
389	0.356\\
390	0.342\\
391	0.358\\
392	0.346\\
393	0.374\\
394	0.356\\
395	0.373\\
396	0.358\\
397	0.38\\
398	0.36\\
399	0.341\\
400	0.367\\
401	0.352\\
402	0.375\\
403	0.349\\
404	0.37\\
405	0.349\\
406	0.368\\
407	0.357\\
408	0.378\\
409	0.357\\
410	0.382\\
411	0.359\\
412	0.384\\
413	0.366\\
414	0.352\\
415	0.368\\
416	0.354\\
417	0.373\\
418	0.357\\
419	0.372\\
420	0.357\\
421	0.385\\
422	0.368\\
423	0.387\\
424	0.371\\
425	0.391\\
426	0.37\\
427	0.399\\
428	0.383\\
429	0.36\\
430	0.385\\
431	0.371\\
432	0.385\\
433	0.371\\
434	0.385\\
435	0.373\\
436	0.394\\
437	0.377\\
438	0.4\\
439	0.383\\
440	0.399\\
441	0.387\\
442	0.409\\
443	0.394\\
444	0.373\\
445	0.4\\
446	0.381\\
447	0.397\\
448	0.378\\
449	0.402\\
450	0.383\\
451	0.404\\
452	0.387\\
453	0.416\\
454	0.402\\
455	0.426\\
456	0.412\\
457	0.425\\
458	0.409\\
459	0.394\\
460	0.409\\
461	0.39\\
462	0.408\\
463	0.392\\
464	0.413\\
465	0.401\\
466	0.415\\
467	0.405\\
468	0.419\\
469	0.406\\
470	0.423\\
471	0.411\\
472	0.425\\
473	0.409\\
474	0.401\\
475	0.415\\
476	0.402\\
477	0.418\\
478	0.404\\
479	0.415\\
480	0.406\\
481	0.423\\
482	0.411\\
483	0.429\\
484	0.417\\
485	0.432\\
486	0.417\\
487	0.442\\
488	0.423\\
489	0.446\\
490	0.427\\
491	0.419\\
492	0.434\\
493	0.415\\
494	0.438\\
495	0.42\\
496	0.443\\
497	0.426\\
498	0.449\\
499	0.435\\
500	0.455\\
501	0.437\\
502	0.457\\
503	0.436\\
504	0.454\\
505	0.443\\
506	0.425\\
507	0.448\\
508	0.438\\
509	0.451\\
510	0.436\\
511	0.453\\
512	0.437\\
513	0.456\\
514	0.44\\
515	0.457\\
516	0.436\\
517	0.459\\
518	0.44\\
519	0.458\\
520	0.446\\
521	0.432\\
522	0.45\\
523	0.434\\
524	0.447\\
525	0.437\\
526	0.452\\
527	0.436\\
528	0.454\\
529	0.439\\
530	0.457\\
531	0.445\\
532	0.458\\
533	0.441\\
534	0.457\\
535	0.443\\
536	0.461\\
537	0.446\\
538	0.434\\
539	0.45\\
540	0.436\\
541	0.449\\
542	0.435\\
543	0.454\\
544	0.442\\
545	0.458\\
546	0.444\\
547	0.454\\
548	0.443\\
549	0.463\\
550	0.446\\
551	0.462\\
552	0.452\\
553	0.468\\
554	0.455\\
555	0.44\\
556	0.461\\
557	0.447\\
558	0.463\\
559	0.448\\
560	0.466\\
561	0.45\\
562	0.464\\
563	0.453\\
564	0.473\\
565	0.458\\
566	0.477\\
567	0.467\\
568	0.479\\
569	0.465\\
570	0.485\\
571	0.469\\
572	0.455\\
573	0.473\\
574	0.46\\
575	0.475\\
576	0.465\\
577	0.475\\
578	0.467\\
579	0.478\\
580	0.471\\
581	0.488\\
582	0.477\\
583	0.491\\
584	0.481\\
585	0.499\\
586	0.483\\
587	0.5\\
588	0.485\\
589	0.469\\
590	0.484\\
591	0.471\\
592	0.484\\
593	0.477\\
594	0.49\\
595	0.472\\
596	0.49\\
597	0.481\\
598	0.49\\
599	0.48\\
600	0.497\\
601	0.48\\
602	0.499\\
603	0.485\\
604	0.506\\
605	0.494\\
606	0.485\\
607	0.503\\
608	0.491\\
609	0.507\\
610	0.494\\
611	0.509\\
612	0.489\\
613	0.506\\
614	0.491\\
615	0.511\\
616	0.495\\
617	0.512\\
618	0.493\\
619	0.512\\
620	0.496\\
621	0.513\\
622	0.495\\
623	0.481\\
624	0.503\\
625	0.482\\
626	0.506\\
627	0.49\\
628	0.51\\
629	0.492\\
630	0.509\\
631	0.495\\
632	0.508\\
633	0.496\\
634	0.518\\
635	0.502\\
636	0.52\\
637	0.508\\
638	0.525\\
639	0.513\\
640	0.499\\
641	0.515\\
642	0.499\\
643	0.521\\
644	0.502\\
645	0.522\\
646	0.502\\
647	0.521\\
648	0.508\\
649	0.52\\
650	0.504\\
651	0.525\\
652	0.496\\
653	0.523\\
654	0.506\\
655	0.533\\
656	0.513\\
657	0.529\\
658	0.509\\
659	0.493\\
660	0.517\\
661	0.505\\
662	0.524\\
663	0.513\\
664	0.536\\
665	0.52\\
666	0.544\\
667	0.528\\
668	0.546\\
669	0.53\\
670	0.546\\
671	0.534\\
672	0.546\\
673	0.533\\
674	0.549\\
675	0.539\\
676	0.527\\
677	0.546\\
678	0.533\\
679	0.556\\
680	0.539\\
681	0.555\\
682	0.533\\
683	0.555\\
684	0.536\\
685	0.549\\
686	0.535\\
687	0.562\\
688	0.548\\
689	0.57\\
690	0.549\\
691	0.572\\
692	0.552\\
693	0.573\\
694	0.555\\
695	0.541\\
696	0.56\\
697	0.55\\
698	0.566\\
699	0.553\\
700	0.57\\
701	0.555\\
702	0.571\\
703	0.545\\
704	0.569\\
705	0.549\\
706	0.572\\
707	0.563\\
708	0.578\\
709	0.565\\
710	0.582\\
711	0.571\\
712	0.558\\
713	0.573\\
714	0.559\\
715	0.577\\
716	0.558\\
717	0.576\\
718	0.562\\
719	0.58\\
720	0.562\\
721	0.581\\
722	0.57\\
723	0.583\\
724	0.569\\
725	0.583\\
726	0.568\\
727	0.588\\
728	0.578\\
729	0.594\\
730	0.58\\
731	0.562\\
732	0.58\\
733	0.57\\
734	0.584\\
735	0.572\\
736	0.586\\
737	0.572\\
738	0.591\\
739	0.58\\
740	0.595\\
741	0.578\\
742	0.595\\
743	0.589\\
744	0.606\\
745	0.59\\
746	0.606\\
747	0.592\\
748	0.608\\
749	0.597\\
750	0.581\\
751	0.597\\
752	0.581\\
753	0.601\\
754	0.594\\
755	0.608\\
756	0.595\\
757	0.608\\
758	0.596\\
759	0.607\\
760	0.595\\
761	0.608\\
762	0.596\\
763	0.61\\
764	0.596\\
765	0.606\\
766	0.596\\
767	0.611\\
768	0.599\\
769	0.587\\
770	0.605\\
771	0.594\\
772	0.616\\
773	0.599\\
774	0.62\\
775	0.606\\
776	0.618\\
777	0.606\\
778	0.62\\
779	0.611\\
780	0.625\\
781	0.609\\
782	0.627\\
783	0.611\\
784	0.631\\
785	0.613\\
786	0.635\\
787	0.619\\
788	0.609\\
789	0.623\\
790	0.611\\
791	0.625\\
792	0.613\\
793	0.624\\
794	0.611\\
795	0.625\\
796	0.613\\
797	0.628\\
798	0.616\\
799	0.629\\
800	0.612\\
801	0.63\\
802	0.618\\
803	0.632\\
804	0.62\\
805	0.636\\
806	0.622\\
807	0.639\\
808	0.624\\
809	0.613\\
810	0.632\\
811	0.615\\
812	0.629\\
813	0.614\\
814	0.631\\
815	0.618\\
816	0.634\\
817	0.62\\
818	0.633\\
819	0.621\\
820	0.635\\
821	0.622\\
822	0.642\\
823	0.628\\
824	0.647\\
825	0.63\\
826	0.647\\
827	0.636\\
828	0.62\\
829	0.641\\
830	0.632\\
831	0.647\\
832	0.635\\
833	0.655\\
834	0.643\\
835	0.651\\
836	0.642\\
837	0.654\\
838	0.641\\
839	0.654\\
840	0.64\\
841	0.655\\
842	0.644\\
843	0.65\\
844	0.64\\
845	0.649\\
846	0.637\\
847	0.656\\
848	0.643\\
849	0.633\\
850	0.646\\
851	0.64\\
852	0.656\\
853	0.64\\
854	0.655\\
855	0.646\\
856	0.659\\
857	0.648\\
858	0.659\\
859	0.654\\
860	0.665\\
861	0.656\\
862	0.668\\
863	0.66\\
864	0.672\\
865	0.662\\
866	0.673\\
867	0.665\\
868	0.654\\
869	0.669\\
870	0.657\\
871	0.669\\
872	0.66\\
873	0.67\\
874	0.661\\
875	0.671\\
876	0.66\\
877	0.671\\
878	0.663\\
879	0.675\\
880	0.665\\
881	0.681\\
882	0.665\\
883	0.676\\
884	0.671\\
885	0.684\\
886	0.672\\
887	0.685\\
888	0.672\\
889	0.658\\
890	0.667\\
891	0.656\\
892	0.666\\
893	0.658\\
894	0.672\\
895	0.659\\
896	0.67\\
897	0.659\\
898	0.676\\
899	0.663\\
900	0.676\\
901	0.668\\
902	0.679\\
903	0.666\\
904	0.68\\
905	0.67\\
906	0.682\\
907	0.669\\
908	0.685\\
909	0.678\\
910	0.666\\
911	0.679\\
912	0.668\\
913	0.678\\
914	0.671\\
915	0.684\\
916	0.673\\
917	0.679\\
918	0.671\\
919	0.68\\
920	0.673\\
921	0.681\\
922	0.672\\
923	0.686\\
924	0.677\\
925	0.686\\
926	0.683\\
927	0.692\\
928	0.688\\
929	0.678\\
930	0.684\\
931	0.679\\
932	0.686\\
933	0.68\\
934	0.693\\
935	0.678\\
936	0.686\\
937	0.681\\
938	0.695\\
939	0.684\\
940	0.699\\
941	0.689\\
942	0.7\\
943	0.689\\
944	0.699\\
945	0.69\\
946	0.702\\
947	0.693\\
948	0.708\\
949	0.696\\
950	0.689\\
951	0.701\\
952	0.69\\
953	0.704\\
954	0.69\\
955	0.704\\
956	0.697\\
957	0.711\\
958	0.705\\
959	0.714\\
960	0.704\\
961	0.712\\
962	0.698\\
963	0.713\\
964	0.702\\
965	0.714\\
966	0.702\\
967	0.718\\
968	0.699\\
969	0.717\\
970	0.703\\
971	0.69\\
972	0.705\\
973	0.695\\
974	0.71\\
975	0.699\\
976	0.715\\
977	0.702\\
978	0.722\\
979	0.705\\
980	0.726\\
981	0.712\\
982	0.725\\
983	0.713\\
984	0.724\\
985	0.717\\
986	0.73\\
987	0.722\\
988	0.736\\
989	0.719\\
990	0.733\\
991	0.723\\
992	0.735\\
993	0.723\\
994	0.71\\
995	0.724\\
996	0.71\\
997	0.725\\
998	0.713\\
999	0.727\\
1000	0.719\\
};
\addlegendentry{Bayes test}

\end{axis}
\end{tikzpicture}%
      \caption{Comparison of the rejection probability for the null and the Bayesian test when $\model_1$ is true.}
    \end{figure}
    \only<article>{However, both methods are able to reject the null hypothesis more often when it is false, as long as we have more data.}
  }
\end{frame}
\begin{frame}
  \frametitle{Further reading}
  \begin{block}{Points of significance (Nature Methods)}
    \begin{itemize}
    \item Importance of being uncertain \url{https://www.nature.com/articles/nmeth.2613}
    \item Error bars \url{https://www.nature.com/articles/nmeth.2659}
    \item P values and the search for significance \url{https://www.nature.com/articles/nmeth.4120}
    \item Bayes' theorem \url{https://www.nature.com/articles/nmeth.3335}
    \item Sampling distributions and the bootstrap \url{https://www.nature.com/articles/nmeth.3414}
    \end{itemize}
  \end{block}
\end{frame}



%%% Local Variables:
%%% mode: latex
%%% TeX-master: "notes"
%%% End:
