\section{Naive Bayes classifiers}
\only<presentation>{
  \begin{frame}
    \tableofcontents[ 
    currentsection, 
    hideothersubsections, 
    sectionstyle=show/shaded
    ] 
  \end{frame}
}

\begin{frame}
  \only<article>{ One special case of this idea is in classification,
    when each hypothesis corresponds to a specific class. Then, given
    a new example vector of data $\bx$, we would like to calculate the
    probability of different classes $C$ given the data,
    $\Pr(C \mid \bx)$. So here, the class is the hypothesis.

    From Bayes's theorem, we see that we can write this as }
  \[
    \Pr(C \mid \bx) = \frac{\Pr(\bx \mid C) \Pr(C)}{\sum_{i} \Pr(\bx
      \mid C_i) \Pr(C_i)}
  \]
  \only<article>{ for any class $C$. This directly gives us a method
    for classifying new data, as long as we have a way to obtain
    $\Pr(\bx \mid C)$ and $\Pr(C)$.  }

  But should we use for the probability model $\Pr$?

  \subsubsection{Naive Bayes classifier}
  \only<article>{Naive Bayes classifiers are one of the simplest classification methods. They can have a full Bayesian interpretation under some assumptions, but otherwise they are too simplistic to be useful.}

  \begin{block}{Calculating the prior probability of classes}
    A simple method is to simply count the number of times each class
    appears in the training data $\Training = ((x_t,
    y_t))_{t=1}^T$. Then we can set
    \[
      \Pr(C) = 1/T \sum_{t=1}^T \ind{y_t = C}
    \]
  \end{block}

  \only<article>{ The Naive Bayes classifier uses the following model
    for observations, where observations are independent of each other
    given the class. Thus, for example the result of three different
    tests for lung cancer (stethoscope, radiography and biopsy) only
    depend on whether you have cancer, and not on each other.  }
  \begin{block}{Probability model for observations}
    \[
      \Pr(\bx \mid C) = \Pr(x(1), \ldots, x(n) \mid C) = \prod_{k=1}^n
      \Pr(x(k) \mid C).
    \]
  \end{block}

\end{frame}

\begin{frame}
  \only<article>{There are two different types of models we can have,
    one of which is mostly useful for continuous attributes and the
    other for discrete attributes.  In the first, we just need to
    count the number of times each feature takes different values in
    different classes.  }
  \begin{block}{Discrete attribute model.}
    \only<article>{Here we simply count the average number of times
      that the attribute $k$ had the value $i$ when the label was
      $C$. This is in fact analogous to the conditional probability
      definition.}
    \[
      \Pr(x(k) = i \mid C) = \frac{\sum_{t=1}^T \ind{x_t(k) = i \wedge
          y_t = C}} {\sum_{t=1}^T \ind{y_t = C}} = \frac{N_k(i,
        C)}{N(C)},
    \]
    where $N_k(i, C)$ is the numb l l .er of examples in class $C$
    whose $k$-th attribute has the value $i$, and $N(C)$ is the number
    of examples in class $C$.
  \end{block}

  \begin{alertblock}{Full Bayesian approach versus maximum likelihood}
    This estimation is simple maximum likelihood, as it  does not maintain a distribution over the parameters. 
  \end{alertblock}

  \only<article>{ Sometimes we need to be able to deal with cases
    where there are no examples at all of one class. In that case,
    that class would have probability zero. To get around this
    problem, we add ``fake observations'' to our data. This is called
    \emph{Laplace smoothing}.
    \begin{remark} In Laplace smoothing with constant $\lambda$, our
      probability model is
      \[
        \Pr(x(k) = i \mid C) = \frac{\sum_{t=1}^T \ind{x_t(k) = i \wedge
            y_t = C} + \lambda} {\sum_{t=1}^T \ind{y_t = C} + n_k
          \lambda} = \frac{N_k(i, C) + \lambda}{N(C) + n_k \lambda}.
      \]
      where $n_k$ is the number of values that the $k$-th attribute
      can take. This is necessary, because we want
      $\sum_{i=1}^{n_k} \Pr(x(k) = i \mid C) = 1$. (You can check that
      this is indeed the case as a simple exercise).
    \end{remark}

    \begin{remark}
      In fact, the Laplace smoothing model corresponds to a so-called
      Dirichelt prior over polynomial parameters with a marginal
      probability of observation equal to the Laplace smoothing. This
      is an extension of Beta-Bernoulli example from binary outcomes
      to multiple outcomes.
    \end{remark}
  }
\end{frame}


\begin{frame}
  \begin{block}{Continuous attribute model.}
    \only<article>{Here we can use a Gaussian model for each
      continuous dimension.}
    \[
      \Pr(x(k) = v \mid C) = \frac{1}{\sigma \sqrt{2 \pi}} e^{\frac{(v -
          \mu)^2}{\sigma^2}},
    \]
    where $\mu$ and $\sigma$ are the mean and variance of the
    Gaussian, typically calculated from the training data as:
    \begin{align*}
      \mu &=   \frac{\sum_{t=1}^T x_t(k) \ind{y_t = C}}
            {\sum_{t=1}^T \ind{y_t = C}},
    \end{align*}
    i.e. $\mu$ is the mean of the $k$-th attribute when the label is
    $C$ and
    \begin{align*}
      \sigma &=   \frac{\sum_{t=1}^T [x_t(k) - \mu]^2 \ind{y_t = C}}
               {\sum_{t=1}^T \ind{y_t = C}},
    \end{align*}
    i.e. $\sigma$ is the variance of the $k$-th attribute when the
    label is $C$.  Sometimes we can just fix $\sigma$ to a constant
    value, i.e. $\sigma = 1$.
  \end{block}

  \begin{alertblock}{Full Bayesian approach}
    This estimation is simple maximum likelihood, as it selects a single parameter pair $\vectorsym{\mu} = (\mu_1, \ldots, \mu_n)$ and $\vectorsym{\sigma} = (\sigma_1, \ldots, \sigma_n)$ for every class and does not maintain a distribution over the parameters. \only<article>{It also assumes independence between the features. The full Bayesian approach considers an arbitrary covariance matrix $\matrixsym{\Sigma}$ and maintains a distribution $\bel(\vectorsym{\mu}, \matrixsym{\Sigma})$.}
  \end{alertblock}
\end{frame}

