\section{Graphical models}
\only<article>{
  Graphical models are a very useful tool for modelling the relationship between multiple variables. The simplest such models, probabilistic graphical models (otherwise known as Bayesian networks) involve directed acyclic graphs between random variables. There are two other types of probabilistic models, factor graph and undirected graphical models, which are equivalent to each other, though not to directed models.}
\begin{frame}
  \frametitle{Graphical models}
  \begin{figure}[H]
    \centering
    \begin{tikzpicture}
      \node[RV] at (2,0) (xi) {$x_3$};
      \node[RV] at (0,0) (xB) {$x_1$};
      \node[RV] at (1,1) (xD) {$x_2$};
      \draw[->] (xB) to (xD);
      \draw[->] (xD) to (xi);
      \draw[->] (xB) to (xi);
    \end{tikzpicture}
    \label{fig:bn}
    \caption{Graphical model (directed acyclic graph) for three variables.}
  \end{figure}
  \only<article>{Consider for example the model in Figure~\ref{fig:bn}. It involves three variables, $x_1, x_2, x_3$ and there are three arrows, which show how one variable depends on another. Simply put, if you think of each $x_k$ as a stochastic function, then $x_k$'s value only depends on the values of its parents, i.e. the nodes that are point to it. In this example, $x_1$ does not depend on any other variable, but the value of $x_2$ depends on the value of $x_1$. Such models are useful when we want to describe the joint probability distribution of all the variables in the collection.}
  \begin{block}{Joint probability}
    Let $\bx = (x_1, \ldots, x_n)$. Then $\bx : \Omega \to X$, $X = \prod_i X_i$ and:
    \[
    \Pr(\bx \in A) = P(\cset{\omega \in \Omega}{\bx(\omega) \in A}).
    \]
  \end{block}
  \only<article>{
    When $X_i$ are finite, we can typically write
    \[
    \Pr(\bx = \ba) = P(\cset{\omega \in \Omega}{\bx(\omega) = \ba}),
    \]
    for the probability that $x_i = a_i$ for all $i \in [n]$.
  }
  \begin{block}{Factorisation}
    \only<article>{
      For any subsets $B \subset [n]$ and its complement $C$ so that
      $\bx_B = (x_i)_{i \in B}$,     $\bx_C = (x_i)_{i \notin B}$
    }
    \only<1>{
      \[
      \Pr(\bx) = \Pr(\bx_B \mid \bx_C) \Pr(\bx_C)
      \only<presentation>{,\qquad B, C \subset [n]}
      \]
    }
    \uncover<2->{
      So we can write any joint distribution as
      \[
      \Pr(x_1) \Pr(x_2 \mid x_1) \Pr(x_3 \mid x_1, x_2) \cdots \Pr(x_n \mid x_1, \ldots, x_{n-1}).
      \]
    }
  \end{block}
  \only<article>{Although the above factorisation is always possible to do, sometimes our graphical model has a structure that makes the factors much simpler. In fact, the main reason for introducing graphical models is to represent dependencies between variables. For a given model, we can infer whether some variables are in fact dependent, independent, or conditionally independent.}
\end{frame}
\begin{frame}
  \frametitle{Directed graphical models and conditional independence}
  \begin{figure}[H]
    \centering
    \begin{tikzpicture}
      \node[RV] at (2,0) (xi) {$x_3$};
      \node[RV] at (0,0) (xB) {$x_1$};
      \node[RV] at (1,1) (xD) {$x_2$};
      \draw[->] (xB)--(xD);
      \draw[->] (xD)--(xi);
    \end{tikzpicture}
    \label{fig:bn}
    \caption{Graphical model for the factorisation $\Pr(x_3 \mid x_2) \Pr(x_2 \mid x_1) \Pr(x_1)$.}
  \end{figure}
  \begin{block}{Conditional independence}
    We say $x_i$ is conditionally independent of $\bx_B$ given $\bx_D$ and write $x_i \mid \bx_D \indep \bx_B$ iff
    \[
    \Pr(x_i, \bx_B \mid \bx_D)
    =
    \Pr(x_i \mid \bx_D)
    \Pr(\bx_B \mid \bx_D).
    \]
  \end{block}

  \frametitle{Directed graphical models}
  \only<article>{
    A graphical model is a convenient way to represent conditional independence between variables. There are many variants of graphical models, whose name is context dependent. Other names used in the literature are probabilistic graphical models, Bayesian networks, causal graphs, or decision diagrams. In this set of notes we focus on directed graphical models that depict dependencies between ranom variables.

    \begin{definition}[Directed graphical model] A collection of $n$ random variables $x_i : \Omega \to X_i$, and let $X \defn \prod_i X_i$, with underlying probability measure $P$ on $\Omega$.
      Let $\bx = (x_i)_{i=1}^n$ and for any subset $B \subset[n]$ let
      \begin{align}
        \bx_B &\defn (x_i)_{i \in B}\\
        \bx_{-j} &\defn (x_i)_{i \neq i}
      \end{align}
    \end{definition}
  }
  \only<article>{In a graphical model, conditional independence is represented through directed edges.}


\end{frame}

\begin{frame}
  \begin{example}[Smoking and lung cancer]
    \begin{figure}[H]
      \centering
      \begin{tikzpicture}
        \node[RV] at (0,0) (x1) {$S$};
        \node[RV] at (1,1) (x2) {$C$};
        \node[RV] at (2,0) (x3) {$A$};
        \draw[->] (x1)--(x2);
        \draw[->] (x3)--(x2);
      \end{tikzpicture}
      \caption{Smoking and lung cancer graphical model, where $S$: Smoking, $C$: cancer, $A$: asbestos exposure.}
    \end{figure}
    \only<article>{
      It has been found by~\citet{lee2001relation} that lung  incidence not only increases with both asbestos exposure and smoking. This is in agreement with the graphical model shown. The study actually found that there is an amplification effect, whereby smoking and asbestos exposure increases cancer risk by 28 times compared to non-smokers. This implies that the risk is not simply additive. The graphical model only tells us that there is a dependency, and does not describe the nature of this dependency precisely.}
  \end{example}
  \begin{block}{Explaining away}
    Even though $S, A$ are independent, they become dependent once you know $C$. \only<article>{For example, let us say we know that you have cancer and that our model says that it's very unlikely to have cancer unless you either smoke or are exposed to asbestos. When we also learn that you do not have asbestos exposure, smoking becomes more likely. In either words, if cancer is caused by either smoking or asbestos, and we rule out asbestos, the only remaining explanation is smoking. This is what is generally called \alert{explaining away.}}
  \end{block}
\end{frame}

\begin{frame}
  \begin{example}[Time of arrival at work]
    \begin{figure}[H]
      \centering
      \begin{tikzpicture}
        \node[RV] at (0,0) (x1) {$x_1$};
        \node[RV] at (1,1) (x2) {$T$};
        \node[RV] at (2,0) (x3) {$x_2$};
        \draw[->] (x2)--(x3);
        \draw[->] (x2)--(x1);
      \end{tikzpicture}
      \caption{Time of arrival at work graphical model where $T$ is a traffic jam and $x_1$ is the time John arrives at the office and $x_2$ is the time Jane arrives at the office.}
    \end{figure}
    \only<article>{
      In this model, the arrival times of John and Jane may seem correlated. However, there is a common cause: The existence of a traffic jam. Whenever there is a traffic jam, both John and Jane are usually late. Whenever there is not a traffic jam, they are both mostly on time.
    }
  \end{example}
  \begin{block}{Conditional independence}
    Even though $x_1, x_2$ are correlated, they become independent once you know $T$.
  \end{block}
\end{frame}

\begin{frame}
  \begin{example}[Treatment effects]
    \begin{figure}[H]
      \centering
      \begin{tikzpicture}
        \node[RV] at (0,0) (x) {$x$};
        \node[RV] at (1,1) (y) {$y$};
        \node[RV] at (2,0) (a) {$a$};
        \draw[->] (x)--(y);
        \draw[->] (x)--(a);
        \draw[->] (a)--(y);
      \end{tikzpicture}
      \caption{Kidney treatment model, where $x$: severity, $y$: result, $a$: treatment applied}
    \end{figure}
    \begin{table}[H]
      \begin{tabular}{l|r|r}
        & Treatment A  & Treatment B\\
        \hline
        Small stones & 87  & 270\\
        Large stones  & 263 &  80
      \end{tabular}
      \begin{tabular}{l|r|r}
        Severity & Treatment A  & Treatment B\\
        \hline
        Small stones ) & 93\%  & 87\%\\
        Large stones  & 73\% &  69\%\\
        \hline
        Average & 78\% & 83\%
      \end{tabular}
    \end{table}
    \only<article>{
      A curious example is that of applying one of two treatments for kidneys. In the data, it is clear that one treatment is best for both large and small stones. However, when the data is aggregated it appears as though treatment B is best. This is because treatment A is chosen much more frequently when the stones are large, and that's when both treatments perform worse.}
  \end{example}
\end{frame}


\begin{frame}
  \begin{example}[School admission]
    \begin{figure}[H]
      \centering
      \begin{subfigure}{0.45\textwidth}
        \begin{tikzpicture}
          \node[RV] at (0,0) (z) {$z$};
          \node[RV] at (1,1) (s) {$s$};
          \node[RV] at (2,0) (a) {$a$};
          \draw[->] (z)--(s);
          \draw[->] (z)--(s);
          \draw[->] (s)--(a);
        \end{tikzpicture}
      \end{subfigure}
      \begin{subfigure}{0.45\textwidth}
        \begin{tikzpicture}
          \node[RV] at (0,0) (z) {$z$};
          \node[RV] at (1,1) (s) {$s$};
          \node[RV] at (2,0) (a) {$a$};
          \draw[->] (z)--(s);
          \draw[->] (z)--(s);
          \draw[->] (s)--(a);
          \draw[->] (z)--(a);
        \end{tikzpicture}
      \end{subfigure}
      \caption{School admission graphical model, where $z$: gender, $s$: school applied to, $a$: whether you were admitted. }
    \end{figure}
    \begin{table}[H]
      \begin{tabular}{l|r|r}
        School & Male  & Female\\
        \hline
        A & 62\% & 82\%\\
        B & 63\% & 68\%\\
        C & 37\% & 34\%\\
        D & 33\% & 35\%\\
        E & 28\% & 24\%\\
        F &  6\% &  7\%\\
        \hline
        \emph{Average} & \emph{45\%} & \emph{38\%}
      \end{tabular}
    \end{table}
    \only<article>{In this example, it appears as though female candidates have a lower acceptance rate than males. However what is missing is the fact that many more males are applying to easier schools. Thus, it is possible that the data is explainable by the fact that admission only reflects the difficulty of each school, and the overall gender imbalance is due to the choices made by the applicants. However, an alternative model is that the admissions process also explicitly takes gender into account. However, both of these models may be inadequate, as we do not have data about each individual applicant, such as their grades. We shall discuss this issue further when we talk about causality, confounding variables and counterfactuals.}
  \end{example}
\end{frame}


\begin{frame}
  \begin{exercise}
    Factorise the following graphical model.
    \centering
    \begin{tikzpicture}
      \node[RV] at (0,0) (x1) {$x_1$};
      \node[RV] at (1,-1) (x2) {$x_2$};
      \node[RV] at (1,1) (x3) {$x_3$};
      \node[RV] at (2,0) (x4) {$x_4$};
      \draw[->] (x1)--(x2);
      \draw[->] (x1)--(x3);
    \end{tikzpicture}
  \end{exercise}
  \uncover<2>{
    \[
    \Pr(\bx)
    = \Pr(x_1) \Pr(x_2 \mid x_1) \Pr(x_3 \mid x_1) \Pr(x_4)
    \]
  }

\end{frame}

\begin{frame}
  \begin{exercise}
    Factorise the following graphical model.
    \centering
    \begin{tikzpicture}
      \node[RV] at (0,0) (x1) {$x_1$};
      \node[RV] at (1,-1) (x2) {$x_2$};
      \node[RV] at (1,1) (x3) {$x_3$};
      \node[RV] at (2,0) (x4) {$x_4$};
      \draw[->] (x1)--(x2);
      \draw[->] (x1)--(x3);
      \draw[->] (x3)--(x4);
    \end{tikzpicture}
  \end{exercise}
  \uncover<2>{
    \[
    Xb      \Pr(\bx)
    = \Pr(x_1) \Pr(x_2 \mid x_1) \Pr(x_3 \mid x_1) \Pr(x_4 \mid x_3)
    \]
  }

\end{frame}

\begin{frame}
  \begin{exercise}
    What dependencies does the following factorisation imply?
    \[
    \Pr(\bx)
    = \Pr(x_1) \Pr(x_2 \mid x_1) \Pr(x_3 \mid x_1) \Pr(x_4 \mid x_2, x_3)
    \]
    \centering
    \begin{tikzpicture}
      \node[RV] at (0,0) (x1) {$x_1$};
      \node[RV] at (1,-1) (x2) {$x_2$};
      \node[RV] at (1,1) (x3) {$x_3$};
      \node[RV] at (2,0) (x4) {$x_4$};
      \uncover<2>{
        \draw[->] (x1)--(x2);
        \draw[->] (x1)--(x3);
        \draw[->] (x3)--(x4);
        \draw[->] (x2)--(x4);
      }
    \end{tikzpicture}
  \end{exercise}
\end{frame}


%%% Local Variables:
%%% mode: latex
%%% TeX-master: "notes"
%%% End:



\begin{frame}
  \begin{alertblock}{Deciding conditional independence}
    There is an algorithm for deciding conditional independence of any two variables in a graphical model.
    \only<article>{However, this is beyond the scope of these notes. Here, we shall just use these models as a way to encode dependencies that we assume exist.}
  \end{alertblock}

\end{frame}

\subsection{Testing conditional independence}
\begin{frame}
  \frametitle{Measuring independence}
  \only<article>{The simplest way to measure independence is by looking at whether or not the distribution of the possibly dependent variable changes when we change the value of the other variables. }

  \begin{theorem}
    If $x_i \mid \bx_D \indep \bx_B$ then
    \[
    \Pr(x_i \mid \bx_B, \bx_D)
    =
    \Pr(x_i \mid \bx_D)
    \]
  \end{theorem}
  \uncover<2->{
    This implies
    \[
    \Pr(x_i \mid \bx_B = b, \bx_D)
    =
    \Pr(x_i \mid \bx_B = b', \bx_D)
    \]
    so we can measure independence by seeing how the distribution of $x_i$ changes when we vary $\bx_B$, keeping $\bx_D$ fixed.
  }
    \only<article>{For any given model, there is either a dependence or there is not. However, sometimes we might be able to tolerate some amount of dependence. Thus, we can simply measure the deviation from independence through a metric or divergence on distributions.}
  \begin{example}
    \[
    \|\Pr(a \mid y, z) - \Pr(a \mid y)\|_1
    \]
    which for discrete $a,y,z$ is:
    \[
    \max_{i,j} \|\Pr(a \mid y = i, z = j) - \Pr(a \mid y= i)\|_1
    =
    \max_{i,j} \|\sum_k \Pr(a = k \mid y = i, z = j) - \Pr(a =k \mid y= i)\|_1.
    \]
    See also \url{src/fairness/ci_test.py}
  \end{example}
\end{frame}

\begin{frame}
\frametitle{Coin tossing, revisited}
\begin{example}{The Beta-Bernoulli prior}
  \begin{figure}[H]
    \centering
    \begin{tikzpicture}
      \node[RV] at (0,0) (bel) {$\bel$}; \node[RV, hidden] at (1,0)
      (param) {$\param$}; \node[RV] at (2,0) (data) {$x$}; \draw[->]
      (bel)--(param); \draw[->] (param)--(data);
    \end{tikzpicture}
    \caption{Graphical model for a Beta-Bernoulli prior}
  \end{figure}
  \begin{align}
    \param &\sim \BetaDist(\bel_1, \bel_2), && \textrm{i.e.  $\bel$ are Beta distribution parameters}\\
    x \mid \param &\sim \Bernoulli(\param), && \textrm{i.e. $P_\param(x)$ is a Bernoulli}
  \end{align}
  \only<article>{In this example, it is obvious why we use the notation above for describing hierarchical models. We simply state what is the distribution on one variable conditioned on the other variables. Here, $\bel$ is fixed, and it is something we can choose arbitrarily. The data $x$ is observed, while the parameter $\param$ remains \emindex{latent}. Using Bayes theorem, we can derive the distribution for $\bel(\theta \mid x)$.}
\end{example}
\end{frame}

\begin{frame}
\begin{example}{An alternative model for coin-tossing}
  This is an elaboration of Example~\ref{ex:bayesian-compound-hypothesis-test} for hypothesis testing.
  \begin{figure}[H]
    \centering
    \begin{tikzpicture}
      \node[RV] at (-1,0) (hbel) {$\hyperparam$};
      \node[RV] at (1,1) (bel) {$\bel$};
      \node[RV, hidden] at (0,0) (model) {$\model$};
      \node[RV, hidden] at (1,0) (param) {$\param$};
      \node[RV] at (2,0) (data) {$x$};
      \draw[->] (hbel)--(model);
      \draw[->] (bel)--(param);
      \draw[->] (model)--(param);
      \draw[->] (param)--(data);
    \end{tikzpicture}
    \caption{Graphical model for a hierarchical prior}
  \end{figure}
  \begin{itemize}
  \item $\model_1$: A Beta-Bernoulli model with $\BetaDist(\bel_1, \bel_2)$
  \item $\model_0$: The coin is fair.
  \end{itemize}
  \begin{align}
    \param \mid \model = \model_0 &\sim \Singular(0.5), && \textrm{i.e. $\param$ is always 0.5}\\
    \param \mid \model = \model_1 &\sim \BetaDist(\bel_1, \bel_2), && \textrm{i.e. $\param$ has a Beta distribution}\\
    x \mid \param &\sim \Bernoulli(\param), && \textrm{i.e. $P_\param(x)$ is Bernoulli}
  \end{align}
  Here the posterior over the two models is simply
  \[
  \hyperparam(\model_0 \mid x) = \frac{P_{0.5}(x) \hyperparam(\model_0)}{P_{0.5}(x) \hyperparam(\model_0) + \Pr_{\model_1}(x) \hyperparam(\model_1)},
  \qquad
  \Pr_{\model_1(x)} = \int_0^1 P_\param(x) \dd \bel(\param).
  \]
\end{example}

\end{frame}

\begin{frame}
  \frametitle{Bayesian testing of independence}
  \only<article>{
    For a given distributional model $P_\param$, conditional independence either holds or does not. Consider the set of model parameters $\Param_0$ where, for each parameter $\param \in \Param_0$, we have a conditional independence condition, while $\Param_1$ may corresponds to models where there may not be independence. To make this more concrete, let's give an example.}

    \begin{figure}[H]
      \centering
      \begin{subfigure}[b]{0.45\textwidth}
        \begin{tikzpicture}
          \node[RV] at (0,0) (x1) {$x_1$};
          \node[RV] at (1,1) (x2) {$x_2$};
          \node[RV] at (2,0) (x3) {$x_3$};
          \draw[->] (x1)--(x2);
          \draw[->] (x2)--(x3);
        \end{tikzpicture}
        \caption{$\Param_0$ assumes independence}
      \end{subfigure}
     \begin{subfigure}[b]{0.45\textwidth}
     \begin{tikzpicture}
        \node[RV] at (0,0) (x1) {$x_1$};
        \node[RV] at (1,1) (x2) {$x_2$};
        \node[RV] at (2,0) (x3) {$x_3$};
        \draw[->] (x1)--(x2);
        \draw[->] (x2)--(x3);
        \draw[alert,->] (x1)--(x3);
      \end{tikzpicture}
      \caption{$\Param_1$ does \alert{not} assume independence}
    \end{subfigure}
    \only<article>{\caption{The two alternative models}}
    \end{figure}
  \begin{example}
    \only<1>{
    Assume data $D = \cset{x^t_1, x^t_2, x^t_3}{t = 1, \ldots, T}$ with $x_i^t \in \set{0,1}$.
    \only<article>{First consider model $\Param_0$ where the following conditional independence holds
    \[
    P_\param(x_3 \mid x_2, x_1) = P_\param(x_2 \mid x_1), \qquad \forall \param \in \Param_0.
    \]
    In the alternative model $\Param_1$ there is no
      independence assumption. So the likelihood for either a model in either set is}
    \begin{align}
    P_\param(D) &= \prod_t P_\param(x^t_3 \mid x^t_2) P_\param(x^t_2 \mid x^t_1) P_\param(x^t_1), \qquad \param \in \Param_0\\
    P_\param(D) &= \prod_t P_\param(x^t_3 \mid x^t_2, \alert{x_1^t}) P_\param(x^t_2 \mid x^t_1) P_\param(x^t_1), \qquad \param \in \Param_1
    \end{align}
  }
  \only<2>{
    \only<article>{The parameters for this example can be defined as follows}
    \begin{align}
      \param_1 &\defn P_\param(x^t_1 = 1) \tag{$\model_0,\model_1$}\\
      \param_{2|1}^{i} &\defn P_\param(x^t_2 = 1 \mid x^t_1 = i) \tag{$\model_0, \model_1$}\\
      \param_{3|2}^{j} &\defn P_\param(x^t_3 = 1 \mid x^t_2 = j) \tag{$\model_0$}\\
      \param_{3|2,1}^{i,j} &\defn P_\param(x^t_3 = 1\mid x^t_2=j, x_1^t=i)\tag{$\model_1$}
    \end{align}
    \only<article>{
      We model each one of these parameters with a separate Beta-Bernoulli distribution.
    }
  }
  \end{example}

\end{frame}




\begin{frame}
\only<article>{Given some data $D$, the Bayesian approach would involve specifying a hierarchical prior $\bel$ so that $\hyperparam(\model_1) = 1 - \hyperparam(\model_0)$ specifies a probability on the two model structures, while for the $i$-th model we define a prior $\bel_i(\param)$ over $\Param_i$, so that we obtain the following hierarchical model
  }
  \begin{figure}[H]
    \centering
    \begin{tikzpicture}
      \node[RV, hidden] at (0,0) (model) {$\model$};
      \node[RV, hidden] at (1,0) (param) {$\param$};
      \node[RV] at (2,0) (data) {$D$};
      \draw[->] (model)--(param);
      \draw[->] (param)--(data);
    \end{tikzpicture}
        \caption{Hierarchical model.}
    \label{fig:hierarchical-model}
  \end{figure}
\only<article>{Here the specific model $\model$ is unobserved, as well as its parameters $\param$. Only the data $D$ is observed. Our prior distribution is omitted from the graph.}
\only<1>{
\begin{align}
  \model_i &\sim \hyperparam\\
  \param \mid \model = \model_i &\sim \bel_i
\end{align}
}
\only<1->{
  \begin{block}{Marginal likelihood}
    \only<article>{This gives the the following marginal likelihood for the combined models and each of the models respectively.}
    \begin{align}
      \Pr_\hyperparam(D) &= \hyperparam(\model_0) \Pr_{\model_0}(D)  + \hyperparam(\model_1) \Pr_{\model_1}(D)\\
      \Pr_{\model_i}(D) &= \int_{\Param_{i}} P_{\param}(D) \dd \bel_i(\param).
    \end{align}
  \end{block}
}
\only<2->{
  \begin{block}{Model posterior}
    \begin{equation}
      \hyperparam(\model \mid D) = 
      \frac{\Pr_\model(D) \hyperparam(\model)}
      {\sum_{i} \Pr_{\model_i} (D) \hyperparam(\model_i)}
    \end{equation}
  \end{block}
}
\end{frame}

\begin{frame}
  \frametitle{Calculating the marginal likelihood}
  \only<article>{Generally speaking, calculating the marginal likelihood for a model with an uncountable parameter set is hard. However, conjugate models admit closed form solutions and efficient calculations. Firstly, let's rewrite the marginal likelihood in terms of an integral Monte-Carlo approximation.}
  \begin{block}{Monte-Carlo approximation}
    \begin{align}
      \label{eq:mc-marginal-likelihood}
      \int_\Param P_\param(D) \dd \bel(\param)
      &\approx
        \sum_{n=1}^N P_{\param_n}(D)  + O(1/\sqrt{N}), & \param_n &\sim \bel
    \end{align}
  \end{block}
  \only<article>{Even though this approximation is reasonable at first glance, the problem is that the leading constant of the error scales approximately proportionally to the maximum likelihood $\max_\param P_\param(D)$. Thus, the more data we have the more samples we need to get a good approximation with this simple Monte Carlo approach. For that reason, one typically uses a sample from a proposal distribution $\psi$ which is different from $\bel$. Then it holds}
  \begin{block}{Importance sampling}
    \begin{align}
      \label{eq:importance-sampling}
      \int_\Param P_\param(D) \dd \bel(\param)
      &
        \only<2>{
        =
        \int_\Param P_\param(D) \frac{\dd \psi(\param)}{\dd \psi(\param)} \dd \bel(\param)
        }
        \only<3>{
        =
        \int_\Param P_\param(D) \frac{\dd \bel(\param)}{\dd \psi(\param)} \dd \psi(\param)
        }
        \uncover<4>{
        \approx
        \sum_{n=1}^N P_\param(D) \frac{\dd \bel(\param_n)}{\dd \psi(\param_n)}, & \param_n \sim \psi
        }
    \end{align}
  \end{block}
\end{frame}

\begin{frame}
  \begin{block}{Sequential updating of the marginal likelihood}
    \begin{align}
      \label{eq:sequential-likelihood}
      \Pr_\bel(D)
      \uncover<2->{
      &= \Pr_\bel(x_1, \ldots, x_T) 
        }
      \uncover<3->{
      \\
      & = \Pr_\bel(x_2, \ldots, x_T \mid x_1) \Pr_\bel(x_1)
        }
      \uncover<4->{
      \\
      & = \alert{\prod_{t=1}^T \Pr_\bel(x_t \mid x_1, \ldots, x_{t-1})}
        }
        \uncover<5->{
      \\
      & = \prod_{t=1}^T \int_\Param P_{\param_n}(x_t) \dd \underbrace{\bel(\param \mid x_1, \ldots, x_{t-1})}_{\textrm{posterior at time $t$}}
        }
    \end{align}
    \only<article>{The nice thing about this break down is that for a simple model such as Beta-Bernoulli, the individual datapoint marginal likelihoods are easy to compute}
  \end{block}
  \begin{example}[Beta-Bernoulli]
    \[\Pr_\bel(x_t = 1 \mid x_1, \ldots, x_{t-1}) = \frac{\alpha_t}{\alpha_t + \beta_t},
    \]
    with
    $\alpha_t = \alpha_0 + \sum_{n=1}^{t-1} x_n, \quad \beta_t = \beta_0 + \sum_{n=1}^{t-1} (1 - x_n)$
  \end{example}
\end{frame}
\begin{frame}
  \frametitle{Further reading}
  \begin{block}{Python sources}
    \begin{itemize}
    \item A simple python measure of conditional independence \url{src/fairness/ci_test.py}
    \item A simple test for discrete Bayesian network \url{src/fairness/DirichletTest.py}
    \item Using the PyMC package \url{https://docs.pymc.io/notebooks/Bayes_factor.html} 
    \end{itemize}
  \end{block}
\end{frame}

%%% Local Variables:
%%% mode: latex
%%% TeX-engine: xetex
%%% TeX-master: "notes"
%%% End:
